\documentclass[12pt]{article}

%%%% GRAPHICS %%%%
\usepackage{tikz}
\usepackage[siunitx, american, RPvoltages]{circuitikz}
\usetikzlibrary{arrows.meta}
\usepackage{tikz-3dplot}
\usepackage{graphicx}
\usepackage{pgfplots}
  \pgfplotsset{compat=1.18}
\usetikzlibrary{arrows}
\newcommand{\midarrow}{\tikz \draw[-triangle 90] (0,0) -- +(.1,0);}

%%%% FIGURES %%%%
\usepackage{subcaption}
\usepackage{wrapfig}
\usepackage{float}
\usepackage[skip=5pt, font=footnotesize]{caption}

%%%% FORMATTING %%%%
\usepackage{parskip}
\usepackage{tcolorbox}
\usepackage{ulem}

%%%% TABLE FORMATTING %%%%
\usepackage{tabularray}
\UseTblrLibrary{booktabs}

%%%% MATH AND LOGIC %%%%
\usepackage{xifthen}
\usepackage{amsmath}
\usepackage{amssymb}
\usepackage{amsfonts}

%%%% TEXT AND SYMBOLS %%%%
\usepackage[T1]{fontenc}
\usepackage{textcomp}
\usepackage{gensymb}

%%%% OTHER %%%%
\usepackage{standalone}

%%%% LOGIC SYMBOLS %%%%
\newcommand*\xor{\oplus}

%%%% STYLES %%%%

% Packages
\usepackage[paper=letterpaper,tmargin=45pt,bmargin=45pt,lmargin=45pt,rmargin=45pt]{geometry}
\usepackage{titlesec}
\usepackage[rgb]{xcolor}
\selectcolormodel{natural}
\usepackage{ninecolors}
\selectcolormodel{rgb}

% Colors
\definecolor{pg}{HTML}{24273A}
\definecolor{fg}{HTML}{FFFFFF}
\definecolor{bg}{HTML}{24273A}
\definecolor{re}{HTML}{F38BA8}
\definecolor{gr}{HTML}{A6E3A1}
\definecolor{ye}{HTML}{F9E2AF}
\definecolor{or}{HTML}{FAB387}
\definecolor{bl}{HTML}{89B4FA}
\definecolor{ma}{HTML}{CBA6F7}
\definecolor{cy}{HTML}{94E2D5}
\definecolor{pi}{HTML}{F2CDCD}

\definecolor{copper}{HTML}{B87333}

\usepackage{nameref}
\makeatletter
\newcommand*{\currentname}{\@currentlabelname}
\makeatother

\titleformat{\section}
  {\normalfont\scshape\Large\bfseries}
  {\thesection}
  {0.75em}
  {}

\titleformat{\subsection}
  {\normalfont\scshape\large\bfseries}
  {\thesubsection}
  {0.75em}
  {}

\titleformat{\subsubsection}
  {\normalfont\scshape\normalsize\bfseries}
  {\thesubsubsection}
  {0.75em}
  {}

% Formula
\newcounter{formula}[section]
\newenvironment{formula}[1]{
  \stepcounter{formula}
  \begin{tcolorbox}[
    standard jigsaw, % Allows opacity
    colframe={fg},
    boxrule=1px,
    colback=bg,
    opacityback=0,
    sharp corners,
    sidebyside,
    righthand width=18px,
    coltext={fg}
  ]
  \centering
  \textbf{\uline{#1}}
}{
  \tcblower
  \textbf{\thesection.\theformula}
  \end{tcolorbox}
}

% Definition
\newcounter{definition}[section]

\newenvironment{definition*}[1]{
  \begin{tcolorbox}[
    standard jigsaw, % Allows opacity
    colframe={fg},
    boxrule=1px,
    colback=bg,
    opacityback=0,
    sharp corners,
    coltext={fg}
  ]
  \textbf{#1 \hfill}
  \vspace{5px}
  \hrule
  \vspace{5px}
  \noindent
}{
  \end{tcolorbox}
}

\newenvironment{definition}[1]{
  \stepcounter{definition}
  \begin{tcolorbox}[
    standard jigsaw, % Allows opacity
    colframe={fg},
    boxrule=1px,
    colback=bg,
    opacityback=0,
    sharp corners,
    coltext={fg}
  ]
  \textbf{#1 \hfill \thesection.\thedefinition}
  \vspace{5px}
  \hrule
  \vspace{5px}
  \noindent
}{
  \end{tcolorbox}
}

% Example Problem
\newcounter{example}[section]
\newenvironment{example}{
  \stepcounter{example}
  \begin{tcolorbox}[
    standard jigsaw, % Allows opacity
    colframe={fg},
    boxrule=1px,
    colback=bg,
    opacityback=0,
    sharp corners,
    coltext={fg}
  ]
  \textbf{Example \hfill \thesection.\theexample}
  \vspace{5px}
  \hrule
  \vspace{5px}
  \noindent
}{
  \end{tcolorbox}
}

\tikzset{
  cubeBorder/.style=fg,
  cubeFilling/.style={fg!20!bg, opacity=0.25},
  gridLine/.style={very thin, gray},
  graphLine/.style={-latex, thick, fg},
}

\pgfplotsset{
  basicAxis/.style={
    grid,
    major grid style={line width=.2pt,draw=fg!50!bg},
    axis lines = box,
    axis line style = {line width = 1px},
  }
}

%%%% REFERENCES %%%%
\usepackage{hyperref}
\hypersetup{
  colorlinks  = true,
  linkcolor   = bl,
  anchorcolor = bl,
  citecolor   = bl,
  filecolor   = bl,
  menucolor   = bl,
  runcolor    = bl,
  urlcolor    = bl,
}

\author{Ethan Anthony}
\newcommand*{\equal}{=}


\title{Lecture 007}
\date{February 25, 2025}

\begin{document}
\newpage
\section{Higher Order Linear ODE}
\label{sec:higherOrderLinearODE}

\subsection{General Theory}
\label{ssec:generalTheory}

\subsubsection{Homogeneous Case}
\label{sssec:homogeneousCase}

For homogeneous higher order linear ODEs:
\begin{equation*}
  y^{(n)} + p_1(t)y^{(n-1)} + \hdots + p_{n-1}(t)y' + p_n(t)y = 0
\end{equation*}
The fundamental set of solutions is a set of solutions ($y_1, \hdots, y_n$) that are:
\begin{itemize}
  \itemsep0em
  \item solutions to the ODE
  \item linearly independent (Subsection \ref{ssec:wronskianForLinearIndependence})
  \item can be summed, through the principle of superposition, to find the general solution:
    \begin{equation*}
      y = C_1y_1 + \hdots + C_ny_n
    \end{equation*}
\end{itemize}

\subsubsection{Non-Homogeneous Case}
\label{sssec:nonHomogeneousCase}

For non-homogeneous higher order linear ODEs:
\begin{equation*}
  y^{(n)} + p_1(t)y^{(n-1)} + \hdots + p_{n-1}(t)y' + p_n(t)y = g(t)
\end{equation*}
The structure of the general solution is $y = y_c + Y$:
\begin{itemize}
  \itemsep0em
  \item $y_c$ is the complementary solution (the general solution to the corresponding homogeneous ODE)
  \item $Y$ is a particular solution to the non-homogeneous ODE
\end{itemize}
Solving these higher-order linear ODEs is incredibly similar to the second order ODEs in Section \ref{sec:nonHomogeneousODE}.

\subsection{Constant Coefficients}
\label{ssec:constantCoefficients}

\subsubsection{Homogeneous Case}
\label{sssec:homogeneousCase2}

For the homogeneous ODE
\begin{equation*}
  y^{(n)} + p_1(t)y^{(n-1)} + \hdots + p_{n-1}(t)y' + p_n(t)y = 0
\end{equation*}
the auxiliary equation can be formulated as
\begin{equation*}
  a_nr^n + a_{n-1}y^{(n-1)} + \hdots + a_1y' + a_0y = 0
\end{equation*}
where $r$ is a solution to the auxiliary equation. If $r$ is:
\begin{itemize}
  \itemsep0em
  \item a real \textit{single} root, then it contributes \textit{one} function $e^{rt}$ to the fundamental set of solutions
  \item a real root \textit{repeated $m$ times}, then it contributes $m$ functions
    \begin{equation*}
      e^{rt}, te^{rt},\hdots,t^{m-1}e^{rt}
    \end{equation*}
    to the fundamental set of solutions
  \item a complex ($r=\alpha + \beta i$) \textit{single} root, then together with its conjugate ($r=\alpha - \beta i$), two functions are contributed the to fundamental set of solutions
    \begin{equation*}
      e^{\alpha t}\cos(\beta t),\ e^{\alpha t}\sin(\beta t)
    \end{equation*}
  \item a complex ($r=\alpha + \beta i$) root \textit{repeated $m$ times}, then together with its conjugate ($r=\alpha - \beta i$), $2m$ functions are contributed the to fundamental set of solutions
    \begin{equation*}
      e^{\alpha t}\cos(\beta t),\ e^{\alpha t}\sin(\beta t),\ te^{\alpha t}\cos(\beta t),\ te^{\alpha t}\sin(\beta t),\ \hdots\ ,\ t^{m-1}e^{\alpha t}\cos(\beta t),\ t^{m-1}e^{\alpha t}\sin(\beta t)
    \end{equation*}
\end{itemize}

\subsubsection{Non-Homogeneous Case}
\label{sssec:nonHomogeneousCase2}

For the ODE:
\begin{equation*}
  y^{(n)} + p_1(t)y^{(n-1)} + \hdots + p_{n-1}(t)y' + p_n(t)y = g(t)
\end{equation*}
To find the complementary solution to a non-homogeneous ODE, the same exact steps as in Subsubsection \ref{sssec:homogeneousCase2} can be taken with the associated homogeneous ODE.

Similarly, the particular solution can be found just as it was in Subsection \ref{ssec:ansatz}.

\end{document}
