\documentclass[12pt]{article}

\input{../../xlatex/imports/preamble}

\title{Lecture 007}
\date{February 25, 2025}

\begin{document}
\newpage
\section{Higher Order Linear ODE}
\label{sec:higherOrderLinearODE}

\subsection{General Theory}
\label{ssec:generalTheory}

\subsubsection{Homogeneous Case}
\label{sssec:homogeneousCase}

For homogeneous higher order linear ODEs:
\begin{equation*}
  y^{(n)} + p_1(t)y^{(n-1)} + \hdots + p_{n-1}(t)y' + p_n(t)y = 0
\end{equation*}
The fundamental set of solutions is a set of solutions ($y_1, \hdots, y_n$) that are:
\begin{itemize}
  \itemsep0em
  \item solutions to the ODE
  \item linearly independent (Subsection \ref{ssec:wronskianForLinearIndependence})
  \item can be summed, through the principle of superposition, to find the general solution:
    \begin{equation*}
      y = C_1y_1 + \hdots + C_ny_n
    \end{equation*}
\end{itemize}

\subsubsection{Non-Homogeneous Case}
\label{sssec:nonHomogeneousCase}

For non-homogeneous higher order linear ODEs:
\begin{equation*}
  y^{(n)} + p_1(t)y^{(n-1)} + \hdots + p_{n-1}(t)y' + p_n(t)y = g(t)
\end{equation*}
The structure of the general solution is $y = y_c + Y$:
\begin{itemize}
  \itemsep0em
  \item $y_c$ is the complementary solution (the general solution to the corresponding homogeneous ODE)
  \item $Y$ is a particular solution to the non-homogeneous ODE
\end{itemize}
Solving these higher-order linear ODEs is incredibly similar to the second order ODEs in Section \ref{sec:nonHomogeneousODE}.

\subsection{Constant Coefficients}
\label{ssec:constantCoefficients}

\subsubsection{Homogeneous Case}
\label{sssec:homogeneousCase2}

For the homogeneous ODE
\begin{equation*}
  y^{(n)} + p_1(t)y^{(n-1)} + \hdots + p_{n-1}(t)y' + p_n(t)y = 0
\end{equation*}
the auxiliary equation can be formulated as
\begin{equation*}
  a_nr^n + a_{n-1}y^{(n-1)} + \hdots + a_1y' + a_0y = 0
\end{equation*}
where $r$ is a solution to the auxiliary equation. If $r$ is:
\begin{itemize}
  \itemsep0em
  \item a real \textit{single} root, then it contributes \textit{one} function $e^{rt}$ to the fundamental set of solutions
  \item a real root \textit{repeated $m$ times}, then it contributes $m$ functions
    \begin{equation*}
      e^{rt}, te^{rt},\hdots,t^{m-1}e^{rt}
    \end{equation*}
    to the fundamental set of solutions
  \item a complex ($r=\alpha + \beta i$) \textit{single} root, then together with its conjugate ($r=\alpha - \beta i$), two functions are contributed the to fundamental set of solutions
    \begin{equation*}
      e^{\alpha t}\cos(\beta t),\ e^{\alpha t}\sin(\beta t)
    \end{equation*}
  \item a complex ($r=\alpha + \beta i$) root \textit{repeated $m$ times}, then together with its conjugate ($r=\alpha - \beta i$), $2m$ functions are contributed the to fundamental set of solutions
    \begin{equation*}
      e^{\alpha t}\cos(\beta t),\ e^{\alpha t}\sin(\beta t),\ te^{\alpha t}\cos(\beta t),\ te^{\alpha t}\sin(\beta t),\ \hdots\ ,\ t^{m-1}e^{\alpha t}\cos(\beta t),\ t^{m-1}e^{\alpha t}\sin(\beta t)
    \end{equation*}
\end{itemize}

\subsubsection{Non-Homogeneous Case}
\label{sssec:nonHomogeneousCase2}

For the ODE:
\begin{equation*}
  y^{(n)} + p_1(t)y^{(n-1)} + \hdots + p_{n-1}(t)y' + p_n(t)y = g(t)
\end{equation*}
To find the complementary solution to a non-homogeneous ODE, the same exact steps as in Subsubsection \ref{sssec:homogeneousCase2} can be taken with the associated homogeneous ODE.

Similarly, the particular solution can be found just as it was in Subsection \ref{ssec:ansatz}.

\end{document}
