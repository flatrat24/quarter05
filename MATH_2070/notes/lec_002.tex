\documentclass[12pt]{article}

%%%% GRAPHICS %%%%
\usepackage{tikz}
\usepackage[siunitx, american, RPvoltages]{circuitikz}
\usetikzlibrary{arrows.meta}
\usepackage{tikz-3dplot}
\usepackage{graphicx}
\usepackage{pgfplots}
  \pgfplotsset{compat=1.18}
\usetikzlibrary{arrows}
\newcommand{\midarrow}{\tikz \draw[-triangle 90] (0,0) -- +(.1,0);}

%%%% FIGURES %%%%
\usepackage{subcaption}
\usepackage{wrapfig}
\usepackage{float}
\usepackage[skip=5pt, font=footnotesize]{caption}

%%%% FORMATTING %%%%
\usepackage{parskip}
\usepackage{tcolorbox}
\usepackage{ulem}

%%%% TABLE FORMATTING %%%%
\usepackage{tabularray}
\UseTblrLibrary{booktabs}

%%%% MATH AND LOGIC %%%%
\usepackage{xifthen}
\usepackage{amsmath}
\usepackage{amssymb}
\usepackage{amsfonts}

%%%% TEXT AND SYMBOLS %%%%
\usepackage[T1]{fontenc}
\usepackage{textcomp}
\usepackage{gensymb}

%%%% OTHER %%%%
\usepackage{standalone}

%%%% LOGIC SYMBOLS %%%%
\newcommand*\xor{\oplus}

%%%% STYLES %%%%

% Packages
\usepackage[paper=letterpaper,tmargin=45pt,bmargin=45pt,lmargin=45pt,rmargin=45pt]{geometry}
\usepackage{titlesec}
\usepackage[rgb]{xcolor}
\selectcolormodel{natural}
\usepackage{ninecolors}
\selectcolormodel{rgb}

% Colors
\definecolor{pg}{HTML}{24273A}
\definecolor{fg}{HTML}{FFFFFF}
\definecolor{bg}{HTML}{24273A}
\definecolor{re}{HTML}{F38BA8}
\definecolor{gr}{HTML}{A6E3A1}
\definecolor{ye}{HTML}{F9E2AF}
\definecolor{or}{HTML}{FAB387}
\definecolor{bl}{HTML}{89B4FA}
\definecolor{ma}{HTML}{CBA6F7}
\definecolor{cy}{HTML}{94E2D5}
\definecolor{pi}{HTML}{F2CDCD}

\definecolor{copper}{HTML}{B87333}

\usepackage{nameref}
\makeatletter
\newcommand*{\currentname}{\@currentlabelname}
\makeatother

\titleformat{\section}
  {\normalfont\scshape\Large\bfseries}
  {\thesection}
  {0.75em}
  {}

\titleformat{\subsection}
  {\normalfont\scshape\large\bfseries}
  {\thesubsection}
  {0.75em}
  {}

\titleformat{\subsubsection}
  {\normalfont\scshape\normalsize\bfseries}
  {\thesubsubsection}
  {0.75em}
  {}

% Formula
\newcounter{formula}[section]
\newenvironment{formula}[1]{
  \stepcounter{formula}
  \begin{tcolorbox}[
    standard jigsaw, % Allows opacity
    colframe={fg},
    boxrule=1px,
    colback=bg,
    opacityback=0,
    sharp corners,
    sidebyside,
    righthand width=18px,
    coltext={fg}
  ]
  \centering
  \textbf{\uline{#1}}
}{
  \tcblower
  \textbf{\thesection.\theformula}
  \end{tcolorbox}
}

% Definition
\newcounter{definition}[section]

\newenvironment{definition*}[1]{
  \begin{tcolorbox}[
    standard jigsaw, % Allows opacity
    colframe={fg},
    boxrule=1px,
    colback=bg,
    opacityback=0,
    sharp corners,
    coltext={fg}
  ]
  \textbf{#1 \hfill}
  \vspace{5px}
  \hrule
  \vspace{5px}
  \noindent
}{
  \end{tcolorbox}
}

\newenvironment{definition}[1]{
  \stepcounter{definition}
  \begin{tcolorbox}[
    standard jigsaw, % Allows opacity
    colframe={fg},
    boxrule=1px,
    colback=bg,
    opacityback=0,
    sharp corners,
    coltext={fg}
  ]
  \textbf{#1 \hfill \thesection.\thedefinition}
  \vspace{5px}
  \hrule
  \vspace{5px}
  \noindent
}{
  \end{tcolorbox}
}

% Example Problem
\newcounter{example}[section]
\newenvironment{example}{
  \stepcounter{example}
  \begin{tcolorbox}[
    standard jigsaw, % Allows opacity
    colframe={fg},
    boxrule=1px,
    colback=bg,
    opacityback=0,
    sharp corners,
    coltext={fg}
  ]
  \textbf{Example \hfill \thesection.\theexample}
  \vspace{5px}
  \hrule
  \vspace{5px}
  \noindent
}{
  \end{tcolorbox}
}

\tikzset{
  cubeBorder/.style=fg,
  cubeFilling/.style={fg!20!bg, opacity=0.25},
  gridLine/.style={very thin, gray},
  graphLine/.style={-latex, thick, fg},
}

\pgfplotsset{
  basicAxis/.style={
    grid,
    major grid style={line width=.2pt,draw=fg!50!bg},
    axis lines = box,
    axis line style = {line width = 1px},
  }
}

%%%% REFERENCES %%%%
\usepackage{hyperref}
\hypersetup{
  colorlinks  = true,
  linkcolor   = bl,
  anchorcolor = bl,
  citecolor   = bl,
  filecolor   = bl,
  menucolor   = bl,
  runcolor    = bl,
  urlcolor    = bl,
}

\author{Ethan Anthony}
\newcommand*{\equal}{=}


\title{Lecture 002}
\date{January 12, 2025}

\begin{document}
\setcounter{equation}{0}
\newpage

\section{First Order Equations}
\label{sec:firstOrderEquations}

A first order ODE is an equation in the form of:
\begin{equation}
  \frac{dy}{dx} = f(x,y) \ \ \textup{or} \ \ y' = f(x,y)
  \label{eq:201}
\end{equation}
There is no strict process to find a solution to these equations. Thus, a lot of this class is spent on the various different ways to find solutions. To see one of the simpler ways, consider an equation where $f$ is a function of only $x$:
\begin{equation*}
  y' = f(x)
\end{equation*}
Integrating both sides with respect to $x$, the equation becomes:
\begin{align*}
  \int_{}^{} y' \, dx &= \int_{}^{} f(x) \, dx + C \\
  y(x) &= \int_{}^{} f(x) \, dx + C
\end{align*}
This $y(x)$ is the general solution to (\ref{eq:201}). Thus, to solve a differential equation with just a single dependent variable $x$, finding the antiderivative of $f(x)$ is sufficient to find the general solution.
\begin{example}
  Find the general solution for:
  \begin{equation*}
    y' = 3x^2
  \end{equation*}
  Integrating each side gives:
  \begin{align*}
    \int_{}^{} y' \, dx &= \int_{}^{} 3x^2 \, dx + C \\
    y(x) &= x^3 + C
  \end{align*}
  Thus, the general solution for $y' = 3x^2$ is $y(x) = x^3 + C$.
\end{example}
Generally, there will also be a condition that the solution to a differential equation must satisfy. In general terms, this might look like:
\begin{equation}
  y' = f(x), \ y(x_0) = y_0
  \label{eq:202}
\end{equation}
Leaving the solution to (\ref{eq:202}) as:
\begin{equation*}
  y(x) = \int_{x_0}^{x} f(t) \,dt + y_0
\end{equation*}
Verifying the solution, first $y'$ is computed based on our solution.
\begin{align*}
  \frac{d}{dx}\big(y(x)\big) = \frac{d}{dx}\left(\int_{x_0}^{x} f(t) \,dt + y_0\right) \\
  y' = f(x)
\end{align*}
Second, to verify that the initial condition is satisfied:
\begin{align*}
  y(x) &= \int_{x_0}^{x} f(t) \,dt + y_0 \\
  y(x_0) &= \int_{x_0}^{x_0} f(t) \,dt + y_0 \\
  y(x_0) &= 0 + y_0 \\
  y(x_0) &= y_0
\end{align*}
This confirms the initial condition, and thus it can be seen that the solution to the differential equation with an initial condition has been found.
\begin{example}
  Solve
  \begin{equation*}
    y' = e^{-x^2}, \ y({\color{gr} 0}) = {\color{re} 1}
  \end{equation*}
  First, finding the solution, both sides can be integrated:
  \begin{equation*}
    y(x) = \int_{{\color{gr} 0}}^{x} e^{-t^2} \,dt + {\color{re} 1}
  \end{equation*}
  And to verify the solution:
  \begin{figure}[H]
    \centering
    \begin{subfigure}[H]{0.45\textwidth}
      \centering
      \begin{align*}
        \frac{d}{dx}\big(y(x)\big) &= \frac{d}{dx}\left(\int_{0}^{x} e^{-t^2} \,dt + 1\right) \\
        y' &= e^{-x^2}
      \end{align*}
    \end{subfigure}
    \begin{subfigure}[H]{0.45\textwidth}
      \centering
      \begin{align*}
        y(0) &= \int_{0}^{x} e^{t^2} \,dt + 1 \\
        y(0) &= 0 + 1 \\
        y(0) &= 1
      \end{align*}
    \end{subfigure}
  \end{figure}
  The solution passes both verification tests, and so it can be safely said that the solution has been found.
\end{example}

Using the same method as before, equations of the form in (\ref{eq:203}) can be solved as well.
\begin{equation}
  y' = f(y)\ \ \textup{or}\ \ \frac{dy}{dx} = f(y)
  \label{eq:203}
\end{equation}
(\ref{eq:203}) can be rewritten using the inverse function theorem from calculus to switch the roles of $x$ and $y$ to get:
\begin{equation*}
  \frac{dx}{dy} = \frac{1}{f(y)}
\end{equation*}
Finally, at this point both sides can be integrated with respect to $y$ to get:
\begin{equation*}
  x(y) = \int_{}^{} \frac{1}{f(y)} \,dy + C
\end{equation*}
From here, it is just a matter of solving for $y$.
\begin{example}
  In Subsection \ref{ssec:fourFundamentalEquations}, the claim was made that the solution for $y'=ky$ is $y=Ce^{kx}$ for $k>0$. To show this, the method of integration can be used:
  \begin{gather*}
    \frac{dy}{dx} = ky \rightarrow \frac{dx}{dy} = \frac{1}{ky} \\
    x(y) = \int_{}^{} \frac{1}{ky} \, dy \\
    x(y) = \frac{1}{k}\ln|y| + D \\
  \end{gather*}
  Now solving for $y$:
  \begin{gather*}
    x(y) = \frac{1}{k}\ln|y| + D \\
  \end{gather*}
\end{example}

\end{document}
