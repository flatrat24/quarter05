\documentclass[12pt]{article}

%%%% GRAPHICS %%%%
\usepackage{tikz}
\usepackage[siunitx, american, RPvoltages]{circuitikz}
\usetikzlibrary{arrows.meta}
\usepackage{tikz-3dplot}
\usepackage{graphicx}
\usepackage{pgfplots}
  \pgfplotsset{compat=1.18}
\usetikzlibrary{arrows}
\newcommand{\midarrow}{\tikz \draw[-triangle 90] (0,0) -- +(.1,0);}

%%%% FIGURES %%%%
\usepackage{subcaption}
\usepackage{wrapfig}
\usepackage{float}
\usepackage[skip=5pt, font=footnotesize]{caption}

%%%% FORMATTING %%%%
\usepackage{parskip}
\usepackage{tcolorbox}
\usepackage{ulem}

%%%% TABLE FORMATTING %%%%
\usepackage{tabularray}
\UseTblrLibrary{booktabs}

%%%% MATH AND LOGIC %%%%
\usepackage{xifthen}
\usepackage{amsmath}
\usepackage{amssymb}
\usepackage{amsfonts}

%%%% TEXT AND SYMBOLS %%%%
\usepackage[T1]{fontenc}
\usepackage{textcomp}
\usepackage{gensymb}

%%%% OTHER %%%%
\usepackage{standalone}

%%%% LOGIC SYMBOLS %%%%
\newcommand*\xor{\oplus}

%%%% STYLES %%%%

% Packages
\usepackage[paper=letterpaper,tmargin=45pt,bmargin=45pt,lmargin=45pt,rmargin=45pt]{geometry}
\usepackage{titlesec}
\usepackage[rgb]{xcolor}
\selectcolormodel{natural}
\usepackage{ninecolors}
\selectcolormodel{rgb}

% Colors
\definecolor{pg}{HTML}{24273A}
\definecolor{fg}{HTML}{FFFFFF}
\definecolor{bg}{HTML}{24273A}
\definecolor{re}{HTML}{F38BA8}
\definecolor{gr}{HTML}{A6E3A1}
\definecolor{ye}{HTML}{F9E2AF}
\definecolor{or}{HTML}{FAB387}
\definecolor{bl}{HTML}{89B4FA}
\definecolor{ma}{HTML}{CBA6F7}
\definecolor{cy}{HTML}{94E2D5}
\definecolor{pi}{HTML}{F2CDCD}

\definecolor{copper}{HTML}{B87333}

\usepackage{nameref}
\makeatletter
\newcommand*{\currentname}{\@currentlabelname}
\makeatother

\titleformat{\section}
  {\normalfont\scshape\Large\bfseries}
  {\thesection}
  {0.75em}
  {}

\titleformat{\subsection}
  {\normalfont\scshape\large\bfseries}
  {\thesubsection}
  {0.75em}
  {}

\titleformat{\subsubsection}
  {\normalfont\scshape\normalsize\bfseries}
  {\thesubsubsection}
  {0.75em}
  {}

% Formula
\newcounter{formula}[section]
\newenvironment{formula}[1]{
  \stepcounter{formula}
  \begin{tcolorbox}[
    standard jigsaw, % Allows opacity
    colframe={fg},
    boxrule=1px,
    colback=bg,
    opacityback=0,
    sharp corners,
    sidebyside,
    righthand width=18px,
    coltext={fg}
  ]
  \centering
  \textbf{\uline{#1}}
}{
  \tcblower
  \textbf{\thesection.\theformula}
  \end{tcolorbox}
}

% Definition
\newcounter{definition}[section]

\newenvironment{definition*}[1]{
  \begin{tcolorbox}[
    standard jigsaw, % Allows opacity
    colframe={fg},
    boxrule=1px,
    colback=bg,
    opacityback=0,
    sharp corners,
    coltext={fg}
  ]
  \textbf{#1 \hfill}
  \vspace{5px}
  \hrule
  \vspace{5px}
  \noindent
}{
  \end{tcolorbox}
}

\newenvironment{definition}[1]{
  \stepcounter{definition}
  \begin{tcolorbox}[
    standard jigsaw, % Allows opacity
    colframe={fg},
    boxrule=1px,
    colback=bg,
    opacityback=0,
    sharp corners,
    coltext={fg}
  ]
  \textbf{#1 \hfill \thesection.\thedefinition}
  \vspace{5px}
  \hrule
  \vspace{5px}
  \noindent
}{
  \end{tcolorbox}
}

% Example Problem
\newcounter{example}[section]
\newenvironment{example}{
  \stepcounter{example}
  \begin{tcolorbox}[
    standard jigsaw, % Allows opacity
    colframe={fg},
    boxrule=1px,
    colback=bg,
    opacityback=0,
    sharp corners,
    coltext={fg}
  ]
  \textbf{Example \hfill \thesection.\theexample}
  \vspace{5px}
  \hrule
  \vspace{5px}
  \noindent
}{
  \end{tcolorbox}
}

\tikzset{
  cubeBorder/.style=fg,
  cubeFilling/.style={fg!20!bg, opacity=0.25},
  gridLine/.style={very thin, gray},
  graphLine/.style={-latex, thick, fg},
}

\pgfplotsset{
  basicAxis/.style={
    grid,
    major grid style={line width=.2pt,draw=fg!50!bg},
    axis lines = box,
    axis line style = {line width = 1px},
  }
}

%%%% REFERENCES %%%%
\usepackage{hyperref}
\hypersetup{
  colorlinks  = true,
  linkcolor   = bl,
  anchorcolor = bl,
  citecolor   = bl,
  filecolor   = bl,
  menucolor   = bl,
  runcolor    = bl,
  urlcolor    = bl,
}

\author{Ethan Anthony}
\newcommand*{\equal}{=}


\title{Lecture 002}
\date{January 12, 2025}

\begin{document}
\setcounter{equation}{0}
\newpage

\section{Ordinary Differential Equations}
\label{sec:ordinaryDifferentialEquations}

Ordinary differential equations (ODE) are differential equations with just a single input, generally thought of as time ($t$).

Consider the relationships between {\color{re} position} ($x$), {\color{gr} velocity} ($v$), and {\color{bl} acceleration} ($a$). Velocity is the derivative of position and acceleration the derivative of velocity.

\begin{figure}[H]
  \centering
  \includestandalone{figures/fig_001}
  \caption{Position, Velocity, and Acceleration}
  \label{fig:001}
\end{figure}

If only the acceleration of an object across time is known ($g = 9.8 \frac{m}{s^2}$), and nothing else about that object is known, then differential equations can be used to solve for the object's velocity and acceleration.
\begin{align*}
  y''(t) &= -g \\
  \frac{d(?)}{dt}(t) &= -g
\end{align*}
Based on this, if a function can be found to have a derivative of $-g$ then the velocity of the object can be said to have a velocity equal to that function. In this example, it is as simple as integrating the function for acceleration:
\begin{equation*}
  \frac{d(-gt + v_0)}{dt}(t) = -g
\end{equation*}
Going one step further and integrating the velocity will yield the position of the object.
\begin{equation*}
  \frac{d\left(-\frac{1}{2}gt^2+v_0t+x_0\right)}{dt}(t) = -gt + v_0
\end{equation*}
The last thing to consider with this example are the initial conditions of the differential equation. $v_0$ and $x_0$ are not known quantities, however, if they were to be specified then the differential would have an initial condition to satisfy.

\subsection{First Order Linear ODE}
\label{ssec:firstOrderLinearODE}

Before approaching First Order ODEs in Subsection \ref{ssec:firstOrderODE}, First Order Linear ODEs follow a simple pattern for their solutions:

If a differential equation is given in \textbf{standard form}:
\begin{equation*}
  y' + p(t)y = g(t)
\end{equation*}
Where $p(t)$ and $g(t)$ are arbitrary functions of $t$, then the general solution can be expressed as:
\begin{equation*}
  y(t) = \frac{\int_{}^{} \mu(t)g(t) \,dt + C}{\mu(t)}
\end{equation*}
Where the \textbf{integrating factor} ($\mu(t)$) is:
\begin{equation*}
  \mu(t) = e^{\int_{}^{} p(t) \,dt}
\end{equation*}

\begin{example}
  \begin{equation*}
    y' + 2y = e^{3t},\ y(0) = 3
  \end{equation*}
  This ODE is already in standard form, so the integrating factor can be written as:
  \begin{align*}
    \mu(t) &= e^{\int_{}^{} 2 \,dt} \\
           &= e^{2t}
  \end{align*}
  And thus the general solutions is:
  \begin{align*}
    y(t) &= \frac{\int_{}^{} e^{2t} \cdot e^{3t} \,dt + C}{e^{2t}} \\
         &= \frac{\int_{}^{} e^{5t} \,dt + C}{e^{2t}} \\
         &= \frac{\frac{1}{5} e^{5t} + C}{e^{2t}} \\
         &= \frac{1}{5} \frac{e^{5t}}{e^{2t}} + \frac{C}{e^{2t}} \\
         &= \frac{1}{5} \cdot e^{3t} + \frac{C}{e^{2t}} \\
  \end{align*}
  Using the initial condition to find the specific solution:
  \begin{align*}
    y(t) &= \frac{1}{5} \cdot e^{3t} + \frac{C}{e^{2t}} \\
    y(0) &= \frac{1}{5} \cdot e^{3 \cdot 0} + \frac{C}{e^{2 \cdot 0}} \\
    3    &= \frac{1}{5} + C \\
    \frac{14}{5} &= C
  \end{align*}
  Thus, the solution of the IVP is:
  \begin{equation*}
    y(t) = \frac{1}{5} \cdot e^{3t} + \frac{14}{15} \cdot e^{-2t} \\
  \end{equation*}
\end{example}

\subsection{First Order ODE}
\label{ssec:firstOrderODE}

A first order ODE is an equation in the form of:
\begin{equation}
  \frac{dy}{dx} = f(x,y) \ \ \textup{or} \ \ y' = f(x,y)
  \label{eq:201}
\end{equation}
There is no strict process to find a solution to these equations. Thus, a lot of this class is spent on the various different ways to find solutions. To see one of the simpler ways, consider an equation where $f$ is a function of only $x$:
\begin{equation*}
  y' = f(x)
\end{equation*}
Integrating both sides with respect to $x$, the equation becomes:
\begin{align*}
  \int_{}^{} y' \, dx &= \int_{}^{} f(x) \, dx + C \\
  y(x) &= \int_{}^{} f(x) \, dx + C
\end{align*}
This $y(x)$ is the general solution to (\ref{eq:201}). Thus, to solve a differential equation with just a single dependent variable $x$, finding the antiderivative of $f(x)$ is sufficient to find the general solution.
\begin{example}
  Find the general solution for:
  \begin{equation*}
    y' = 3x^2
  \end{equation*}
  Integrating each side gives:
  \begin{align*}
    \int_{}^{} y' \, dx &= \int_{}^{} 3x^2 \, dx + C \\
    y(x) &= x^3 + C
  \end{align*}
  Thus, the general solution for $y' = 3x^2$ is $y(x) = x^3 + C$.
\end{example}
Generally, there will also be a condition that the solution to a differential equation must satisfy. In general terms, this might look like:
\begin{equation}
  y' = f(x), \ y(x_0) = y_0
  \label{eq:202}
\end{equation}
Leaving the solution to (\ref{eq:202}) as:
\begin{equation*}
  y(x) = \int_{x_0}^{x} f(t) \,dt + y_0
\end{equation*}
Verifying the solution, first $y'$ is computed based on our solution.
\begin{align*}
  \frac{d}{dx}\big(y(x)\big) = \frac{d}{dx}\left(\int_{x_0}^{x} f(t) \,dt + y_0\right) \\
  y' = f(x)
\end{align*}
Second, to verify that the initial condition is satisfied:
\begin{align*}
  y(x) &= \int_{x_0}^{x} f(t) \,dt + y_0 \\
  y(x_0) &= \int_{x_0}^{x_0} f(t) \,dt + y_0 \\
  y(x_0) &= 0 + y_0 \\
  y(x_0) &= y_0
\end{align*}
This confirms the initial condition, and thus it can be seen that the solution to the differential equation with an initial condition has been found.
\begin{example}
  Solve
  \begin{equation*}
    y' = e^{-x^2}, \ y({\color{gr} 0}) = {\color{re} 1}
  \end{equation*}
  First, finding the solution, both sides can be integrated:
  \begin{equation*}
    y(x) = \int_{{\color{gr} 0}}^{x} e^{-t^2} \,dt + {\color{re} 1}
  \end{equation*}
  And to verify the solution:
  \begin{figure}[H]
    \centering
    \begin{subfigure}[H]{0.45\textwidth}
      \centering
      \begin{align*}
        \frac{d}{dx}\big(y(x)\big) &= \frac{d}{dx}\left(\int_{0}^{x} e^{-t^2} \,dt + 1\right) \\
        y' &= e^{-x^2}
      \end{align*}
    \end{subfigure}
    \begin{subfigure}[H]{0.45\textwidth}
      \centering
      \begin{align*}
        y(0) &= \int_{0}^{x} e^{t^2} \,dt + 1 \\
        y(0) &= 0 + 1 \\
        y(0) &= 1
      \end{align*}
    \end{subfigure}
  \end{figure}
  The solution passes both verification tests, and so it can be safely said that the solution has been found.
\end{example}

Using the same method as before, equations of the form in (\ref{eq:203}) can be solved as well.
\begin{equation}
  y' = f(y)\ \ \textup{or}\ \ \frac{dy}{dx} = f(y)
  \label{eq:203}
\end{equation}
(\ref{eq:203}) can be rewritten using the inverse function theorem from calculus to switch the roles of $x$ and $y$ to get:
\begin{equation*}
  \frac{dx}{dy} = \frac{1}{f(y)}
\end{equation*}
Finally, at this point both sides can be integrated with respect to $y$ to get:
\begin{equation*}
  x(y) = \int_{}^{} \frac{1}{f(y)} \,dy + C
\end{equation*}
From here, it is just a matter of solving for $y$.
\begin{example}
  In Subsection \ref{ssec:fourFundamentalEquations}, the claim was made that the solution for $y'=ky$ is $y=Ce^{kx}$ for $k>0$. To show this, the method of integration can be used:
  \begin{gather*}
    \frac{dy}{dx} = ky \rightarrow \frac{dx}{dy} = \frac{1}{ky} \\
    x(y) = \int_{}^{} \frac{1}{ky} \, dy \\
    x(y) = \frac{1}{k}\ln|y| + D \\
  \end{gather*}
  Now solving for $y$:
  \begin{gather*}
    x(y) = \frac{1}{k}\ln|y| + D \\
  \end{gather*}
\end{example}

\begin{example}
  \begin{equation*}
    y' + y = \cos(2t)
  \end{equation*}
  This is already in standard form, so we can go on to calculate the integrating factor:
  \begin{equation*}
    \mu(t) = e^{\int_{}^{} 1 \,dt} = e^t
  \end{equation*}
  Thus, the general solution will be:
  \begin{equation*}
    y = \frac{\int_{}^{} e^t \cdot \cos(2t) \,dt}{e^t}
  \end{equation*}
  Integrating the numerator:
  \begin{gather*}
    \int_{}^{} e^t \cdot \cos(2t) \,dt \\
    \vdots \\
    \int_{}^{} e^t \cdot \cos(2t) \,dt = \frac{1}{5}\left(e^t \cos(2t) + 2e^t \sin(2t)\right) + C
  \end{gather*}
  Thus:
  \begin{equation*}
    y = \frac{\frac{1}{5}\left(e^t \cos(2t) + 2e^t \sin(2t)\right) + C}{e^t} \\
  \end{equation*}
  or:
  \begin{equation*}
    y = \frac{1}{5}\left(\cos(2t) + 2\sin(2t)\right) + Ce^{-t}
  \end{equation*}
\end{example}

\subsection{Separable ODE}
\label{ssec:separableODE}

The basic form of a separable ODE is:
\begin{equation*}
  \frac{dy}{dx} = f(y) \cdot g(x)
\end{equation*}
i.e., the derivative is a product of two functions, one depending on $x$ and the other on $y$.

This type of ODE is solved first by separating the variables:
\begin{equation*}
  \frac{dy}{f(y)} = g(x) dx
\end{equation*}
Now that there is a single type of variable on each side (only $x$ or only $y$), both sides can be integrated.

\begin{example}
  \begin{equation*}
    y' = xy
  \end{equation*}
  This equation can be rewritten as:
  \begin{equation*}
    \frac{dy}{dx} = x \cdot y
  \end{equation*}
  Here, it can be easily seen that this equation is separable:
  \begin{align*}
    \frac{dy}{y} &= x\ dx \\
    \int_{}^{} \frac{1}{y} \, dy &= \int_{}^{} x \, dx + C \\
    \ln|y| &= \frac{x^2}{2} + C \\
    e^{\ln|y|} &= e^{\frac{x^2}{2} + C} \\
    |y| &= De^{\frac{x^2}{2}}
  \end{align*}
  Where $D>0$. Since $y=0$ is a solution as well, this can be simplified to:
  \begin{equation*}
    y = De^{\frac{x^2}{2}}
  \end{equation*}
\end{example}

\textbf{Principle:} If an ODE is given without initial values, a one-parameter family of solutions is sufficient. However, if an initial value is given (IVP), an explicit solution might be possible with an interval of existence.

\subsection{Implicit and Explicit Solutions}
\label{ssec:implicitAndExplicitSolutions}

In general, anytime an ODE is given \textit{without} initial values, then a single-parameter family of \textit{implicit} solutions is sufficient.

If given an IVP, however, then finding an \textit{explicit} solution and an \textit{interval of existence} should be attempted. In some situations, this won't be possible, but it should be attempted at least.

\subsubsection{Implicit Solutions}
\label{sssec:implicitSolution}

Sometimes a wall is reached even if the integration is possible. Consider:
\begin{equation}
  y' = \frac{xy}{y^2 + 1}
  \label{eq:204}
\end{equation}
Using the technique described in Subsection \ref{ssec:separableODE}, this can be separated into:

\begin{align*}
  y' &= \frac{xy}{y^2 + 1} \\
  \frac{dy}{dx} &= \frac{xy}{y^2 + 1} \\
  \frac{y^2 + 1}{y}dy  &= x\ dx \\
  \left(y+\frac{1}{y}\right)dy  &= x\ dx
\end{align*}
Integrating both sides gives:
\begin{equation*}
  \frac{y^2}{2} + \ln|y| = \frac{x^2}{2} + C
\end{equation*}
The integration of this equation is quite simple. However, try to solve for $y$ and see how difficult that will be. Though solving for $y$ itself is too difficult, this form is still a solution and can still be verified.
\begin{align*}
  \frac{d}{dx}\left(\frac{y^2}{2} + \ln|y|\right) &= \frac{d}{dx}\left(\frac{x^2}{2} + C\right) \\
  y'\left(y + \frac{1}{y}\right) &= x \\
  y \cdot \left(y'\left(y + \frac{1}{y}\right) \right) &= y \cdot x \\
  y'\left(y^2 + 1\right) &= y \cdot x \\
  y' &= \frac{xy}{y^2 + 1}
\end{align*}
Producing the exact same equation in (\ref{eq:204}), thus verifying the solution.

Since these solutions are implicit, they might not be able to be graphed as a valid function. In those cases, other information, such as an initial condition, can be used to further inform the appropriate solution.

\subsection{Intervals of Existence}
\label{ssec:intervalsOfExistence}

\begin{definition}{Existence and Uniqueness Theorem}
  Consider:
  \begin{equation*}
    y' + p(t)y = g(t) + C,\ y(t_0) = y_0
  \end{equation*}
  where the ODE is \textit{linear} and in its \textit{standard form}.
  Assuming that:
  \begin{enumerate}
    \itemsep-0.15em
    \item Both $p(t)$ and $g(t)$ are continuous over the open interval $(a,b)$
    \item The open interval $(a,b)$ contains $t_0$
  \end{enumerate}
  Then there exists a unique function $y = y(t)$ over the interval $(a,b)$ that solves the IVP.
\end{definition}

Based on this theorem, it can be asserted that there exists a unique solution over an interval $(a,b)$ provided that this interval contains $t_0$ from the IVP and both functions $p(t)$ and $g(t)$ are continuous over this interval.

\begin{example}
  \begin{equation*}
    ty' + (t-1)y = -e^{-t},\ y\left(\ln(2)\right) = \frac{1}{2}
  \end{equation*}
  To use the theorem, first the equation must be expressed in its standard form:
  \begin{equation*}
    y' + \frac{t-1}{t}\cdot y = -\frac{e^{-t}}{t}
  \end{equation*}
  Where:
  \begin{align*}
    p(t) &= \frac{t-1}{t} \\
    g(t) &= -\frac{e^{-t}}{t}
  \end{align*}
  Since, for both of these functions, the only point of discontinuity is $t=0$, then it is known that the only two possible intervals of existence are either $(-\infty,0)$ or $(0,\infty)$. \\

  Furthermore, since the initial condition is given as $y\left(\ln(2)\right) = \frac{1}{2}$, $t_0 = \ln(2)$. Thus, the interval of existence for the solution of this ODE would be the interval of $(0,\infty)$.
\end{example}

In some differential equations, a solution that is found is only valid over a certain interval. Consider (\ref{eq:205}):
\begin{equation}
  \frac{dy}{dx} = \frac{4}{(x-1)^{\frac{2}{3}}},\ y(0) = -10
  \label{eq:205}
\end{equation}
This differential equation can be solved pretty straightforwardly:
\begin{align*}
  y &= \int_{}^{} \frac{4}{(x-1)^{2/3}},\ dx \\
  y &= 12(x-1)^{\frac{1}{3}} + C
\end{align*}
Then using the initial condition to find the value of $C$:
\begin{align*}
  -10 &= 12(0-1)^{\frac{1}{3}} + C \\
  -10 &= 12(-1)^{\frac{1}{3}} + C\\
  -10 &= 12(-1) + C \\
  -10 &= -12(-1) + C \\
  2   &= C \\
\end{align*}
And finally the solution to (\ref{eq:205}):
\begin{equation*}
  y = 12(x-1)^{\frac{1}{3}} + 2
\end{equation*}
Now that the solution has been found, over what interval is this solution valid? This question is the crux of what the interval of existence means for a differential equation.

Looking at the solution: $y = 12(x-1)^{\frac{1}{3}} + 2$, it can be seen that the solution is infinitely continuous, or continuous over the interval of $(-\infty, \infty)$.

However, the original equation given: $\frac{dy}{dx} = \frac{4}{(x-1)^{\frac{2}{3}}}$ is differentiable over the interval $(-\infty,1)\cup(1,\infty)$.

Since there is then a discontinuity (caused by the derivative being non-differentiable at a point) at $1$, the interval of existence will only be the portion of continuity that contains the initial condition of $y(0)=-10$.

Thus, the interval of existence of (\ref{eq:205}) is $(-\infty, 1)$. Note that, with different initial conditions, the interval of existence could be different even with the same differential equation.

\subsection{Single Solutions}
\label{ssec:singleSolutions}

Consider
\begin{equation}
  y' = xy^3\left(1+x^2\right)^{-\frac{1}{2}}
  \label{eq:206}
\end{equation}
(\ref{eq:206}) is a separable ODE that can be reorganized as:
\begin{equation*}
  y' = y^3 \cdot \frac{x}{\sqrt{1+x^2}}
\end{equation*}
And subsequently separated into:
\begin{equation*}
  \frac{dy}{y^3} = \frac{x}{\sqrt{1+x^2}}\ dx
\end{equation*}
And solved:
\begin{align*}
  \frac{dy}{y^3} &= \frac{x}{\sqrt{1+x^2}}\ dx \\
  \int_{}^{} \frac{1}{y^3} \, dy &= \int_{}^{} \frac{x}{\sqrt{1+x^2}}\,dx \\
  -\frac{1}{2y^2} &= \sqrt{1+x^2} + C
\end{align*}
It would now be tempting to say that $-\frac{1}{2y^2} = \sqrt{1+x^2} + C$ is a general solution to (\ref{eq:206}). However, $y=0$ is also a solution to (\ref{eq:206})$\hdots$

The true answer then would be to say that $-\frac{1}{2y^2} = \sqrt{1+x^2} + C$ is a \textit{family} of solutions rather than the general solution and that $y=0$ is a singular solution.

\subsection{Autonomous ODE}
\label{ssec:autonomousODE}

\begin{equation*}
  y' = f(y)
\end{equation*}
Such that the right hand side (RHS) does not involve $t$. In other words, it is a function purely in terms of $y$.

The equilibrium solution:
\begin{equation*}
  y = y_0
\end{equation*}
Such that:
\begin{equation*}
  f(y_0) = 0
\end{equation*}
\begin{example}
  \begin{equation*}
    y' = y^3 - y
  \end{equation*}
  Clearly, the RHS is purely in terms of $y$, so this would be able to be solved as an autonomous ODE. So, taking the RHS and solving for $0$:
  \begin{align*}
    y^3 - y &= 0 \\
    y\left(y^2 - 1\right) &= 0 \\
    y\left(y - 1\right)\left(y + 1\right) &= 0 \\
    y = 0;\ y = 1;\ y &= -1
  \end{align*}
  Thus, there are three equilibrium solutions.
\end{example}

\subsubsection{Stability of an Equilibrium Solution}
\label{sssec:stabilityOfAnEqulibriumSolution}

\begin{definition}{Stability}
  Let $y=y_0$ be an equilibrium solution of $y' = f(y)$. $y=y_0$ is \textit{stable from above} if, for every $y>y_0$ near $y_0$, $f(y)<0$.
\end{definition}

\begin{figure}[H]
  \centering
  \begin{subfigure}[H]{0.45\textwidth}
    \centering
    \includestandalone{figures/fig_002}
    \caption{Stable From Above}
    \label{fig:002}
  \end{subfigure}
  \begin{subfigure}[H]{0.45\textwidth}
    \centering
    \includestandalone{figures/fig_003}
    \caption{Stable From Below}
    \label{fig:003}
  \end{subfigure}
  \caption{Stability}
  \label{fig:stability}
\end{figure}
Using Figure \ref{fig:002} to visualize this, the {\color{gr} solution} is stable from above because, if you were to deviate from the solution, the direction field above will guide you back down to the solution. Figure \ref{fig:003} is stable from below for the same reasons.

It follows then, that if the direction field around the solution points \textit{away} from the solution, then the solution is considered to be \textit{un}stable. Lastly, a solution is \textit{semi}stable if it is stable from one direction and unstable from the other.

\begin{example}
  \begin{equation*}
    \frac{dP}{dt} = R - \frac{P}{V}\cdot W;\ P(0) = 0
  \end{equation*}
  This is an autonomous ODE since the RHS \textit{does not} depend on $t$. Thus, setting the RHS $= 0$:
  \begin{gather*}
    R - \frac{P}{V}\cdot W=0 \\
    \vdots \\
    P= \frac{VR}{W}
  \end{gather*}
  Finding the stability of this solution: % TODO: SHOW STABILITY
\end{example}

\begin{example}
  Falling object from a great height, subject to acceleration due to gravity ($mg$) and air resistance ($kv^2$). Newton's second law:
  \begin{equation*}
    m \cdot \frac{dv}{dt} = mg - kv^2;\ v(0) = 0
  \end{equation*}
  This ODE is autonomous since the RHS has no $t$ involved, so the equilibrium solution will be obtained by setting the RHS $= 0$.
  \begin{gather*}
    mg-kv^2 = 0 \\
    \vdots \\
    v = \pm \sqrt{\frac{mg}{k}}
  \end{gather*}
  Finding the stability of this solution: % TODO: SHOW STABILITY


  Based on this model, the terminal velocity of the falling object will be:
  \begin{equation*}
    v = \sqrt{\frac{mg}{k}}
  \end{equation*}
  % \hrule
  % \vspace{12pt}
  % This ODE can also be solved by finding the explicit solution, first by rearranging the equation:
  % \begin{gather*}
  %   m \cdot \frac{dv}{dt} = mg - kv^2 \\
  %   \vdots \\
  %   \frac{m}{mg-kv^2}dv = dt
  % \end{gather*}
  % Then integrate both sides. Keep in mind that $g$, $k$, and $m$ are constants.
  % \begin{gather*}
  %   \int_{}^{} \frac{m}{mg-kv^2} \,dv = \int_{}^{}  \,dt \\
  %   \frac{m}{k}\int_{}^{} \frac{m}{\frac{mg}{k}-v^2} \,dv = \int_{}^{}  \,dt \\
  %   \frac{m}{k}\int_{}^{} \frac{m}{\left(\sqrt{\frac{mg}{k}}-v\right)\left(\sqrt{\frac{mg}{k}}+v\right)} \,dv = \int_{}^{}  \,dt \\
  % \end{gather*}
\end{example}
\begin{example}
  
\end{example}

\end{document}
