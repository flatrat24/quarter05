\documentclass[12pt]{article}

%%%% GRAPHICS %%%%
\usepackage{tikz}
\usepackage[siunitx, american, RPvoltages]{circuitikz}
\usetikzlibrary{arrows.meta}
\usepackage{tikz-3dplot}
\usepackage{graphicx}
\usepackage{pgfplots}
  \pgfplotsset{compat=1.18}
\usetikzlibrary{arrows}
\newcommand{\midarrow}{\tikz \draw[-triangle 90] (0,0) -- +(.1,0);}

%%%% FIGURES %%%%
\usepackage{subcaption}
\usepackage{wrapfig}
\usepackage{float}
\usepackage[skip=5pt, font=footnotesize]{caption}

%%%% FORMATTING %%%%
\usepackage{parskip}
\usepackage{tcolorbox}
\usepackage{ulem}

%%%% TABLE FORMATTING %%%%
\usepackage{tabularray}
\UseTblrLibrary{booktabs}

%%%% MATH AND LOGIC %%%%
\usepackage{xifthen}
\usepackage{amsmath}
\usepackage{amssymb}
\usepackage{amsfonts}

%%%% TEXT AND SYMBOLS %%%%
\usepackage[T1]{fontenc}
\usepackage{textcomp}
\usepackage{gensymb}

%%%% OTHER %%%%
\usepackage{standalone}

%%%% LOGIC SYMBOLS %%%%
\newcommand*\xor{\oplus}

%%%% STYLES %%%%

% Packages
\usepackage[paper=letterpaper,tmargin=45pt,bmargin=45pt,lmargin=45pt,rmargin=45pt]{geometry}
\usepackage{titlesec}
\usepackage[rgb]{xcolor}
\selectcolormodel{natural}
\usepackage{ninecolors}
\selectcolormodel{rgb}

% Colors
\definecolor{pg}{HTML}{24273A}
\definecolor{fg}{HTML}{FFFFFF}
\definecolor{bg}{HTML}{24273A}
\definecolor{re}{HTML}{F38BA8}
\definecolor{gr}{HTML}{A6E3A1}
\definecolor{ye}{HTML}{F9E2AF}
\definecolor{or}{HTML}{FAB387}
\definecolor{bl}{HTML}{89B4FA}
\definecolor{ma}{HTML}{CBA6F7}
\definecolor{cy}{HTML}{94E2D5}
\definecolor{pi}{HTML}{F2CDCD}

\definecolor{copper}{HTML}{B87333}

\usepackage{nameref}
\makeatletter
\newcommand*{\currentname}{\@currentlabelname}
\makeatother

\titleformat{\section}
  {\normalfont\scshape\Large\bfseries}
  {\thesection}
  {0.75em}
  {}

\titleformat{\subsection}
  {\normalfont\scshape\large\bfseries}
  {\thesubsection}
  {0.75em}
  {}

\titleformat{\subsubsection}
  {\normalfont\scshape\normalsize\bfseries}
  {\thesubsubsection}
  {0.75em}
  {}

% Formula
\newcounter{formula}[section]
\newenvironment{formula}[1]{
  \stepcounter{formula}
  \begin{tcolorbox}[
    standard jigsaw, % Allows opacity
    colframe={fg},
    boxrule=1px,
    colback=bg,
    opacityback=0,
    sharp corners,
    sidebyside,
    righthand width=18px,
    coltext={fg}
  ]
  \centering
  \textbf{\uline{#1}}
}{
  \tcblower
  \textbf{\thesection.\theformula}
  \end{tcolorbox}
}

% Definition
\newcounter{definition}[section]

\newenvironment{definition*}[1]{
  \begin{tcolorbox}[
    standard jigsaw, % Allows opacity
    colframe={fg},
    boxrule=1px,
    colback=bg,
    opacityback=0,
    sharp corners,
    coltext={fg}
  ]
  \textbf{#1 \hfill}
  \vspace{5px}
  \hrule
  \vspace{5px}
  \noindent
}{
  \end{tcolorbox}
}

\newenvironment{definition}[1]{
  \stepcounter{definition}
  \begin{tcolorbox}[
    standard jigsaw, % Allows opacity
    colframe={fg},
    boxrule=1px,
    colback=bg,
    opacityback=0,
    sharp corners,
    coltext={fg}
  ]
  \textbf{#1 \hfill \thesection.\thedefinition}
  \vspace{5px}
  \hrule
  \vspace{5px}
  \noindent
}{
  \end{tcolorbox}
}

% Example Problem
\newcounter{example}[section]
\newenvironment{example}{
  \stepcounter{example}
  \begin{tcolorbox}[
    standard jigsaw, % Allows opacity
    colframe={fg},
    boxrule=1px,
    colback=bg,
    opacityback=0,
    sharp corners,
    coltext={fg}
  ]
  \textbf{Example \hfill \thesection.\theexample}
  \vspace{5px}
  \hrule
  \vspace{5px}
  \noindent
}{
  \end{tcolorbox}
}

\tikzset{
  cubeBorder/.style=fg,
  cubeFilling/.style={fg!20!bg, opacity=0.25},
  gridLine/.style={very thin, gray},
  graphLine/.style={-latex, thick, fg},
}

\pgfplotsset{
  basicAxis/.style={
    grid,
    major grid style={line width=.2pt,draw=fg!50!bg},
    axis lines = box,
    axis line style = {line width = 1px},
  }
}

%%%% REFERENCES %%%%
\usepackage{hyperref}
\hypersetup{
  colorlinks  = true,
  linkcolor   = bl,
  anchorcolor = bl,
  citecolor   = bl,
  filecolor   = bl,
  menucolor   = bl,
  runcolor    = bl,
  urlcolor    = bl,
}

\author{Ethan Anthony}
\newcommand*{\equal}{=}


\title{Lecture 002}
\date{January 12, 2025}

\begin{document}
\setcounter{equation}{0}
\newpage

\section{First Order ODE}
\label{sec:firstOrderODE}

A first order ODE is an equation in the form of:
\begin{equation}
  \frac{dy}{dx} = f(x,y) \ \ \textup{or} \ \ y' = f(x,y)
  \label{eq:202}
  \vspace{-10pt}
\end{equation}

\begin{definition}{First Order ODE}
  A differential equation whose highest order derivative is a first order derivative, with all derivatives being with respect to a single variable (usually $x$ or $t$).
\end{definition}

There is no strict process to find a solution to these equations. Thus, a lot of this class is spent on the various different ways to find solutions. To see one of the simpler ways, consider an equation where $f$ is a function of only $x$:
\vspace{-10pt}
\begin{equation*}
  y' = f(x)
\end{equation*}
Integrating both sides with respect to $x$, the equation becomes:
\begin{equation*}
  \int_{}^{} y' \, dx = \int_{}^{} f(x) \, dx + C \rightarrow \left[y(x) = \int_{}^{} f(x) \, dx + C\right]
\end{equation*}
$y(x) = \int_{}^{} f(x) \, dx + C$ is the general solution to (\ref{eq:202}). Thus, to solve a differential equation with just a single dependent variable $x$, finding the antiderivative of $f(x)$ is sufficient to find the general solution.
\begin{example}{First Order ODE by Direct Integration}
  \begin{equation*}
    y' = 3x^2
  \end{equation*}
  Integrating each side gives:
  \begin{equation*}
    \int_{}^{} y' \, dx = \int_{}^{} 3x^2 \, dx + C \rightarrow \left[y(x) = x^3 + C\right]
  \end{equation*}
\end{example}

\begin{example}{First Order Linear ODE by Standard Form}
  \begin{equation*}
    y' + y = \cos(2t)
  \end{equation*}
  This is already in standard form: $p(t) = {\color{ye} 1}$, $g(t) = {\color{ma} cos(2t)}$. Calculating $\mu(t)$:
  \begin{equation*}
    \mu(t) = e^{\int_{}^{} {\color{ye} 1} \,dt} = {\color{gr} e^t}
  \end{equation*}
  Thus, the general solution will be:
  \begin{equation*}
    y = \frac{\int_{}^{} {\color{gr} e^t} {\color{ma} \cos(2t)} \,dt}{{\color{gr} e^t}}
  \end{equation*}
  Integrating the numerator:
  \begin{gather*}
    \int_{}^{} e^t \cdot \cos(2t) \,dt = \hdots = \frac{1}{5}\left(e^t \cos(2t) + 2e^t \sin(2t)\right) + C
  \end{gather*}
  Thus:
  \begin{equation*}
    y = \frac{\frac{1}{5}\left(e^t \cos(2t) + 2e^t \sin(2t)\right) + C}{e^t}\ \ \ \textup{or}\ \ \ y = \frac{1}{5}\left(\cos(2t) + 2\sin(2t)\right) + Ce^{-t}
  \end{equation*}
\end{example}

\subsection{First Order Linear ODE}
\label{ssec:firstOrderLinearODE}

\begin{definition}{First Order Linear ODE}
  A differential equation that 1) doesn't contain a derivative beyond the first order, 2) contains only derivatives with respect to a single variable, and 3) has its dependent variable appear linearly (not part of a sin, cos, square, etc.).
\end{definition}

Solving a first order linear ODE starts by putting the ODE into \textbf{standard form}. Though there are different "standard forms" for different kinds of ODEs, "standard form" refers to the generic (and arbitrary) way of writing an ODE.

\begin{formula}{Standard Form of a First Order Linear ODE}
  \begin{equation*}
    y' + {\color{gr} p(t)}y = {\color{ma} g(t)}
  \end{equation*}
\end{formula}

If a differential equation is given in \textbf{standard form}, where {\color{gr} $p(t)$} and {\color{ma} $g(t)$} are arbitrary functions of $t$, then the general solution can be expressed as:
\begin{equation*}
  y(t) = \frac{\int_{}^{} {\color{or} \mu(t)}{\color{ma} g(t)} \,dt + C}{{\color{or} \mu(t)}},\ \textup{where}\ {\color{or} \mu(t)} = e^{\int_{}^{} {\color{gr} p(t)} \,dt}
\end{equation*}
% Where the \textbf{integrating factor} $\big[{\color{or} \mu(t)}\big]$ is:
% \begin{equation*}
%   {\color{or} \mu(t)} = e^{\int_{}^{} {\color{gr} p(t)} \,dt}
% \end{equation*}

\begin{example}{First Order Linear IVP in Standard Form}
  \begin{equation*}
    y' + 2y = {\color{ma} e^{3t}},\ y(0) = {\color{ye} 3}
  \end{equation*}
  This ODE is already in standard form, so the integrating factor can be written as:
  \begin{equation*}
    \mu(t) = e^{\int_{}^{} 2 \,dt} = {\color{or} e^{2t}}
  \end{equation*}
  And thus the general solutions is:
  \begin{align*}
    y(t) &= \frac{\int_{}^{} {\color{or} e^{2t}} {\color{ma} e^{3t}} \,dt + C}{{\color{or} e^{2t}}} = \frac{\int_{}^{} e^{5t} \,dt + C}{e^{2t}} = \frac{\frac{1}{5} e^{5t} + C}{e^{2t}} = \frac{1}{5} \frac{e^{5t}}{e^{2t}} + \frac{C}{e^{2t}} = {\color{bl} \frac{1}{5} e^{3t} + \frac{C}{e^{2t}}}
  \end{align*}
  Using the initial condition to find the specific solution:
  \begin{equation*}
    y(t) = {\color{bl} \frac{1}{5} e^{3t} + \frac{C}{e^{2t}}} \rightarrow \left[y(0) = \frac{1}{5} e^{3 \cdot 0} + \frac{C}{e^{2 \cdot 0}}\right] \rightarrow \left[{\color{ye} 3} = \frac{1}{5} + C\right] \rightarrow \left[{\color{re} \frac{14}{5}} = C\right]
  \end{equation*}
  Thus, the solution of the IVP is:
  \begin{equation*}
    y(t) = \frac{1}{5} e^{3t} + {\color{re} \frac{14}{5}} e^{-2t} \\
  \end{equation*}
\end{example}

\subsection{First Order Non-Linear ODE}
\label{ssec:firstOrderNonLinearODE}

Generally, non-linear ODEs are quite difficult to solve. The scope of this course doesn't cover non-linear ODEs in their entirety, but rather a select subset.

\subsubsection{Separable ODE}
\label{sssec:separableODE}

The basic form of a separable ODE is:
\begin{equation*}
  \frac{{\color{re} dy}}{{\color{gr} dx}} = {\color{bl} f(y)} \cdot {\color{ma} g(x)}
\end{equation*}
i.e., the derivative is a product of two functions, one depending on $x$ and the other on $y$.

This type of ODE is solved first by separating the variables. The LHS is reserved for $y$ and the RHS for $x$:
\begin{equation*}
  \frac{{\color{re} dy}}{{\color{bl} f(y)}} = {\color{ma} g(x)} {\color{gr} dx}
\end{equation*}
Now that there is a single type of variable on each side (only $x$ or only $y$), both sides can be integrated. This will yield a one-parameter family of \textit{implicit solutions} (see Subsection \ref{ssec:implicitAndExplicitSolutions}).
\begin{example}{First Order Non-Linear ODE by Separation of Variables}
  \begin{equation*}
    y' = xy
  \end{equation*}
  This equation can be rewritten as:
  \begin{equation*}
    \frac{{\color{re} dy}}{{\color{gr} dx}} = {\color{bl} y} \cdot {\color{ma} x}
  \end{equation*}
  Here, it can be easily seen that this equation is separable:
  \begin{equation*}
    \frac{{\color{re} dy}}{{\color{bl} y}} = {\color{ma} x}\ {\color{gr} dx} \rightarrow \left[\int_{}^{} \frac{1}{y} \, dy = \int_{}^{} x \, dx + C\right] \rightarrow \left[{\color{or} \ln|y| = \frac{x^2}{2} + C}\right]
  \end{equation*}
  Having found the \textit{implicit solution}, it can be seen that the \textit{explicit solution} can also be found through algebra:
  \begin{equation*}
    {\color{or} \ln|y| = \frac{x^2}{2} + C} \rightarrow \left[e^{\ln|y|} = e^{\frac{x^2}{2} + C}\right] \rightarrow \left[y = e^{\frac{x^2}{2}} \cdot {\color{pi} e^{C}}\right] \rightarrow \left[y = {\color{pi} D}e^{\frac{x^2}{2}}\right]
  \end{equation*}
  Where $D>0$ (since $e^C>0$). Since $y=0$ is a solution as well, this can be simplified to:
  \begin{equation*}
    \forall D \geq 0,\ y = De^{\frac{x^2}{2}}
  \end{equation*}
\end{example}

\begin{example}{First Order Non-Linear ODE by Separation of Variables}
  \begin{equation*}
    \frac{dy}{dx} = \frac{e^x - x}{e^{-y} + y}
  \end{equation*}
  First, separating the variables to each side:
  \begin{align*}
    \frac{dy}{dx} &= \left(e^x - x\right) \cdot \frac{1}{e^{-y} + y} \\
    \left(e^{-y} + y\right) dy &= \left(e^x - x\right) dx
  \end{align*}
  Then integrating each side appropriately:
  \begin{align*}
    \int e^{-y} + y \, dy &= \int e^x - x \, dx \\
    -e^{-y} + \frac{1}{2}y^2 &= e^x - \frac{1}{2}x^2 + C
  \end{align*}
  And thus, a one-parameter family of implicit solutions has been found.
\end{example}

\subsubsection{Singular Solutions for Separable ODE}
\label{sssec:singularSolutionsForSeparableODE}

For a separable ODE:
\begin{equation*}
  {\color{ma} \frac{dy}{dx}} = {\color{gr} f(y) \cdot g(x)}
\end{equation*}
the constant function $y={\color{re} a}$ is a singular solution for all numbers ${\color{re} a}$ satisfying $f({\color{re} a})=0$. This can be seen in two parts. First, with $f({\color{re} a})=0$, the {\color{gr} RHS} of the equation will always be zero:
\begin{equation*}
  \textup{RHS} = f({\color{re} a}) \cdot g(x) = 0 \cdot g(x) = 0
\end{equation*}
Second, the {\color{ma} LHS} will always be zero since $y={\color{re} a}$ is a constant function.
\begin{equation*}
  \textup{LHS} = y' = \frac{dy}{dx} = \frac{d}{dx}\left({\color{re} a}\right) = 0
\end{equation*}

\end{document}
