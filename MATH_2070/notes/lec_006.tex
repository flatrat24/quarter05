\documentclass[12pt]{article}

%%%% GRAPHICS %%%%
\usepackage{tikz}
\usepackage[siunitx, american, RPvoltages]{circuitikz}
\usetikzlibrary{arrows.meta}
\usepackage{tikz-3dplot}
\usepackage{graphicx}
\usepackage{pgfplots}
  \pgfplotsset{compat=1.18}
\usetikzlibrary{arrows}
\newcommand{\midarrow}{\tikz \draw[-triangle 90] (0,0) -- +(.1,0);}

%%%% FIGURES %%%%
\usepackage{subcaption}
\usepackage{wrapfig}
\usepackage{float}
\usepackage[skip=5pt, font=footnotesize]{caption}

%%%% FORMATTING %%%%
\usepackage{parskip}
\usepackage{tcolorbox}
\usepackage{ulem}

%%%% TABLE FORMATTING %%%%
\usepackage{tabularray}
\UseTblrLibrary{booktabs}

%%%% MATH AND LOGIC %%%%
\usepackage{xifthen}
\usepackage{amsmath}
\usepackage{amssymb}
\usepackage{amsfonts}

%%%% TEXT AND SYMBOLS %%%%
\usepackage[T1]{fontenc}
\usepackage{textcomp}
\usepackage{gensymb}

%%%% OTHER %%%%
\usepackage{standalone}

%%%% LOGIC SYMBOLS %%%%
\newcommand*\xor{\oplus}

%%%% STYLES %%%%

% Packages
\usepackage[paper=letterpaper,tmargin=45pt,bmargin=45pt,lmargin=45pt,rmargin=45pt]{geometry}
\usepackage{titlesec}
\usepackage[rgb]{xcolor}
\selectcolormodel{natural}
\usepackage{ninecolors}
\selectcolormodel{rgb}

% Colors
\definecolor{pg}{HTML}{24273A}
\definecolor{fg}{HTML}{FFFFFF}
\definecolor{bg}{HTML}{24273A}
\definecolor{re}{HTML}{F38BA8}
\definecolor{gr}{HTML}{A6E3A1}
\definecolor{ye}{HTML}{F9E2AF}
\definecolor{or}{HTML}{FAB387}
\definecolor{bl}{HTML}{89B4FA}
\definecolor{ma}{HTML}{CBA6F7}
\definecolor{cy}{HTML}{94E2D5}
\definecolor{pi}{HTML}{F2CDCD}

\definecolor{copper}{HTML}{B87333}

\usepackage{nameref}
\makeatletter
\newcommand*{\currentname}{\@currentlabelname}
\makeatother

\titleformat{\section}
  {\normalfont\scshape\Large\bfseries}
  {\thesection}
  {0.75em}
  {}

\titleformat{\subsection}
  {\normalfont\scshape\large\bfseries}
  {\thesubsection}
  {0.75em}
  {}

\titleformat{\subsubsection}
  {\normalfont\scshape\normalsize\bfseries}
  {\thesubsubsection}
  {0.75em}
  {}

% Formula
\newcounter{formula}[section]
\newenvironment{formula}[1]{
  \stepcounter{formula}
  \begin{tcolorbox}[
    standard jigsaw, % Allows opacity
    colframe={fg},
    boxrule=1px,
    colback=bg,
    opacityback=0,
    sharp corners,
    sidebyside,
    righthand width=18px,
    coltext={fg}
  ]
  \centering
  \textbf{\uline{#1}}
}{
  \tcblower
  \textbf{\thesection.\theformula}
  \end{tcolorbox}
}

% Definition
\newcounter{definition}[section]

\newenvironment{definition*}[1]{
  \begin{tcolorbox}[
    standard jigsaw, % Allows opacity
    colframe={fg},
    boxrule=1px,
    colback=bg,
    opacityback=0,
    sharp corners,
    coltext={fg}
  ]
  \textbf{#1 \hfill}
  \vspace{5px}
  \hrule
  \vspace{5px}
  \noindent
}{
  \end{tcolorbox}
}

\newenvironment{definition}[1]{
  \stepcounter{definition}
  \begin{tcolorbox}[
    standard jigsaw, % Allows opacity
    colframe={fg},
    boxrule=1px,
    colback=bg,
    opacityback=0,
    sharp corners,
    coltext={fg}
  ]
  \textbf{#1 \hfill \thesection.\thedefinition}
  \vspace{5px}
  \hrule
  \vspace{5px}
  \noindent
}{
  \end{tcolorbox}
}

% Example Problem
\newcounter{example}[section]
\newenvironment{example}{
  \stepcounter{example}
  \begin{tcolorbox}[
    standard jigsaw, % Allows opacity
    colframe={fg},
    boxrule=1px,
    colback=bg,
    opacityback=0,
    sharp corners,
    coltext={fg}
  ]
  \textbf{Example \hfill \thesection.\theexample}
  \vspace{5px}
  \hrule
  \vspace{5px}
  \noindent
}{
  \end{tcolorbox}
}

\tikzset{
  cubeBorder/.style=fg,
  cubeFilling/.style={fg!20!bg, opacity=0.25},
  gridLine/.style={very thin, gray},
  graphLine/.style={-latex, thick, fg},
}

\pgfplotsset{
  basicAxis/.style={
    grid,
    major grid style={line width=.2pt,draw=fg!50!bg},
    axis lines = box,
    axis line style = {line width = 1px},
  }
}

%%%% REFERENCES %%%%
\usepackage{hyperref}
\hypersetup{
  colorlinks  = true,
  linkcolor   = bl,
  anchorcolor = bl,
  citecolor   = bl,
  filecolor   = bl,
  menucolor   = bl,
  runcolor    = bl,
  urlcolor    = bl,
}

\author{Ethan Anthony}
\newcommand*{\equal}{=}


\title{Lecture 006}
\date{February 10, 2025}

\begin{document}
\newpage
\section{Non-Homogeneous ODE}
\label{sec:nonHomogeneousODE}

Non-Homogeneous ODE's are of the form:
\begin{equation*}
  y'' + p(t)y' + q(t)y = g(t)
\end{equation*}
These ODEs behave differently to previous ones. As such, the principle of superposition would now say:

\begin{definition}{Principle of Superposition (Non-Homogeneous)}
  If $Y_1$, $Y_2$ are two solutions of
  \begin{equation*}
    y'' + p(t)y' + q(t)y = g(t)
  \end{equation*}
  then $Y_1-Y_2$ is a solution of
  \begin{equation*}
    y'' + p(t)y' + q(t)y = 0
  \end{equation*}
\end{definition}

The proof of this is as follows:
\begin{align*}
  y_1'' + p(t)y_1' + q(t)y_1 &= g(t) \\
  y_2'' + p(t)y_2' + q(t)y_2 &= g(t)
\end{align*}
Then:
\begin{equation*}
  Y_1 - Y_2 = y_1''-y_2'' + p(t)\left(y_1'-y_2'\right) + q(t)\left(y_1-y_2\right) = g(t)-g(t) = 0
\end{equation*}
\vspace{12pt}
\hrule
\vspace{12pt}
The general solutions of $y'' + p(t)y' + q(t)y = g(t)$ is of the form:
\begin{align*}
  y &= y_c + Y \\
    &= C_1y_1 + C_2y_2 + Y
\end{align*}
where $y_c = C_1y_1 + C_2y_2$ is the general solution of the \textit{homogeneous} ODE $y'' + p(t)y' + q(t)y = 0$, and $Y$ is a particular solution of the \textit{non-homogeneous} $y'' + p(t)y' + q(t)y = g(t)$.

The terminology to describe the parts of this general solutions is:
\begin{itemize}
  \itemsep0em
  \item $y_c = C_1y_1 + C_2y_2$ is the {\color{re} \textbf{complementary solution}}
  \item $Y$ is the {\color{re} \textbf{particular solution}}
\end{itemize}

% The proof of this starts with verifying that $y_c + Y$ satisfies the ODE:
% \begin{align*}
%   \textup{LHS} &= y_c'' + Y'' + p(t)\left(y_c' + Y'\right) + g(t)\left(y_c + Y\right) \\
%                &= y_c'' + p(t)y_c' + g(t)y_c + Y'' + p(t)Y' + g(t)Y \\
%                &= g(t) = \textup{RHS}
% \end{align*}
% Since:
% \begin{gather*}
%   y_c'' + p(t)y_c' + g(t)y_c = 0 \\
%   Y'' + p(t)Y' + g(t)Y = g(t)
% \end{gather*}
% Any solution of the non-homogeneous ODE is of the form $y_c+Y$ for some $C_1$, $C_2$.
%
% Set $y$ to be any solution of $y'' + p(t)y' + g(t)y = g(t)$.
%
% Principle $\Rightarrow$ $y-Y$ is a solutions of $y'' + p(t)y' + g(t)y = 0$ $\Rightarrow$ $y-Y = C_1y_1 + C_2y_2$ for some $C_1$, $C_2$.
%
% \vspace{12pt}
% \hrule
% \vspace{12pt}

Finding the complementary solution is simple. All that needs to be done is described in Subsection \ref{ssec:linearHomogeneousODEWithConstantCoefficients}. To find the particular solution, two methods can be used:
\begin{enumerate}
  \itemsep0em
  \item Method of Undetermined Coefficients
  \item Variation of Parameter
\end{enumerate}

\newpage
\subsection{Method of Undetermined Coefficients}
\label{ssec:methodOfUndeterminedCoefficients}
The method makes it easy to generalize to higher order ODEs, however it only works for:
\begin{equation*}
  ay'' + by' + c = g(t)
\end{equation*}
Where $g(t)$ is a product of exponential polynomials, $\sin$, and $\cos$ functions. However, these limitations aren't very relevant for engineering applications.

\subsubsection{Constant Functions}
\label{sssec:constantFunctions}
Consider the ODE:
\begin{equation*}
  y'' - 2y' - 3y = 3
\end{equation*}
% First, using the complementary equation, namely:
% \begin{equation*}
%   y'' - 2y' - 3y = 0
% \end{equation*}
% The auxiliary equation can be formulated as:
% \begin{equation*}
%   r^2 - 2r - 3
% \end{equation*}
% Giving roots of $r_1=3$, $r_2=-1$, thus giving the general solution as:
% \begin{equation*}
%   y_c = C_1e^{3t} + C_2e^{-t}
% \end{equation*}

Since the RHS is a constant function, set $Y$ equal to some constant function $A$:
\begin{align*}
  Y = A   \\
  Y' = 0  \\
  Y'' = 0
\end{align*}
and plug in the ODE:
\begin{align*}
  Y'' - 2Y' - 3Y  &= 0 - 0 - 3A = 3 \\
  \Rightarrow -3A &= 3 \\
  \Rightarrow A   &= -1
\end{align*}
Giving $Y=-1$ as a particular solution, so the particular solution is:
\begin{equation*}
  Y = -1
\end{equation*}
% And the general solutions is:
% \begin{equation*}
%   y = c_1e^{3t} + C_2e^{-t} - 1
% \end{equation*}
\subsubsection{Polynomial Functions}
\label{sssec:polynomialFunctions}
Consider the ODE:
\begin{equation*}
  y'' - y' - 2y = t^2 + 1
\end{equation*}

% By inspection, the complimentary solution of $y'' - y' - 2y = 0$ is:
% \begin{equation*}
%   y_c = C_1e^{-t}+C_2e^{2t}
% \end{equation*}

Since the RHS is a polynomial function of degree $2$, $Y$ can be set equal to a generic polynomial function of degree $2$. Set:
\begin{align*}
  Y &= At^2 + Bt + C \\
  Y' &= 2At + B \\
  Y'' &= 2A
\end{align*}
Thus:
\begin{align*}
  Y'' - Y' - 2Y &= 2A - (2At+B) - 2\left(At^2+Bt+C\right) \\
                &= (-2A)t^2 + (-2A-2B)t + (2A - B - 2C) \\
                &= t^2 + 1
\end{align*}
Setting $t^2 + 1$ and $(2A)t^2 + (-2A-2B)t^1 + (2A - B - 2C)t^0$ equal to each other, it can be found that:
\begin{align*}
  -2At^2 = 1t^2         &\Rightarrow -2A = 1 \\
  (-2A-2B)t^1 = (0)t^1  &\Rightarrow -2A-2B = 0 \\
  (2A-B-2C)t^0 = (1)t^0 &\Rightarrow 2A-B-2C = 1
\end{align*}
Then solving for each value of $A$, $B$, and $C$:
\begin{gather*}
  A = -\frac{1}{2} \\
  B = -A = \frac{1}{2} \\
  C = \frac{2A-B-1}{2} = -\frac{5}{4}
\end{gather*}
Giving the particular solution $Y$ as:
\begin{equation*}
  Y = -\frac{1}{2}t^2 + \frac{1}{2}t - \frac{5}{4}
\end{equation*}
% and the general solution as:
% \begin{equation*}
%   y = C_1e^{-t} + C_2e^{2t} - \frac{1}{2}t^2 + \frac{1}{2}t - \frac{5}{4}
% \end{equation*}

It should be noted that if the RHS is a polynomial of degree $n$, $Y$ should be set as a generic polynomial of degree $n$. Even when $g(t) = t$, $Y$ should still start from degree $n$, and work all the way down to the constant. For example, if:
\begin{align*}
  g(t) = t^2 + t + 1  &\Rightarrow Y = At^2 + Bt + C \\
  g(t) = 6t^4         &\Rightarrow Y = At^4 + Bt^3 + Ct^2 + Dt + E \\
  g(t) = t            &\Rightarrow Y = At \\
  g(t) = -t^3 + t - 8 &\Rightarrow Y = At^3 + Bt^2 + Ct^1 + D
\end{align*}

As a full example, consider the ODE:
\begin{equation*}
  y'' - y' - 2y = t^3
\end{equation*}
First, setting the particular solution $Y$ as some generic polynomial of the $n^{th}$ degree:
\begin{align*}
  Y &= At^3 - Bt^2 + Ct + D \\
  Y' &= 3At^2 - 2Bt + C \\
  Y'' &= 6At^1 - 2B
\end{align*}
and plugging it in:
\begin{equation*}
  Y'' - Y' - 2Y = -2A^3 + \left(-3A-2B\right)t^2 + (6A-2B-2C)t + (2B-C-2D)
\end{equation*}
Comparing $Y = At^3 - Bt^2 + Ct + D$ with $-2A^3 + \left(-3A-2B\right)t^2 + (6A-2B-2C)t + (2B-C-2D)$, a system of equations can be found as:
\begin{gather*}
  -2A = 1 \\
  -3A-2B = 0 \\
  6A - 2B - 2C = 0 \\
  2B - C - 2D = 0
\end{gather*}
Solving the system gives:
\begin{align*}
  A &= - \frac{1}{2} \\
  B &= - \frac{3}{2}A = \frac{3}{4} \\
  C &= \frac{1}{2}(6A-2B) = -\frac{9}{4} \\
  D &= \frac{1}{2}(2B-C) = \frac{15}{8}
\end{align*}
Thus finding the particular solution as:
\begin{equation*}
  Y = - \frac{1}{2}t^3 + \frac{3}{4}t^2 + -\frac{9}{4}t + \frac{15}{8}
\end{equation*}
and subsequently, the general solution as:
\begin{equation*}
  y = C_1e^{2t} + C_2e^{-t} - \frac{1}{2}t^3 + \frac{3}{4}t^2 + -\frac{9}{4}t + \frac{15}{8}
\end{equation*}

\subsubsection{Exponential Functions}
\label{sssec:exponentialFunctions}

Consider the ODE:
\begin{equation*}
  y'' - y' - 2y = e^{3t}
\end{equation*}
Since the RHS is an exponential function, formulate $Y$ as some generic exponential function with \textbf{the same} exponential coefficient.
\begin{align*}
  Y   &= Ae^{3t} \\
  Y'  &= 3Ae^{3t} \\
  Y'' &= 9Ae^{3t}
\end{align*}
then plug into the ODE:
\begin{align*}
  Y'' - Y' - 2Y &= 9Ae^{3t} - 3Ae^{3t} - 2\left(Ae^{3t}\right) \\
                &= 9Ae^{3t} - 3Ae^{3t} - 2\left(Ae^{3t}\right) \\
                &= 4Ae^{3t} = e^{3t}
\end{align*}
Thus showing that:
\begin{equation*}
  A = \frac{1}{4} \rightarrow Y = Ae^{3t} = \frac{1}{4}e^{3t}
\end{equation*}

\subsubsection{Sine and Cosine Functions}
\label{sssec:sinAndCosFunctions}

Consider the ODE:
\begin{equation*}
  y'' - 2y' + y = 3\sin(3t)
\end{equation*}
Since the RHS is some linear combination of $\sin$ and $\cos$ functions, formulate $Y$ as a generic summation of $\sin$ and $\cos$ with \textbf{the same} trigonometric coefficient.
\begin{align*}
  Y   &= A\sin(3t) + B\cos(3t) \\
  Y'  &= 3A\cos(3t) - 3B\sin(3t) \\
  Y'' &= -9A\sin(3t) - 9B\cos(3t)
\end{align*}
then plug into the ODE:
\begin{align*}
  Y'' - 2Y' + Y &= \left[-9A\sin(3t) - 9B\cos(3t)\right] - 2\left[3A\cos(3t) - 3B\sin(3t)\right] + \left[A\sin(3t) + B\cos(3t)\right] \\
                &= (-8A+6B)\sin(3t) + (-8B-6A)\cos(3t) = 3\sin(3t)
\end{align*}
Thus showing that:
\begin{align*}
  -8A + 6B &= 3 \\
  -8B - 6A &= 0
\end{align*}
and, in turn, that:
\begin{equation*}
  A = -\frac{6}{25}\ \ \\textup{,}\ \ \ B =  \frac{9}{50} \rightarrow Y = -\frac{6}{25}\sin(3t)+\frac{9}{50}\cos(3t)
\end{equation*}
Note that, for any linear combination of $\sin$ and $\cos$, this process works. It doesn't matter that the example case of $g(t)$ was without a $\cos$ term. For example:
\begin{align*}
  g(t) = 3\cos(2t)              &\Rightarrow Y = A\sin(2t) + B\cos(2t) \\
  g(t) = \frac{1}{2}\sin(t)     &\Rightarrow Y = A\sin(t) + B\cos(t) \\
  g(t) = \sin(4t) + \cos(4t)    &\Rightarrow Y = A\sin(4t) + B\cos(4t)
\end{align*}

\subsubsection{First Ansatz}
\label{sssec:firstAnsatz}

These methods described in Subsubsections \ref{sssec:constantFunctions}, \ref{sssec:polynomialFunctions}, \ref{sssec:exponentialFunctions}, and \ref{sssec:sinAndCosFunctions} all work because:
\begin{itemize}
  \itemsep0em
  \item Derivatives of constant functions are constant functions ($0$)
  \item Derivatives of polynomial functions are polynomials (up the same degree)
  \item Derivatives of exponential functions are exponential functions (with the same exponential coefficient)
  \item Derivatives of sine and cosine functions are sine and cosine functions (with the same trigonometric coefficient)
\end{itemize}
And so, the general approach for an ODE in the form:
\begin{equation*}
  ay'' + by' + cy = g(t)
\end{equation*}
If $g(t)$ is:
\begin{itemize}
  \itemsep0em
  \item A constant function, eg. $2$, $16$, $-12$:
    \begin{equation*}
      Y = A
    \end{equation*}
  \item A polynomial function of degree {\color{gr} $n$}, eg. $t^{{\color{gr} 3}} + 2t - 1$, $t - 12 + t^{{\color{gr} 100}}$, $t^{\color{gr} 1}$:
    \begin{equation*}
      Y = At^n + Bt^{n-1} + Ct^{n-2} + \hdots + Dt + E = \sum_{i=0}^{n} C_it^{n-i}
    \end{equation*}
  \item An exponential function with an exponential coefficient of {\color{re} $\alpha$}, eg. $12e^{{\color{re}3}t}$, $e^{{\color{re}1}t}$, $32e^{{\color{re}-5}t}$:
    \begin{equation*}
      Y = Ae^{{\color{re}\alpha}t}
    \end{equation*}
  \item A linear combination of $\sin$ and $\cos$ with a trig coefficient of {\color{ma} $\beta$}, eg. $\sin({\color{ma}3}t)$, $\sin({\color{ma}2}t) - 12\cos({\color{ma}2}t)$:
    \begin{equation*}
      Y = A\sin({\color{ma}\beta}t) + B\cos({\color{ma}\beta}t) \\
    \end{equation*}
\end{itemize}
This list of templates for solving for the particular solution of a non-homogeneous ODE continues with combinations of the basic types in Subsubsections \ref{sssec:constantFunctions}, \ref{sssec:polynomialFunctions}, \ref{sssec:exponentialFunctions}, and \ref{sssec:sinAndCosFunctions}.

If $g(t)$ is:
\begin{itemize}
  \itemsep0em
  \item A product of a {\color{gr} polynomial} and {\color{re} exponential} function, eg. $4t^3e^{3t}$, $\left(t^3 + 2t\right)e^{-t}$
    \begin{equation*}
      Y = \left({\color{gr} \sum_{i=0}^{n} C_it^{n-i}}\right){\color{re} e^{\alpha t}}
    \end{equation*}
  \item A product of a {\color{gr} polynomial} and {\color{ma} trigonometric} function, eg. $4t^2\sin(2t)$, $t^4\sin(3t) + 2t^2\cos(3t)$
    \begin{equation*}
      Y = \left({\color{gr} \sum_{i=0}^{n} C_it^{n-i}}\right){\color{ma} \sin(\beta t)} + \left({\color{gr} \sum_{i=n}^{2n} C_it^{2n-i}}\right){\color{ma} \cos(\beta t)}
    \end{equation*}
    It should be noted that both $\sin$ and $\cos$ are paired with a generic polynomial of the \textbf{same highest degree} from $g(t)$. Additionally, the coefficients of each term in each polynomial are entirely unique. For example, for:
    \begin{equation*}
      g(t) = 2t\cos(3t) + \left(t^2 + 1\right)\sin(3t)
    \end{equation*}
    The highest degree polynomial term is $n=2$, and thus:
    \begin{equation*}
      Y = \left(At^2+Bt+C\right)\sin(3t) + \left(Dt^2+Et+F\right)\cos(3t)
    \end{equation*}
  \item A product of a {\color{re} exponential} and {\color{ma} trigonometric} function, eg. $3e^{3t}\cos(4t)$, $2e^{3t}\cos(4t)-3e^{3t}\sin(4t)$
    \begin{equation*}
      Y = {\color{re} Ae^{\alpha t}}{\color{ma} \sin(\beta t)}+{\color{re} Be^{\alpha t}}{\color{ma} \cos(\beta t)}
    \end{equation*}
  \item A product of a {\color{gr} polynomial}, {\color{re} exponential}, and {\color{ma} trigonometric} function, eg. $18(t^2 - t)e^{3t}\cos(4t) - 7e^{3t}\sin(4t)$
    \begin{equation*}
      Y = \left({\color{gr} \sum_{i=0}^{n} C_it^{n-i}}\right){\color{re} Ae^{\alpha t}}{\color{ma} \sin(\beta t)}+\left({\color{gr} \sum_{i=n}^{2n} C_it^{2n-i}}\right){\color{re} Be^{\alpha t}}{\color{ma} \cos(\beta t)}
    \end{equation*}
\end{itemize}

\subsubsection{Second Ansatz}
\label{sssec:secondAnsatz}

Consider the ODE:

\begin{equation*}
  y''-y'-2y = e^{2t}
\end{equation*}

Based on Subsubsection \ref{sssec:combinationsOfTHeThreeBasicFunctions}, the particular solution $Y$ should be formulated as:
\begin{align*}
  Y   &= Ae^{2t} \\
  Y'  &= 2Ae^{2t} \\
  Y'' &= 4Ae^{2t}
\end{align*}
such that when it's plugged into the ODE:
\begin{align*}
  Y'' - Y' - 2Y &= 4Ae^{2t} - 2Ae^{2t} - 2Ae^{2t} \\
                &= 4Ae^{2t} - 4Ae^{2t} \\
                &= 0 = g(t) = e^{2t}
\end{align*}
There is no such solution to the equation:
\begin{equation*}
  0 = e^{2t}
\end{equation*}
and thus the first ansatz fails. In such a situation, the second ansatz should be used. Namely, by applying the variation of parameters, $Y$ can be formulated as the first ansatz multiplied by a $t$ factor:
\begin{align*}
  Y   = Ae^{2t}    &\rightarrow Y   = Ate^{2t} \\
  Y'  = 2Ae^{2t}  &\rightarrow  Y'  = Ae^{2t} + 2Ate^{2t} \\
  Y'' = 4Ae^{2t} &\rightarrow   Y'' = 2Ae^{2t} + 2Ae^{2t} + 4Ate^{2t}
\end{align*}
and thus, when plugged in:
\begin{align*}
  Y'' - Y' - 2Y &= 2Ae^{2t} + 2Ae^{2t} + 4Ate^{2t} - Ae^{2t} - 2Ate^{2t} - 2Ate^{2t} \\
                &= 3Ae^{2t} = g(t) = e^{2t}
\end{align*}
then, to solve the equation, the result is:
\begin{equation*}
  A = \frac{1}{3} \rightarrow Y = \frac{1}{3}te^{2t}
\end{equation*}

\subsubsection{Third Ansatz}
\label{sssec:thirdAnsatz}

Consider the ODE:
\begin{equation*}
  y'' - 2y' + y = 2te^{t}
\end{equation*}

\subsubsection{Finding the True Ansatz}
\label{sssec:findingTheTrueAnsatz}

Trying the first, then second, then third, etc., ansatz consecutively isn't the \textit{wrong} way to approach finding the particular solution, but it's cumbersome and tedious. Can the ODE be analyzed in such a way that will determine which ansatz ($1^{st}$, $2^{nd}$, $3^{rd}$, etc) will work?

\subsubsection{Principle of Superposition}
\label{sssec:principleOfSuperpositionAnsatz}

In all $g(t)$ seen thus far, all exponential coefficients have been the same. However, this will not always be the case. What should be done in $g(t)$ is a sum of functions with different exponential coefficients?

% \subsection{Variation of Parameters}
% \label{ssec:variationOfParameters}
%
% The most general way to obtain $Y$ is a \textbf{Variation of Parameters}.
%
% Knowing that $y_c = C_1y_1 + C_2y_2$, then $C_1$, $C_2$ are varriated into function such that:
% \begin{equation*}
%   Y = u_1y_1 + u_2y_2
% \end{equation*}
% Then plugging $Y$ into the non-homogeneous ODE:
% \[
%   \begin{cases} 
%     u_1'y_1 + u_2'y_2 = 0 \\
%     u_1'y_1' + u_2'y_2' = g(t)
%   \end{cases}
%   \Rightarrow u_1' = \frac{-y_2 \cdot g}{W(y_1,y_2)}, u_1' = \frac{y_1 \cdot g}{W(y_1,y_2)} \\
% \]
% \begin{equation*}
%   Y = y_1 \cdot \int_{}^{} \frac{-y_2 \cdot g}{W(y_1,y_2)} \,dt + y_2 \cdot \int_{}^{} \frac{y_1 \cdot g}{W(y_1,y_2)} \,dt
% \end{equation*}
% As can probably be seen, higher order generalization is very complicated. Thus introduces the \textbf{Method of Undetermined Coefficients}.

\end{document}
