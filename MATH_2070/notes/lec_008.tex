\documentclass[12pt]{article}

%%%% GRAPHICS %%%%
\usepackage{tikz}
\usepackage[siunitx, american, RPvoltages]{circuitikz}
\usetikzlibrary{arrows.meta}
\usepackage{tikz-3dplot}
\usepackage{graphicx}
\usepackage{pgfplots}
  \pgfplotsset{compat=1.18}
\usetikzlibrary{arrows}
\newcommand{\midarrow}{\tikz \draw[-triangle 90] (0,0) -- +(.1,0);}

%%%% FIGURES %%%%
\usepackage{subcaption}
\usepackage{wrapfig}
\usepackage{float}
\usepackage[skip=5pt, font=footnotesize]{caption}

%%%% FORMATTING %%%%
\usepackage{parskip}
\usepackage{tcolorbox}
\usepackage{ulem}

%%%% TABLE FORMATTING %%%%
\usepackage{tabularray}
\UseTblrLibrary{booktabs}

%%%% MATH AND LOGIC %%%%
\usepackage{xifthen}
\usepackage{amsmath}
\usepackage{amssymb}
\usepackage{amsfonts}

%%%% TEXT AND SYMBOLS %%%%
\usepackage[T1]{fontenc}
\usepackage{textcomp}
\usepackage{gensymb}

%%%% OTHER %%%%
\usepackage{standalone}

%%%% LOGIC SYMBOLS %%%%
\newcommand*\xor{\oplus}

%%%% STYLES %%%%

% Packages
\usepackage[paper=letterpaper,tmargin=45pt,bmargin=45pt,lmargin=45pt,rmargin=45pt]{geometry}
\usepackage{titlesec}
\usepackage[rgb]{xcolor}
\selectcolormodel{natural}
\usepackage{ninecolors}
\selectcolormodel{rgb}

% Colors
\definecolor{pg}{HTML}{24273A}
\definecolor{fg}{HTML}{FFFFFF}
\definecolor{bg}{HTML}{24273A}
\definecolor{re}{HTML}{F38BA8}
\definecolor{gr}{HTML}{A6E3A1}
\definecolor{ye}{HTML}{F9E2AF}
\definecolor{or}{HTML}{FAB387}
\definecolor{bl}{HTML}{89B4FA}
\definecolor{ma}{HTML}{CBA6F7}
\definecolor{cy}{HTML}{94E2D5}
\definecolor{pi}{HTML}{F2CDCD}

\definecolor{copper}{HTML}{B87333}

\usepackage{nameref}
\makeatletter
\newcommand*{\currentname}{\@currentlabelname}
\makeatother

\titleformat{\section}
  {\normalfont\scshape\Large\bfseries}
  {\thesection}
  {0.75em}
  {}

\titleformat{\subsection}
  {\normalfont\scshape\large\bfseries}
  {\thesubsection}
  {0.75em}
  {}

\titleformat{\subsubsection}
  {\normalfont\scshape\normalsize\bfseries}
  {\thesubsubsection}
  {0.75em}
  {}

% Formula
\newcounter{formula}[section]
\newenvironment{formula}[1]{
  \stepcounter{formula}
  \begin{tcolorbox}[
    standard jigsaw, % Allows opacity
    colframe={fg},
    boxrule=1px,
    colback=bg,
    opacityback=0,
    sharp corners,
    sidebyside,
    righthand width=18px,
    coltext={fg}
  ]
  \centering
  \textbf{\uline{#1}}
}{
  \tcblower
  \textbf{\thesection.\theformula}
  \end{tcolorbox}
}

% Definition
\newcounter{definition}[section]

\newenvironment{definition*}[1]{
  \begin{tcolorbox}[
    standard jigsaw, % Allows opacity
    colframe={fg},
    boxrule=1px,
    colback=bg,
    opacityback=0,
    sharp corners,
    coltext={fg}
  ]
  \textbf{#1 \hfill}
  \vspace{5px}
  \hrule
  \vspace{5px}
  \noindent
}{
  \end{tcolorbox}
}

\newenvironment{definition}[1]{
  \stepcounter{definition}
  \begin{tcolorbox}[
    standard jigsaw, % Allows opacity
    colframe={fg},
    boxrule=1px,
    colback=bg,
    opacityback=0,
    sharp corners,
    coltext={fg}
  ]
  \textbf{#1 \hfill \thesection.\thedefinition}
  \vspace{5px}
  \hrule
  \vspace{5px}
  \noindent
}{
  \end{tcolorbox}
}

% Example Problem
\newcounter{example}[section]
\newenvironment{example}{
  \stepcounter{example}
  \begin{tcolorbox}[
    standard jigsaw, % Allows opacity
    colframe={fg},
    boxrule=1px,
    colback=bg,
    opacityback=0,
    sharp corners,
    coltext={fg}
  ]
  \textbf{Example \hfill \thesection.\theexample}
  \vspace{5px}
  \hrule
  \vspace{5px}
  \noindent
}{
  \end{tcolorbox}
}

\tikzset{
  cubeBorder/.style=fg,
  cubeFilling/.style={fg!20!bg, opacity=0.25},
  gridLine/.style={very thin, gray},
  graphLine/.style={-latex, thick, fg},
}

\pgfplotsset{
  basicAxis/.style={
    grid,
    major grid style={line width=.2pt,draw=fg!50!bg},
    axis lines = box,
    axis line style = {line width = 1px},
  }
}

%%%% REFERENCES %%%%
\usepackage{hyperref}
\hypersetup{
  colorlinks  = true,
  linkcolor   = bl,
  anchorcolor = bl,
  citecolor   = bl,
  filecolor   = bl,
  menucolor   = bl,
  runcolor    = bl,
  urlcolor    = bl,
}

\author{Ethan Anthony}
\newcommand*{\equal}{=}


\title{Lecture 008}
\date{March 02, 2025}

\begin{document}
\newpage

\section{Laplace Transform}
\label{sec:laplaceTransform}

The general notation of a Laplace Transform is:
\begin{equation*}
  \mathcal{L}\big\{f(t)\big\} = F(s)
\end{equation*}
where $f(t)$ is an input function that will be transformed into some other function: $F(s)$. Notice, the variable each function depends on is different: $t \rightarrow s$.

The actual formula for computing the Laplace Transform is:
\begin{equation*}
  F(s) = \int_{0}^{\infty} e^{-st}f(t) \,dt
\end{equation*}

\subsection{Linearity of Laplace Transform}
\label{ssec:linearityOfLaplaceTransform}

For some real numbers $\alpha$ and $\beta$, consider the following Laplace Transform:
\begin{equation*}
  \mathcal{L}\big\{\alpha f(t) + \beta g(t)\big\}
\end{equation*}
By the formula of the Laplace Transform in Section \ref{sec:laplaceTransform}:
\begin{align*}
  \mathcal{L}\big\{\alpha f(t) + \beta g(t)\big\} &= \int_{0}^{\infty} \big( \alpha f(t) + \beta g(t) \big)e^{-st} \,dt \\
                                                  &= \int_{0}^{\infty} \alpha f(t) e^{-st} + \beta g(t) e^{-st} \, dt \\
                                                  &= \int_{0}^{\infty} \alpha f(t) e^{-st} \, dt + \int_{0}^{\infty} \beta g(t) e^{-st} \, dt \\
                                                  &= \alpha \int_{0}^{\infty} f(t) e^{-st} \, dt + \beta \int_{0}^{\infty} g(t) e^{-st} \, dt \\
                                                  &= \alpha \mathcal{L}\big\{ f(t) \big\} + \beta \mathcal{L}\big\{ g(t) \big\}
\end{align*}
Thus, it can be seen that, through the linearity of integrals, so too is the Laplace Transformation linear. In short:
\begin{equation*}
  \mathcal{L}\big\{\alpha f(t) + \beta g(t)\big\} = \alpha \mathcal{L}\big\{ f(t) \big\} + \beta \mathcal{L}\big\{ g(t) \big\}
\end{equation*}

\subsection{Basic Laplace Transform Formulas}
\label{ssec:basicLaplaceTransformFormulas}

For \textbf{power functions}, the Laplace Transform is:
\begin{equation*}
  \mathcal{L}\big\{ t^n \big\} = \frac{n!}{s^{n+1}},\ s>0
\end{equation*}
This formula, in tandem with the linearity of Laplace Transforms, can be used to find the Laplace Transform of any polynomial function.

For any \textbf{exponential function}:
\begin{equation*}
  \mathcal{L}\big\{ e^{\alpha t} \big\} = \frac{1}{s- \alpha},\ s>\alpha
\end{equation*}

For \textbf{sine} and \textbf{cosine}:
\begin{align*}
  \mathcal{L}\big\{\cos(\beta t) \big\} &= \frac{s}{s^2 + \beta^2},\ s>0 \\
  \mathcal{L}\big\{\sin(\beta t) \big\} &= \frac{\beta}{s^2 + \beta^2},\ s>0
\end{align*}

For \textbf{hyperbolic} functions:
\begin{align*}
  \mathcal{L}\big\{\cosh(kt) \big\} &= \frac{s}{s^2 - k^2},\ s>k \\
  \mathcal{L}\big\{\sinh(kt) \big\} &= \frac{k}{s^2 - k^2},\ s>k
\end{align*}

\subsection{Existence of the Laplace Transform}
\label{ssec:existenceOfTheLaplaceTransform}

Prior to defining the existence of $\mathcal{L}\big\{f(t)\big\}$, \textbf{piecewise continuity} and \textbf{exponential order} must be understood.

\begin{definition}{Piecewise Continuity}
  A function is said to be piecewise continuous over some interval $[0,\infty)$ if, for any subinterval $[a,b]$, there is a finite number of discontinuities.
\end{definition}

These discontinuities can be either hole or jump discontinuities since those exist at finite points. Consider the function:
\begin{equation*}
  f(t) = \frac{t^2}{t-2}
\end{equation*}
This function is \textit{piecewise} continuous over $[0,\infty)$ since there only exists a discontinuity at $t=2$, which is a finite number of discontinuities ($1$). Now, consider the function:
\begin{equation*}
  f(t) = \sqrt{t-10}
\end{equation*}
Since this function is undefined when $t<10$, the function is discontinuous at infinitely many numbers over the range $(0,10)$. Thus, this function is \textit{not} piecewise continuous.

\begin{definition}{Exponential Order}
  A function is said to be of exponential order if there exist constants {\color{re} $c$}, {\color{gr} $M>0$}, and {\color{bl} $T>0$} such that:
  \begin{equation*}
    \forall t>{\color{bl} T},\ \big|f(t)\big| \leq {\color{gr} M}e^{{\color{re} c}t}
  \end{equation*}
\end{definition}
In other words, \big|f(t)\big| must not grow faster that some generic exponential function with some positive coefficient.

Polynomial functions, exponential functions of the form $e^{\alpha t}$, cosine, and sine functions are all of exponential order. However, consider the function:
\begin{equation*}
  f(t) = e^{t^{2}}
\end{equation*}
This function is \textit{not} of exponential order since, for every possible $c>0$:
\begin{equation*}
  \left|\frac{e^{t^{2}}}{e^{ct}}\right| = e^{t^2-ct} = e^{t(t-c)} \rightarrow \infty
\end{equation*}
and thus $e^{t^{2}}$ grows faster than any possible multiple of $e^{ct}$ and is shown to \textit{not} be of exponential order.

With these two ideas covered, $\mathcal{L}\big\{f(t)\big\}$ exists for $s>c$ given that $f(t)$ is piecewise continuous over $[0,\infty)$ and is of exponential order. These two conditions are sufficient, but not necessary.

\subsection{Long Term Behavior of the Laplace Transform}
\label{ssec:longTermBehaviorOfTheLaplaceTransform}

Continuing with piecewise continuity and exponential order, it can be said that if a function $f(t)$ is both piecewise continuous and of exponential order, then the function $F(s) = \mathcal{L}\big\{f(t)\big\}$ satisfies:
\begin{equation*}
  \lim_{s \to \infty} F(s) = 0
\end{equation*}

\subsection{Inverse Laplace Transform}
\label{ssec:inverseLaplaceTransform}

Just as any other inverse function, the inverse Laplace Transform will essentially reverse the process of the Laplace Transformation. If $F(s)$ represents the Laplace Transform of $f(t)$, then $f(t)$ is the inverse Laplace Transform of $F(s)$.
\begin{equation*}
  \mathcal{L}\big\{f(t)\big\} = F(s) \rightarrow \mathcal{L}^{-1}\big\{F(s)\big\} = f(t)
\end{equation*}

\subsubsection{Linearity of the Inverse Laplace Transform}
\label{sssec:linearityOfTheInverseLaplaceTransform}
The inverse Laplace Transform maintains that same linearity as the Laplace Transform does. As such:
\begin{equation*}
  \mathcal{L}^{-1}\big\{\alpha F(s) + \beta G(s)\big\} = \alpha \mathcal{L}^{-1}\big\{ F(s) \big\} + \beta \mathcal{L}^{-1}\big\{ G(s) \big\}
\end{equation*}

\subsubsection{Basic Inverse Laplace Transform Formulas}
\label{sssec:basicInverseLaplaceTransformFormulas}
Furthermore, just as there are fundamental Laplace Transforms that should be memorized and immediately recognized, so too are the inverse Laplace Transforms to be memorized (they're the same ones, just in reverse).

For \textbf{power functions}:
\begin{equation*}
  \mathcal{L}^{-1}\left\{\frac{1}{s^n}\right\} = \frac{t^{n-1}}{(n-1)!}
\end{equation*}

For \textbf{exponential functions}:
\begin{equation*}
  \forall \alpha \in \mathbb{R},\ \mathcal{L}^{-1}\left\{\frac{1}{s-\alpha}\right\} = e^{\alpha t}
\end{equation*}

For \textbf{sine} and \textbf{cosine}:
\begin{align*}
  \forall \beta \in \mathbb{R},\ \beta > 0 \rightarrow \mathcal{L}^{-1}\left\{\frac{s}{s^2+\beta^2}\right\} &= \cos(\beta t) \\
  \forall \beta \in \mathbb{R},\ \beta > 0 \rightarrow \mathcal{L}^{-1}\left\{\frac{1}{s^2+\beta^2}\right\} &= \frac{1}{\beta}\sin(\beta t)
\end{align*}

For \textbf{hyperbolic functions}:
\begin{align*}
  \forall k \in \mathbb{R},\ k > 0 \rightarrow \mathcal{L}^{-1}\left\{\frac{s}{s^2-k^2}\right\} &= \cosh(kt) \\
  \forall k \in \mathbb{R},\ k > 0 \rightarrow \mathcal{L}^{-1}\left\{\frac{1}{s^2-k^2}\right\} &= \frac{1}{k}\sinh(kt)
\end{align*}

\subsection{Laplace Transform of Derivatives}
\label{ssec:laplaceTransformOfDerivatives}

If $f(t)$ is continuous on the interval $[0,\infty)$, of exponential order, and if $f'(t)$ is piecewise continuous on $[0,\infty)$, then:
\begin{equation*}
  \mathcal{L}\big\{f'(t)\big\} = s\mathcal{L}\big\{f(t)\big\} - f(0) = sF(s) - f(0)
\end{equation*}

To prove this, consider the following Laplace Transform:
\begin{equation*}
  \mathcal{L}\big\{f'(t)\big\} = \int_{0}^{\infty} f'(t)e^{-st} \,dt
\end{equation*}
Through integration by parts, this can be written as:
\begin{align*}
  \int_{0}^{\infty} f'(t)e^{-st} \,dt &= \big[f(t)e^{-st}\big]_{0}^{\infty} - \int_{0}^{\infty} -s \cdot f(t)e^{-st} \,dt \\
                                      &= {\color{ma} \big[f(t)e^{-st}\big]_{0}^{\infty}} + s {\color{gr} \int_{0}^{\infty} f(t)e^{-st} \,dt} \\
                                      &= {\color{ma} \left(\lim_{b \to \infty} f(b)e^{-sb} - f(0)e^0\right)} + s {\color{gr} F(s)}
\end{align*}

Under the assumption that $f(t)$ is of exponential order, for a sufficiently large value of $s$:
\begin{equation*}
  \lim_{t \to \infty} f(t)e^{-st} = 0
\end{equation*}
thus showing that:
\begin{equation*}
  \mathcal{L}\big\{f'(t)\big\} = sF(s) - f(0)
\end{equation*}

This process of proof can be used to prove:
\begin{align*}
  \mathcal{L}\big\{f''(t)\big\} &= s^2F(s) - sf(0) - f'(0) \\
  \mathcal{L}\big\{f'''(t)\big\} &= s^3F(s) - s^2f(0) - sf'(0) - f''(0) \\
                                 &\vdots \\
  \mathcal{L}\big\{f^{(n)}(t)\big\} &= s^nF(s) - s^{n-1}f(0) - s^{n-2}f'(0) - \hdots - f^{(n-1)}(0)
\end{align*}
Provided that, for the Laplace Transform of some derivative $f^{n}(t)$, $f^{n}(t)$ is piecewise continuous and all previous functions $\big[f(t),\ f'(t), \hdots\big]$ are continuous over $[0,\infty)$ and of exponential order. The Laplace of the $n^{th}$ derivative can also be expressed as:
\begin{equation*}
  \mathcal{L}\big\{f^{(n)}(t)\big\} = s^nF(s) - \sum_{i=0}^{n-1} s^{n-1-i}f^{(i)}(0)
\end{equation*}

\subsection{First Translation Theorem}
\label{ssec:firstTranslationTheorem}
\begin{definition}{First Translation Theorem}
  If $\mathcal{L}\big\{f(t)\big\}=F(s)$, then
  \begin{equation*}
    \mathcal{L}\big\{e^{at}f(t)\big\}=F(s-a)
  \end{equation*}
  or
  \begin{equation*}
    \mathcal{L}\big\{e^{at}f(t)\big\}=\mathcal{L}\big\{f(t)\big\}\Big\vert_{s\mapsto s-a}
  \end{equation*}
\end{definition}
Consider the Laplace Transform of:
\begin{equation*}
  \mathcal{L}\big\{e^{at}f(t)\big\}
\end{equation*}
This can be rewritten as:
\begin{equation*}
  \int_{0}^{\infty} e^{at}f(t)e^{-st} \,dt = \int_{0}^{\infty} f(t)e^{-(s-a)t} \, dt = F(s-a)
\end{equation*}
What this shows is that, when a function $f(t)$ is multiplied by an exponential function of $e^{at}$, the resulting Laplace Transform will be whatever the Laplace Transform of $f(t)$ is, with $s \mapsto s-a$.

The First Translation Theorem can be used to transform a function with an exponential factor.

\begin{example}{Laplace Transform with First Translation Theorem}
  Evaluate $\mathcal{L}\big\{e^{5t}t^3\big\}$:
  \begin{equation*}
    \mathcal{L}\big\{e^{5t}t^3\big\} = \mathcal{L}\big\{t^3\big\}\Big\vert_{s \mapsto s-5} = \frac{3!}{s^4}\Big\vert_{s \mapsto s-5} = \frac{6}{\left(s-5\right)^4}
  \end{equation*}
  \hrule
  \vspace{12pt}
  Evaluate $\mathcal{L}\big\{e^{-3t}sin(6t)\big\}$:
  \begin{equation*}
    \mathcal{L}\big\{e^{-3t}sin(6t)\big\} = \mathcal{L}\big\{sin(6t)\big\}\Big\vert_{s \mapsto s+3} = \frac{6}{s^2 + 6^2}\Big\vert_{s \mapsto s+3} = \frac{6}{(s+3)^2 + 36}
  \end{equation*}
\end{example}

It can also be used in reverse to inverse Laplace Transform a fraction with repeated linear factors:
\begin{example}{Inverse Laplace Transform with First Translation Theorem}
  Evaluate $\mathcal{L}^{-1}\left\{\frac{2s+5}{(s-3)^2}\right\}$:

  First, the fraction is broken into partial fractions:
  \begin{equation*}
    \frac{2s+5}{(s-3)^2} = \frac{A}{s-3} + \frac{B}{(s-3)^2} \rightarrow \hdots \rightarrow A = 2,\ B = 11 \rightarrow \frac{2s+5}{(s-3)^2} = \frac{2}{s-3} + \frac{11}{(s-3)^2}
  \end{equation*}
  Therefore:
  \begin{equation*}
    \mathcal{L}^{-1}\left\{\frac{2s+5}{(s-3)^2}\right\} = \mathcal{L}^{-1}\left\{\frac{2}{s-3}\right\} + \mathcal{L}^{-1}\left\{\frac{11}{(s-3)^2}\right\}
  \end{equation*}
  Then, by applying the First Translation Theorem:
  \begin{equation*}
    \mathcal{L}^{-1}\left\{\frac{2}{s}\right\}{\color{gr} \Big\vert_{s \mapsto s-3}} + \mathcal{L}^{-1}\left\{\frac{11}{s^2}\right\}{\color{gr} \Big\vert_{s \mapsto s-3}} = {\color{gr} e^{3t}}{\color{re} \mathcal{L}^{-1}\left\{\frac{{\color{ma} 2}}{s}\right\}} + {\color{gr} e^{3t}}{\color{re} \mathcal{L}^{-1}\left\{\frac{{\color{ma} 11}}{s^2}\right\}} = {\color{gr} e^{3t}}\cdot {\color{ma} 2} \cdot {\color{re} 1} + {\color{gr} e^{3t}} \cdot {\color{ma} 11} \cdot {\color{re} t}
  \end{equation*}
  Thus:
  \begin{equation*}
    \mathcal{L}^{-1}\left\{\frac{2s+5}{(s-3)^2}\right\} = 2e^{3t} + 11te^{3t}
  \end{equation*}
\end{example}
\subsubsection{Application on Fraction With an Irreducible Quadratic Denominator}
\label{sssec:}
The process in the previous example can be generalized into a process that will work with any fraction in the form of (under the assumption that the denominator is irreducible/has no real roots):
\begin{equation*}
  \frac{As+B}{as^2+bs+c}
\end{equation*}
where the general process is completing the square of the denominator to find {\color{gr} $\alpha$} (the number to shift by) and {\color{re} $\beta$} (the trigonometric coefficient), process the numerator to get it into the form of $s- {\color{gr} \alpha}+C$, then separating the numerator such that $s- {\color{gr} \alpha}$ is one fraction and $C$ is another. From there, each fraction can be inverse Laplace Transformed on its own due to the linearity of the inverse Laplace Transform:

\textbf{1) Complete the Square of the Denominator:}
\begin{equation*}
  as^2+bs+c = a\left(s^2+ \frac{b}{a}s+\left(\frac{b}{2a}\right)^2-\left(\frac{b}{2a}\right)^2+\frac{c}{a}\right) = a\left( \left(s+\frac{b}{2a}\right)^2 + \frac{-b^2 + 4ac}{4a^2} \right)
\end{equation*}
During this process, it is assumed that the denominator is \textit{irreducible}, meaning that it has no real roots. Because of this, it can be stated that $-b^2+4ac>0$. Thus:
\begin{equation*}
  {\color{gr} \alpha} = -\frac{b}{2a},\ {\color{re} \beta}=\frac{\sqrt{-b^2+4ac}}{2a}
\end{equation*}
\textbf{2) Process the Numerator to Find $s-{\color{gr} \alpha}$}
\begin{equation*}
  As+B = A(s-\alpha+\alpha)+B = {\color{gr} A(s-\alpha)}+{\color{re} (A\alpha+B)}
\end{equation*}
\textbf{3) Putting it All Together}
\begin{equation*}
  \frac{As+B}{as^2+bs+c} = \frac{{\color{gr} A(s-\alpha)}}{a\big[(s-\alpha)^2+\beta^2\big]} + \frac{{\color{re} A \alpha+B}}{a\big[(s-\alpha)^2+\beta^2\big]} = \frac{A}{a} \cdot \frac{(s-\alpha)}{(s-\alpha)^2+\beta^2} + \frac{A \alpha + B}{a} \cdot \frac{1}{(s-\alpha)^2+\beta^2}
\end{equation*}
\textbf{4) Inverse Laplace Transform the Fractions}
\begin{align*}
     &\frac{A}{a} \cdot \mathcal{L}^{-1}\left\{\frac{(s-\alpha)}{(s-\alpha)^2+\beta^2}\right\} + \frac{A \alpha + B}{a} \cdot \mathcal{L}^{-1}\left\{\frac{1}{(s-\alpha)^2+\beta^2}\right\} \\ \\
  =\ &\frac{A}{a} \cdot \mathcal{L}^{-1}\left\{\frac{s}{s^2+\beta^2}\right\}\Big\vert_{s \mapsto s-\alpha} + \frac{A \alpha + B}{a} \cdot \mathcal{L}^{-1}\left\{\frac{1}{s^2+\beta^2}\right\}\Big\vert_{s \mapsto s-\alpha} \\ \\
  =\ &\frac{A}{a} e^{\alpha t} \cdot \mathcal{L}^{-1}\left\{\frac{s}{s^2+\beta^2}\right\} + \frac{A \alpha + B}{a} e^{\alpha t} \cdot \mathcal{L}^{-1}\left\{\frac{1}{s^2+\beta^2}\right\} \\ \\
  =\ &\frac{A}{a} e^{\alpha t} \cos(\beta t) + \frac{A \alpha + B}{a} \cdot \frac{1}{\beta} e^{\alpha t} \sin(\beta t)
\end{align*}

\subsection{Unit Step Function}
\label{ssec:unitStepFunction}

\begin{wrapfigure}[]{r}{0.3\textwidth}
  \vspace{-20pt}
  \centering
  \includestandalone{figures/fig_012}
  \caption{Unit Step Function}
  \label{fig:012}
\end{wrapfigure}

The unit step function is defined as:
\begin{equation*}
  \mathcal{U}(t-a)=\begin{cases}
    0,\ 0 \leq t < a \\
    1,\ a \leq t \\
  \end{cases}
\end{equation*}
Any piecewise function can be expressed as a sum of functions with unit step functions. Consider the following generic piecewise function:
\begin{equation*}
  f(t) = \begin{cases}
    f_1(t),\ 0\leq t < a \\
    f_2(t),\ t\leq a \\
  \end{cases}
\end{equation*}
This function can be expressed as a sum of functions using the unit step function as follows:
\begin{equation*}
  f(t) = f_1(t)+\big({\color{re} f_2(t)-f_1(t)}\big)\mathcal{U}(t-a)
\end{equation*}
The general concept is that, prior to $t$ reaching $a$, $\mathcal{U}(t-a)=0$, so the {\color{re} $f_2(t)-f_1(t)$} has a coefficient of $0$, and thus has no effect on the value of $f(t)$. However, upon $t$ reaching $a$, $\mathcal{U}(t-a)=1$, and thus the value of $f(t)$ is modified by the difference between $f_1(t)$ and $f_2(t)$.

In other words, once the unit step function's $a$ is reached, the value of $f(t)$ changes from $f_1(t)$ to $f_2(t)$ by adding the difference between the two sub-functions. This "transformation" of a piecewise function can be done an arbitrary number of times:
\begin{equation*}
  f(t) = \begin{cases}
    f_1(t),\ 0 \leq t < a \\
    f_2(t),\ a \leq t < b \\
    f_3(t),\ b \leq t < c \\
    f_4(t),\ c \leq t < d \\
    f_5(t),\ d \leq t \\
  \end{cases}
\end{equation*}
can be equivalently expressed as:
\begin{equation*}
  f_1(t)+\big(f_2(t)-f_1(t)\big)\mathcal{U}(t-a)+\big(f_3(t)-f_2(t)\big)\mathcal{U}(t-b)+\big(f_4(t)-f_3(t)\big)\mathcal{U}(t-c)+\big(f_5(t)-f_4(t)\big)\mathcal{U}(t-d)
\end{equation*}
Furthermore, a function given in terms of unit step functions can be translated back into a piecewise function:
\begin{equation*}
  f(t) = g_1(t) + g_2(t)\mathcal{U}(t-a) + g_3(t)\mathcal{U}(t-b)
\end{equation*}
becomes
\begin{align*}
  f(t) = \begin{cases}
    g_1(t)        ,&\ 0 \leq t < a \\
    g_1+g_2(t)    ,&\ a \leq t < b \\
    g_1+g_2+g_3(t),&\ b \leq t \\
  \end{cases}
\end{align*}

\subsubsection{Second Translation Theorem}
\label{sssec:secondTranslationTheorem}

Considering some function shifted by some value $a$ with the unit step function:

\begin{figure}[H]
  \centering
  \begin{subfigure}[H]{0.45\textwidth}
    \centering
    \includestandalone{figures/fig_013}
    \caption{$y=f(t)$}
    \label{fig:013}
  \end{subfigure}
  \begin{subfigure}[H]{0.45\textwidth}
    \centering
    \includestandalone{figures/fig_014}
    \caption{$y=f(t-a)\mathcal{U}(t-a)$}
    \label{fig:014}
  \end{subfigure}
\end{figure}
the \textbf{Second Translation Theorem} is:

\begin{definition}{Second Translation Theorem}
  If $\mathcal{L}\left\{f(t)\right\}=F(s)$, then
  \begin{equation*}
    \mathcal{L}\left\{f(t-a)\mathcal{U}(t-a)\right\}=e^{-as}F(s)
  \end{equation*}
\end{definition}

The proof of which is as follows. Considering some function $\mathcal{L}\left\{f(t-a)\mathcal{U}(t-a)\right\}$, the Laplace Transform of the function is:
\begin{equation*}
  \int_{0}^{\infty} f(t-a)\mathcal{U}(t-a)e^{-st} \,dt = \int_{a}^{\infty} f(t-a)\mathcal{U}(t-a)e^{-st} \,dt
\end{equation*}
this equivalency is valid since, up until $t=a$, the value of the entire function is $0$ since the unit step function has yet to be "activated". Then, mapping $(t-a) \mapsto T$:
\begin{equation*}
  \int_{0}^{\infty} f(T)\mathcal{U}(T){\color{gr} e^{-s(T+a)}} \,dt = \int_{0}^{\infty} f(T){\color{re} \mathcal{U}(T)}{\color{gr} e^{-sT-sa}} \,dT = {\color{gr} e^{-sa}}\int_{0}^{\infty} f(T)\cdot{\color{re} 1}\cdot{\color{gr} e^{-sT}} \,dT = e^{-as}F(s)
\end{equation*}
Thus:
\begin{equation*}
  \mathcal{L}\left\{f(t-a)\mathcal{U}(t-a)\right\}=e^{-as}F(s)\ \ \ \textup{or}\ \ \ \mathcal{L}\left\{f(t)\mathcal{U}(t-a)\right\}=e^{-as}\mathcal{L}\left\{f(t+a)\right\}
\end{equation*}

\begin{example}{IVP Using Laplace Transform and Unit Step Function}
  Consider the IVP:
  \begin{equation*}
    y'+y=f(t),\ y(0)=5,\ f(t)=\begin{cases}
      0,&\ 0\leq t<\pi \\
      3\cos(t),&\ \pi< t \\
    \end{cases}
  \end{equation*}
  First, rewrite the RHS as a function using the step function $\mathcal{U}(t)$:
  \begin{equation*}
    f(t) = 0 + (3\cos(t)-0)\mathcal{U}(t-\pi) = 3\cos(t)\cdot\mathcal{U}(t-\pi)
  \end{equation*}
  Then, with $Y(s)=\mathcal{L}\left\{y(t)\right\}$, find the Laplace Transform of the equation:
  \begin{gather*}
    \mathcal{L}\left\{y'+y\right\} = sY(s)-y(0)+Y(s) = Y(s)(s+1)-5 \\
    \mathcal{L}\left\{3\cos(t)\cdot\mathcal{U}(t-\pi)\right\} = 3e^{-\pi t}\mathcal{L}\left\{\cos(t+\pi)\right\} = -3e^{-\pi t}\mathcal{L}\left\{\cos(t)\right\} = -3e^{-\pi t}\frac{s}{s^2+1} \\
  Y(s)(s+1)-5 = -3e^{-\pi t}\frac{s}{s^2+1}
  \end{gather*}
  Solve algebraically for $Y(s)$:
  \begin{align*}
    Y(s)(s+1)-5 = -3e^{-\pi t}\frac{s}{s^2+1} \rightarrow Y(s) = -3e^{-\pi t}\frac{s}{(s^2+1)(s+1)} + \frac{5}{s+1}
  \end{align*}
  Find the inverse Laplace Transform of the equation to find $y(t)$:
  \begin{equation*}
    \mathcal{L}^{-1}\left\{-3e^{-\pi t}\frac{s}{(s^2+1)(s+1)} + \frac{5}{s+1}\right\} = {\color{gr} \mathcal{L}^{-1}\left\{-3e^{-\pi t}\frac{s}{(s^2+1)(s+1)}\right\}} + {\color{re} \mathcal{L}^{-1}\left\{\frac{5}{s+1}\right\}}
  \end{equation*}
  For the first Inverse Laplace:
  \begin{gather*}
    {\color{gr} \mathcal{L}^{-1}\left\{-3e^{-\pi t}\frac{s}{(s^2+1)(s+1)}\right\}} = \mathcal{L}^{-1}\left\{\frac{-3s}{(s^2+1)(s+1)}\right\}\Big\vert_{t \mapsto t-\pi} \mathcal{U}(t-\pi) \\
    = \mathcal{L}^{-1}\left\{\frac{\frac{3}{2}}{s+1}-\frac{\frac{3}{2}s}{s^2+1}-\frac{\frac{3}{2}}{s^2+1}\right\}\Big\vert_{t \mapsto t-\pi} \mathcal{U}(t-\pi) = \left(\frac{3}{2}e^{-t}-\frac{3}{2}\cos(t)-\frac{3}{2}\sin(t)\right)\Big\vert_{t \mapsto t-\pi}\mathcal{U}(t-\pi) \\
    \left(\frac{3}{2}e^{-(t-\pi)}+\frac{3}{2}\cos(t)+\frac{3}{2}\sin(t)\right)\mathcal{U}(t-\pi)
  \end{gather*}
  For the second Inverse Laplace:
  \begin{equation*}
    {\color{re} \mathcal{L}^{-1}\left\{\frac{5}{s+1}\right\}} = 5e^{-t}
  \end{equation*}
  Giving a final function of:
  \begin{equation*}
    y(t) = \left(\frac{3}{2}e^{-(t-\pi)}+\frac{3}{2}\cos(t)+\frac{3}{2}\sin(t)\right)\mathcal{U}(t-\pi) + 5e^{-t}
  \end{equation*}
  Which can be translated into a piecewise function:
  \begin{equation*}
    y(t) = \begin{cases}
      5e^{-t},&\ 0 \leq t < \pi \\
      5e^{-t}+\frac{3}{2}e^{-(t-\pi)}+\frac{3}{2}\cos(t)+\frac{3}{2}\sin(t),&\ \pi \leq t
    \end{cases}
  \end{equation*}
\end{example}

\subsection{Periodic Functions}
\label{ssec:periodicFunctions}

\begin{definition}{Periodic Function}
  A function $f(t)$ is said to be periodic if it repeats over some period. Specifically, a function is $T$-periodic if $\forall t \in D,\ f(t+T)=f(t)$ (where $D$ is the domain of $f$) and $T$ is the smallest possible number satisfying this equality.
\end{definition}



Consider some function $f(t)$ that is $T$-periodic on $[0,\infty)$. Finding the Laplace Transform:
\begin{equation*}
  \mathcal{L}\left\{f(t)\right\} = \int_{0}^{\infty} f(t)e^{-st} \,dt = {\color{or} \int_{0}^{T} f(t)e^{-st} \,dt} + {\color{gr} \int_{T}^{\infty} f(t)e^{-st} \,dt}
\end{equation*}
Focusing on the second integral, because $\forall t>T,\ f(t)=f(t-T)$:
\begin{equation*}
  {\color{gr} \int_{T}^{\infty} f(t)e^{-st} \,dt} = \int_{T}^{\infty} f(t-T)e^{-s(t-T+T)} \,dt = {\color{ma} e^{-sT}}{\color{re} \int_{T}^{\infty} f(t-T)e^{-s(t-T)} \,dt}
\end{equation*}
If some dummy variable $u=t-T$, then:
\begin{equation*}
  {\color{re} \int_{T}^{\infty} f(t-T)e^{-s(t-T)} \,dt} = \int_{0}^{\infty} f(u)e^{-su} \,dt = \mathcal{L}\left\{f(u)\right\} = {\color{ma} \mathcal{L}\left\{f(t)\right\}}
\end{equation*}
Therefore, putting both integrals back together:
\begin{equation*}
  \mathcal{L}\left\{f(t)\right\} = {\color{or} \int_{0}^{T} f(t)e^{-st} \,dt} + {\color{ma} e^{-sT}\mathcal{L}\left\{f(t)\right\}}
\end{equation*}
Then, solving for $\mathcal{L}\left\{f(t)\right\}$:
\begin{equation*}
  \mathcal{L}\left\{f(t)\right\} = \frac{1}{1-s^{-sT}}\int_{0}^{T} f(t)e^{-st} \,dt
\end{equation*}
\begin{definition}{Laplace Transform of a Periodic Function}
  The Laplace Transform of a $T$-periodic function $f(t)$ is:
  \begin{equation*}
    \mathcal{L}\left\{f(t)\right\} = \frac{1}{1-s^{-sT}}\int_{0}^{T} f(t)e^{-st} \,dt
  \end{equation*}
\end{definition}
\begin{wrapfigure}[5]{r}{0.45\textwidth}
  \centering
  \vspace{-10pt}
  \includestandalone{figures/fig_015}
  \caption{Square Wave Function}
  \label{fig:015}
\end{wrapfigure}
Consider a square wave function $E(t)$, a $2$-periodic function that is extended from the following piecewise function:
\begin{equation*}
  E(t) = \begin{cases}
    1,&\ 0 \leq t < 1 \\
    0,&\ 1 \leq t < 2
  \end{cases}
\end{equation*}
Using the shown method of finding the Laplace Transform of a periodic function:
\begin{equation*}
  \mathcal{L}\left\{E(t)\right\} = \frac{1}{1-e^{-2s}}\int_{0}^{2} E(t)e^{-st} \,dt = \frac{1}{1-e^{-2s}}\int_{0}^{1} 1 \cdot e^{-st} \,dt = \frac{1}{1-e^{-2s}}\left[-\frac{1}{s}e^{-st}\right]_0^1 = \frac{1}{s\left(1+e^{-s}\right)}
\end{equation*}

\end{document}
