\documentclass[12pt]{article}

\input{../../xlatex/imports/preamble}

\title{Lecture 002}
\date{January 12, 2025}

\begin{document}
\setcounter{equation}{0}
\newpage

% \subsection{Implicit and Explicit Solutions}
% \label{ssec:implicitAndExplicitSolutions}
%
% In general, anytime an ODE is given \textit{without} initial values, then a single-parameter family of \textit{implicit} solutions is sufficient.
%
% If given an IVP, however, then finding an \textit{explicit} solution and an \textit{interval of existence} should be attempted. In some situations, this won't be possible, but it should be attempted at least.
%
% \subsubsection{Implicit Solutions}
% \label{sssec:implicitSolution}
%
% Sometimes a wall is reached even if the integration is possible. Consider:
% \begin{equation}
%   y' = \frac{xy}{y^2 + 1}
%   \label{eq:204}
% \end{equation}
% Using the technique described in Subsection \ref{sssec:separableODE}, this can be separated into:
%
% \begin{align*}
%   y' &= \frac{xy}{y^2 + 1} \\
%   \frac{dy}{dx} &= \frac{xy}{y^2 + 1} \\
%   \frac{y^2 + 1}{y}dy  &= x\ dx \\
%   \left(y+\frac{1}{y}\right)dy  &= x\ dx
% \end{align*}
% Integrating both sides gives:
% \begin{equation*}
%   \frac{y^2}{2} + \ln|y| = \frac{x^2}{2} + C
% \end{equation*}
% The integration of this equation is quite simple. However, try to solve for $y$ and see how difficult that will be. Though solving for $y$ itself is too difficult, this form is still a solution and can still be verified.
% \begin{align*}
%   \frac{d}{dx}\left(\frac{y^2}{2} + \ln|y|\right) &= \frac{d}{dx}\left(\frac{x^2}{2} + C\right) \\
%   y'\left(y + \frac{1}{y}\right) &= x \\
%   y \cdot \left(y'\left(y + \frac{1}{y}\right) \right) &= y \cdot x \\
%   y'\left(y^2 + 1\right) &= y \cdot x \\
%   y' &= \frac{xy}{y^2 + 1}
% \end{align*}
% Producing the exact same equation in (\ref{eq:204}), thus verifying the solution.
%
% Since these solutions are implicit, they might not be able to be graphed as a valid function. In those cases, other information, such as an initial condition, can be used to further inform the appropriate solution.

\subsection{Autonomous ODE}
\label{ssec:autonomousODE}

\begin{equation*}
  y' = f(y)
\end{equation*}
Such that the right hand side (RHS) does not involve $t$. In other words, it is a function purely in terms of $y$.

The equilibrium solution:
\begin{equation*}
  y = y_0
\end{equation*}
Such that:
\begin{equation*}
  f(y_0) = 0
\end{equation*}
This means that the equilibrium solutions must be a value such that $y'(y_0)=0$.
\begin{example}{Autonomous ODE}
  \begin{equation*}
    y' = y^3 - y
  \end{equation*}
  Clearly, the RHS is purely in terms of $y$, so this would be able to be solved as an autonomous ODE. So, taking the RHS and solving for $0$:
  \begin{align*}
    y^3 - y &= 0 \\
    y\left(y^2 - 1\right) &= 0 \\
    y\left(y - 1\right)\left(y + 1\right) &= 0 \\
    y = 0;\ y = 1;\ y &= -1
  \end{align*}
  Thus, there are three equilibrium solutions.
  \begin{gather*}
    @y=0\\
    y' = y\left(y - 1\right)\left(y + 1\right) \Rightarrow 0\left(0 - 1\right)\left(0 + 1\right) = 0 \cdot -1 \cdot 1 = 0
  \end{gather*}
  \begin{gather*}
    @y=1\\
    y' = y\left(y - 1\right)\left(y + 1\right) \Rightarrow 1\left(1 - 1\right)\left(1 + 1\right) = 1 \cdot 0 \cdot 2 = 0
  \end{gather*}
  \begin{gather*}
    @y=-1\\
    y' = y\left(y - 1\right)\left(y + 1\right) \Rightarrow -1\left(-1 - 1\right)\left(-1 + 1\right) = -1 \cdot -2 \cdot 0 = 0
  \end{gather*}
\end{example}

\subsubsection{Stability of an Equilibrium Solution}
\label{sssec:stabilityOfAnEqulibriumSolution}

\begin{definition}{Stability}
  Let $y=y_0$ be an equilibrium solution of $y' = f(y)$. $y=y_0$ is \textit{stable from above} if, for every $y>y_0$ near $y_0$, $f(y)<0$.
\end{definition}

\begin{figure}[H]
  \centering
  \begin{subfigure}[H]{0.45\textwidth}
    \centering
    \includestandalone{figures/fig_002}
    \caption{Stable From Above}
    \label{fig:002}
  \end{subfigure}
  \begin{subfigure}[H]{0.45\textwidth}
    \centering
    \includestandalone{figures/fig_003}
    \caption{Stable From Below}
    \label{fig:003}
  \end{subfigure}
  \caption{Stability}
  \label{fig:stability}
  \vspace{-10pt}
\end{figure}
Using Figure \ref{fig:002} as a visual, the {\color{gr} solution} is stable from above because, if you were to deviate from the solution, the direction field above will guide you back down to the solution. Figure \ref{fig:003} is stable from below for the same reasons.

It follows then, that if the direction field around the solution points \textit{away} from the solution, then the solution is considered to be \textit{un}stable. See this in Figure \ref{fig:instability}.

\begin{figure}[H]
  \centering
  \begin{subfigure}[H]{0.45\textwidth}
    \centering
    \includestandalone{figures/fig_004}
    \caption{Unstable From Above}
    \label{fig:004}
  \end{subfigure}
  \begin{subfigure}[H]{0.45\textwidth}
    \centering
    \includestandalone{figures/fig_005}
    \caption{Unstable From Below}
    \label{fig:005}
  \end{subfigure}
  \caption{Instability}
  \label{fig:instability}
  \vspace{-10pt}
\end{figure}

When considering a solutions behavior from both sides (top and bottom), it can be said to be either stable, unstable, or semistable.

\begin{figure}[H]
  \centering
  \begin{subfigure}[H]{0.30\textwidth}
    \centering
    \includestandalone{figures/fig_006}
    \caption{Stable}
    \label{fig:006}
  \end{subfigure}
  \begin{subfigure}[H]{0.30\textwidth}
    \centering
    \includestandalone{figures/fig_007}
    \caption{Unstable}
    \label{fig:007}
  \end{subfigure}
  \begin{subfigure}[H]{0.30\textwidth}
    \centering
    \includestandalone{figures/fig_008}
    \caption{Semistable}
    \label{fig:008}
  \end{subfigure}
  \caption{Types of Stability}
  \label{fig:typesOfStability}
\end{figure}

\begin{example}{Autonomous ODE: Real World Example}
  Falling object from a great height, subject to acceleration due to gravity ($mg$) and air resistance ($kv^2$). Newton's second law:
  \begin{equation*}
    m \cdot \frac{dv}{dt} = mg - kv^2;\ v(0) = 0
  \end{equation*}
  This ODE is autonomous since the RHS has no $t$ involved, so the equilibrium solution will be obtained by setting the RHS $= 0$.
  \begin{gather*}
    mg-kv^2 = 0 \\
    \vdots \\
    v = \pm \sqrt{\frac{mg}{k}}
  \end{gather*}

  Based on this model, the terminal velocity of the falling object will be:
  \begin{equation*}
    v = \sqrt{\frac{mg}{k}}
  \end{equation*}
  \hrule
  \vspace{12pt}
  This ODE can also be solved by finding the explicit solution, first by rearranging the equation:
  \begin{gather*}
    m \cdot \frac{dv}{dt} = mg - kv^2 \\
    \vdots \\
    \frac{m}{mg-kv^2}dv = dt
  \end{gather*}
  Then integrate both sides. Keep in mind that $g$, $k$, and $m$ are constants.
  \begin{gather*}
    \int_{}^{} \frac{m}{mg-kv^2} \,dv = \int_{}^{}  \,dt \\
    \frac{m}{k}\int_{}^{} \frac{m}{\frac{mg}{k}-v^2} \,dv = \int_{}^{}  \,dt \\
    \frac{m}{k}\int_{}^{} \frac{m}{\left(\sqrt{\frac{mg}{k}}-v\right)\left(\sqrt{\frac{mg}{k}}+v\right)} \,dv = \int_{}^{}  \,dt \\
  \end{gather*}
\end{example}

\newpage
\subsubsection{Phase Line}
\label{sssec:phaseLine}

\begin{wrapfigure}[]{r}{0.45\textwidth}
  \vspace{-20pt}
  \centering
  \includestandalone{figures/fig_009}
  \caption{Phase Line}
  \label{fig:009}
\end{wrapfigure}

When there are multiple solutions to a given ODE, each solution might have a different stability. Though this information could absolutely be expressed as seen in Figure \ref{fig:typesOfStability} with multiple {\color{gr} solutions} drawn, this information could also be visualized in a phase line.

Since these autonomous ODEs don't depend on an $x$, no information is gained by extending the graph into the $x$-axis. A phase line recognizes that by just expressing the stability of each solution along a single $y$-axis. This is seen in Figure \ref{fig:009}.

\end{document}
