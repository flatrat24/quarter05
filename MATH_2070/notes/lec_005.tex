\documentclass[12pt]{article}

\input{../../xlatex/imports/preamble}

\title{Lecture 005}
\date{February 06, 2025}

\begin{document}

\section{Non-Homogeneous ODE}
\label{sec:nonHomogeneousODE}

Non-Homogeneous ODE's are of the form:
\begin{equation*}
  y'' + p(t)y' + g(t)y = g(t)
\end{equation*}
These ODEs behave differently to previous ones. As such, the principle of superposition would now say:

\begin{definition}{Principle of Superposition (Non-Homogeneous)}
  If $Y_1$, $Y_2$ are two solutions of $y'' + p(t)y' + g(t)y = g(t)$, then $Y_1-Y_2$ is a solution of $y'' + p(t)y' + g(t)y = 0$.
\end{definition}

The proof of this is as follows:
\begin{align*}
  y_1'' + p(t)y_1' + g(t)y_1 &= g(t) \\
  y_2'' + p(t)y_2' + g(t)y_2 &= g(t)
\end{align*}
Then:
\begin{equation*}
  Y_1 - Y_2 = y_1''-y_2'' + p(t)\left(y_1'-y_2'\right) + g(t)\left(y_1-y_2\right) = g(t)-g(t) = 0
\end{equation*}
\vspace{12pt}
\hrule
\vspace{12pt}
The general solutions of $y'' + p(t)y' + g(t)y = g(t)$ is of the form:
\begin{align*}
  y &= y_c + Y \\
    &= C_1y_1 + C_2y_2 + Y
\end{align*}
where $y_c = C_1y_1 + C_2y_2$ is the general solution of the \textit{homogeneous} ODE $y'' + p(t)y' + g(t)y = 0$, and $Y$ is a part of the \textit{non-homogeneous} $y'' + p(t)y' + g(t)y = 0$.

The proof of this starts with verifying that $y_c + Y$ satisfies the ODE:
\begin{align*}
  \textup{LHS} &= y_c'' + Y'' + p(t)\left(y_c' + Y'\right) + g(t)\left(y_c + Y\right) \\
               &= y_c'' + p(t)y_c' + g(t)y_c + Y'' + p(t)Y' + g(t)Y \\
               &= g(t) = \textup{RHS}
\end{align*}
Since:
\begin{gather*}
  y_c'' + p(t)y_c' + g(t)y_c = 0 \\
  Y'' + p(t)Y' + g(t)Y = g(t)
\end{gather*}
Any solution of the non-homogeneous ODE is of the form $y_c+Y$ for some $C_1$, $C_2$.

Set $y$ to be any solution of $y'' + p(t)y' + g(t)y = g(t)$.

Principle $\Rightarrow$ $y-Y$ is a solutions of $y'' + p(t)y' + g(t)y = 0$ $\Rightarrow$ $y-Y = C_1y_1 + C_2y_2$ for some $C_1$, $C_2$.

\vspace{12pt}
\hrule
\vspace{12pt}

The most general way to obtain $Y$ is a \textbf{Variation of Parameters}.

Knowing that $y_c = C_1y_1 + C_2y_2$, then $C_1$, $C_2$ are varriated into function such that:
\begin{equation*}
  Y = u_1y_1 + u_2y_2
\end{equation*}
Then plugging $Y$ into the non-homogeneous ODE:
\[
  \begin{cases} 
    u_1'y_1 + u_2'y_2 = 0 \\
    u_1'y_1' + u_2'y_2' = g(t)
  \end{cases}
  \Rightarrow u_1' = \frac{-y_2 \cdot g}{W(y_1,y_2)}, u_1' = \frac{y_1 \cdot g}{W(y_1,y_2)} \\
\]
\begin{equation*}
  Y = y_1 \cdot \int_{}^{} \frac{-y_2 \cdot g}{W(y_1,y_2)} \,dt + y_2 \cdot \int_{}^{} \frac{y_1 \cdot g}{W(y_1,y_2)} \,dt
\end{equation*}
As can probably be seen, higher order generalization is very complicated. Thus introduces the \textbf{Method of Undetermined Coefficients}.
\subsection{Method of Undetermined Coefficients}
\label{ssec:methodOfUndeterminedCoefficients}
The method makes it easy ot generalize higher order ODEs, however it only works for:
\begin{equation*}
  ay'' + by' + c = g(t)
\end{equation*}
Where $g(t)$ is a product of exponential polynomials, $\sin$, and $\cos$ functions. However, these limitations aren't very relevant for engineering applications.

\begin{example}
  \begin{equation*}
    y'' - 2y' - 3y = 3
  \end{equation*}
  RHS is a constant function, so try a constant function.

  First, using the complementary equation, namely:
  \begin{equation*}
    y'' - 2y' - 3y = 0
  \end{equation*}
  The auxiliarry equation can be formulated as:
  \begin{equation*}
    r^2 - 2r - 3
  \end{equation*}
  Giving roots of $r_1=3$, $r_2=-1$, thus giving the general solution as:
  \begin{equation*}
    y_c = C_1e^{3t} + C_2e^{-t}
  \end{equation*}

  Set $Y=A$, and plug in the ODE:
  \begin{equation*}
    Y'' - 2Y' - 3Y = 0 - 0 - 3A = 3 \Rightarrow A = -1
  \end{equation*}
  Giving $Y=-1$ as a particular solution, so then the general solutions is:
  \begin{equation*}
    y = c_1e^{3t} + C_2e^{-t} - 1
  \end{equation*}
\end{example}
It should be noted that if the RHS is apolynomial of degree $n$, $Y$ should be set as a generic polynomial of degree $n$. Even when $g(t) = t$, $Y$ should still start from degree $n$, and work all the way down to the constant.
\begin{example}
  \begin{equation*}
    y'' - y' - 2y = t^2 + 1
  \end{equation*}
  Since the RHS is a polynomial function of degree $2$, so to solve, first try a polynomial function of degree $2$.

  By inspection, the complimentary solution of $y'' - y' - 2y = 0$ is:
  \begin{equation*}
    y_c = C_1e^{-t}+C_2e^{2t}
  \end{equation*}
  Set:
  \begin{align*}
    Y &= At^2 + Bt + C \\
    Y' &= 2At + B \\
    Y'' &= 2A \\
  \end{align*}
  Thus:
  \begin{align*}
    Y'' - Y' - 2Y &= 2A - (2At+B) - 2\left(at^2+Bt+c\right) \\
                  &= 2At^2 + (-2A-2B)t + 2A - B - 2C \\
                  &= t^2 + 1
  \end{align*}
  Thus, setting $t^2 + 1$ and $2At^2 + (-2A-2B)t + 2A - B - 2C$ equal to each other, it can be found that:
  \begin{gather*}
    -2A = 1 \\
    -2A-2B = 0 \\
    2A-B-2C = 1
  \end{gather*}
  Then solving for each value of $A$, $B$, and $C$:
  \begin{gather*}
    A = -\frac{1}{2} \\
    B = -A = \frac{1}{2} \\
    C = \frac{2A-B-1}{2} = -\frac{5}{4}
  \end{gather*}
  So:
  \begin{equation*}
    Y = -\frac{1}{2}t^2 + \frac{1}{2}t - \frac{5}{4}
  \end{equation*}
  Giving a general solution of:
  \begin{equation*}
    y = C_1e^{-t} + C_2e^{2t} - \frac{1}{2}t^2 + \frac{1}{2}t - \frac{5}{4}
  \end{equation*}
\end{example}

\begin{example}
  \begin{equation*}
    y'' - y' - 2y = t^3
  \end{equation*}
  \begin{align*}
    Y &= At^3 - Bt^2 + Ct + D \\
    Y' &= 3At^2 - 2Bt + C \\
    Y'' &= 6At^1 - 2B
  \end{align*}
  Thus:
  \begin{equation*}
    Y'' - Y' - 2Y = -2A^3 + \left(-3A-2B\right)t^2 + (6A-2B-2C)t + (2B-C-2D)
  \end{equation*}
  Comparing $Y = At^3 - Bt^2 + Ct + D$ with $-2A^3 + \left(-3A-2B\right)t^2 + (6A-2B-2C)t + (2B-C-2D)$, a system of equations can be found as:
  \begin{gather*}
    -2A = 1 \\
    -3A-2B = 0 \\
    6A - 2B - 2C = 0 \\
    2B - C - 2D = 0
  \end{gather*}
  Thus:
  \begin{align*}
    A &= - \frac{1}{2} \\
    B &= - \frac{3}{2}A = \frac{3}{4} \\
    C &= \frac{1}{2}(6A-2B) = -\frac{9}{4} \\
    D &= \frac{1}{2}(2B-C) = \frac{15}{8}
  \end{align*}
  Thus:
  \begin{equation*}
    Y = - \frac{1}{2}t^3 + \frac{3}{4}t^2 + -\frac{9}{4}t + \frac{15}{8}
  \end{equation*}
\end{example}

\end{document}
