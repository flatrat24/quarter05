\documentclass[12pt]{article}

%%%% GRAPHICS %%%%
\usepackage{tikz}
\usepackage[siunitx, american, RPvoltages]{circuitikz}
\usetikzlibrary{arrows.meta}
\usepackage{tikz-3dplot}
\usepackage{graphicx}
\usepackage{pgfplots}
  \pgfplotsset{compat=1.18}
\usetikzlibrary{arrows}
\newcommand{\midarrow}{\tikz \draw[-triangle 90] (0,0) -- +(.1,0);}

%%%% FIGURES %%%%
\usepackage{subcaption}
\usepackage{wrapfig}
\usepackage{float}
\usepackage[skip=5pt, font=footnotesize]{caption}

%%%% FORMATTING %%%%
\usepackage{parskip}
\usepackage{tcolorbox}
\usepackage{ulem}

%%%% TABLE FORMATTING %%%%
\usepackage{tabularray}
\UseTblrLibrary{booktabs}

%%%% MATH AND LOGIC %%%%
\usepackage{xifthen}
\usepackage{amsmath}
\usepackage{amssymb}
\usepackage{amsfonts}

%%%% TEXT AND SYMBOLS %%%%
\usepackage[T1]{fontenc}
\usepackage{textcomp}
\usepackage{gensymb}

%%%% OTHER %%%%
\usepackage{standalone}

%%%% LOGIC SYMBOLS %%%%
\newcommand*\xor{\oplus}

%%%% STYLES %%%%

% Packages
\usepackage[paper=letterpaper,tmargin=45pt,bmargin=45pt,lmargin=45pt,rmargin=45pt]{geometry}
\usepackage{titlesec}
\usepackage[rgb]{xcolor}
\selectcolormodel{natural}
\usepackage{ninecolors}
\selectcolormodel{rgb}

% Colors
\definecolor{pg}{HTML}{24273A}
\definecolor{fg}{HTML}{FFFFFF}
\definecolor{bg}{HTML}{24273A}
\definecolor{re}{HTML}{F38BA8}
\definecolor{gr}{HTML}{A6E3A1}
\definecolor{ye}{HTML}{F9E2AF}
\definecolor{or}{HTML}{FAB387}
\definecolor{bl}{HTML}{89B4FA}
\definecolor{ma}{HTML}{CBA6F7}
\definecolor{cy}{HTML}{94E2D5}
\definecolor{pi}{HTML}{F2CDCD}

\definecolor{copper}{HTML}{B87333}

\usepackage{nameref}
\makeatletter
\newcommand*{\currentname}{\@currentlabelname}
\makeatother

\titleformat{\section}
  {\normalfont\scshape\Large\bfseries}
  {\thesection}
  {0.75em}
  {}

\titleformat{\subsection}
  {\normalfont\scshape\large\bfseries}
  {\thesubsection}
  {0.75em}
  {}

\titleformat{\subsubsection}
  {\normalfont\scshape\normalsize\bfseries}
  {\thesubsubsection}
  {0.75em}
  {}

% Formula
\newcounter{formula}[section]
\newenvironment{formula}[1]{
  \stepcounter{formula}
  \begin{tcolorbox}[
    standard jigsaw, % Allows opacity
    colframe={fg},
    boxrule=1px,
    colback=bg,
    opacityback=0,
    sharp corners,
    sidebyside,
    righthand width=18px,
    coltext={fg}
  ]
  \centering
  \textbf{\uline{#1}}
}{
  \tcblower
  \textbf{\thesection.\theformula}
  \end{tcolorbox}
}

% Definition
\newcounter{definition}[section]

\newenvironment{definition*}[1]{
  \begin{tcolorbox}[
    standard jigsaw, % Allows opacity
    colframe={fg},
    boxrule=1px,
    colback=bg,
    opacityback=0,
    sharp corners,
    coltext={fg}
  ]
  \textbf{#1 \hfill}
  \vspace{5px}
  \hrule
  \vspace{5px}
  \noindent
}{
  \end{tcolorbox}
}

\newenvironment{definition}[1]{
  \stepcounter{definition}
  \begin{tcolorbox}[
    standard jigsaw, % Allows opacity
    colframe={fg},
    boxrule=1px,
    colback=bg,
    opacityback=0,
    sharp corners,
    coltext={fg}
  ]
  \textbf{#1 \hfill \thesection.\thedefinition}
  \vspace{5px}
  \hrule
  \vspace{5px}
  \noindent
}{
  \end{tcolorbox}
}

% Example Problem
\newcounter{example}[section]
\newenvironment{example}{
  \stepcounter{example}
  \begin{tcolorbox}[
    standard jigsaw, % Allows opacity
    colframe={fg},
    boxrule=1px,
    colback=bg,
    opacityback=0,
    sharp corners,
    coltext={fg}
  ]
  \textbf{Example \hfill \thesection.\theexample}
  \vspace{5px}
  \hrule
  \vspace{5px}
  \noindent
}{
  \end{tcolorbox}
}

\tikzset{
  cubeBorder/.style=fg,
  cubeFilling/.style={fg!20!bg, opacity=0.25},
  gridLine/.style={very thin, gray},
  graphLine/.style={-latex, thick, fg},
}

\pgfplotsset{
  basicAxis/.style={
    grid,
    major grid style={line width=.2pt,draw=fg!50!bg},
    axis lines = box,
    axis line style = {line width = 1px},
  }
}

%%%% REFERENCES %%%%
\usepackage{hyperref}
\hypersetup{
  colorlinks  = true,
  linkcolor   = bl,
  anchorcolor = bl,
  citecolor   = bl,
  filecolor   = bl,
  menucolor   = bl,
  runcolor    = bl,
  urlcolor    = bl,
}

\author{Ethan Anthony}
\newcommand*{\equal}{=}


\title{Lecture 005}
\date{February 06, 2025}

\begin{document}

\newpage
\section{Complex Numbers}
\label{sec:complexNumbers}

Complex numbers are of the form:
\begin{equation*}
  a + bi
\end{equation*}
where $a$ is the \textit{real} part of the number, and $b$ is the imaginary part. The multiplication of imaginary numbers is handled as:
\begin{gather*}
  (a+bi)(c+di) \\
  \Rightarrow \\
  ac + adi + cbi + bdi^2 \\
  \Rightarrow \\
  (ac-bd)+i(ad+cb)
\end{gather*}
Furthermore, $a+bi$ may be identified with $(a,b)$ in the complex plane where the real part ($a$) is represented by the $x$ coordinate and the imaginary part ($b$) by the $y$ coordinate. See Figure \ref{fig:010}.
\begin{figure}[H]
  \centering
  \includestandalone{figures/fig_010}
  \caption{Geometric Interpretation}
  \label{fig:010}
\end{figure}

\subsection{Euler's Formula}
\label{ssec:eulersFormula}

Euler's Formula provides a way to represent a complex number in terms of $\sin$ and $\cos$:
\begin{gather*}
  e^{i\theta} = \cos(\theta) + i\sin(\theta) \\
  \textup{or} \\
  re^{i\theta} = r\cos(\theta) + ri\sin(\theta),\ r \ge 0
\end{gather*}
As can be seen, Euler's Formula is simply just the polar coordinate transformation.
\begin{equation*}
  (r, \theta) \leftrightarrow (r \cos \theta, r \sin \theta)
\end{equation*}

\begin{wrapfigure}[]{r}{0.2\textwidth}
  \vspace{-20pt}
  \centering
  \includestandalone{figures/fig_011}
\end{wrapfigure}

Given a complex number:
\begin{equation*}
  a + bi
\end{equation*}
the amplitude ($r$) is the distance between $a+bi$ and the origin:
\begin{equation*}
  r = \sqrt{a^2 + b^2}
\end{equation*}
The angle $\theta$, called the phase, is the angle formed between the $x$-axis and the line connecting $a+bi$ to the origin.
\[
  \begin{cases} 
    \theta = \arctan(\frac{b}{a}),\ (a,b) \in \textup{ quadrant } I \textup{ or } IV \\
    \theta = \arctan(\frac{b}{a}) + \pi,\ (a,b) \in \textup{ quadrant } II \textup{ or } III
  \end{cases}
\]

\subsection{Multiplication of Complex Numbers}
\label{ssec:multiplicationOfComplexNumbers}

Consider two complex numbers:
\begin{equation*}
  r_1e^{i \theta_1} = r_1 \left( \cos \theta_1 + i\sin \theta_1 \right)\textup{,}\ r_2e^{i \theta_2} = r_2 \left( \cos \theta_2 + i\sin \theta_2 \right)
\end{equation*}
Multiplying these two complex numbers results in:
\begin{align*}
  r_1e^{i\theta_1}r_2e^{i\theta_2} &= r_1 \left( \cos \theta_1 + i\sin \theta_1 \right) \cdot r_2 \left( \cos \theta_2 + i\sin \theta_2 \right) \\
  r_1r_2e^{i\theta_1 + i\theta_2}  &= r_1r_2 \big[ \cos \theta_1\cos \theta_2 - \sin \theta_1\sin \theta_2 + i\left(\sin \theta_1\cos \theta_2 - \cos \theta_1\sin \theta_2 \right) \big] \\
  r_1r_2e^{i(\theta_1 + \theta_2)} &= r_1r_2 \big[ \cos(\theta_1 + \theta_2) + i\sin(\theta_1 + \theta_2)\big]
\end{align*}
By using the two following trig identities:
\begin{align*}
  \cos(\theta_1 + \theta_2) &= \cos \theta_1 \cos \theta_2 - \sin \theta_1 \sin \theta_2 \\
  \sin(\theta_1 + \theta_2) &= \sin \theta_1 \cos \theta_2 + \cos \theta_1 \sin \theta_2
\end{align*}
Geometrically, this means that the new amplitude is equal to the \textit{product} of the two original amplitudes. The new phase is the \textit{sum} of the two original phases.

\subsection{Powers of Complex Numbers}
\label{ssec:powersOfComplexNumbers}

To square $re^{i \theta} = r\cos(\theta) + ir\sin(\theta)$ using the Subsection \ref{ssec:multiplicationOfComplexNumbers}, it results in:
\begin{equation*}
  r^2e^{2i \theta} = r^2\left(\cos(2\theta) + i\sin(2\theta)\right)
\end{equation*}
Generally, for any integer $n$, the $n^{th}$ power of $re^{i\theta}$ is simply:
\begin{equation*}
  r^ne^{ni \theta} = r^n\left(\cos(n\theta) + i\sin(n\theta)\right)
\end{equation*}
Thus,
\begin{align*}
  (1+i)^4 &= \left(\sqrt{2}e^{i \frac{\pi}{4}}\right)^4 \\
          &= \left(\sqrt{2}\right)^4\left(e^{i \frac{\pi}{4}}\right)^4 \\
          &= 4e^{i\pi} \\
          &= -4
\end{align*}

\subsection{Roots of Complex Numbers}
\label{ssec:rootsOfComplexNumbers}

The roots are somewhat more complicated. Generally, for the $n^{th}$ root $\sqrt{z}$ has $n$ different candidates.

Let:
\begin{equation*}
  z=Re^{i \theta},\ w = \sqrt{z}
\end{equation*}
means that:
\begin{equation*}
 w^n = z
\end{equation*}

If
\begin{equation*}
  w = re^{i \alpha}
\end{equation*}
then,
\begin{equation*}
  r^ne^{in \alpha} = Re^{i \theta}
\end{equation*}
The amplitude $r$ is uniquely $R^{\frac{1}{n}}$, however, the phase $\alpha$ is \textit{not} unique. This is because $e^{in \alpha} = e^{i \theta}$ means that:
\begin{align*}
  n \alpha &= \theta + 2k\pi, k=0,\pm1,\pm2,\hdots \\
  \alpha   &= \frac{\theta + 2k\pi}{n}, k=0,\pm1,\pm2,\hdots
\end{align*}
\begin{example}{Translating a Complex Number into a Real Number}
  \begin{equation*}
    1^{\frac{1}{3}} = \left(e^{i \cdot 2k\pi}\right)^{\frac{1}{3}}  =e^{i \frac{2k\pi}{3}}
  \end{equation*}
  Depending on the value of $k$, there are \textit{three} different possibilities for $e$: \\
  \begin{align*}
    &\textup{If}\ k=0,3,6,\hdots \Rightarrow e^{i0}=1 \\
    &\textup{If}\ k=1,4,7,\hdots \Rightarrow e^{\frac{2\pi}{3}i}=-\frac{1}{2}+\frac{\sqrt{3}}{2}i \\
    &\textup{If}\ k=2,5,8,\hdots \Rightarrow e^{\frac{3\pi}{3}i}=-\frac{1}{2}-\frac{\sqrt{3}}{2}i \\
  \end{align*}
\end{example}

\end{document}
