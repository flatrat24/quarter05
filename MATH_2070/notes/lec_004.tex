\documentclass[12pt]{article}

\input{../../xlatex/imports/preamble}

\title{Lecture 004}
\date{January 30, 2025}

\begin{document}
\newpage
\section{Complex Numbers}
\label{sec:complexNumbers}

Complex numbers are of the form:
\begin{equation*}
  a + bi
\end{equation*}
where $a$ is the \textit{real} part of the number, and $b$ is the imaginary part.

\subsection{Multiplication}
\label{ssec:multiplication}
\begin{gather*}
  (a+bi)(c+di) \\
  \Rightarrow \\
  ac + adi + cbi + bdi^2 \\
  \Rightarrow \\
  (ac-bd)+i(ad+cb)
\end{gather*}

\subsection{Geometric Interpretation}
\label{ssec:geometricInterpretation}

$a+bi$ may be identified with $(a,b)$ in the complex plane where the real part ($a$) is represented by the $x$ coordinate and the imaginary part ($b$) by the $y$ coordinate.

\subsection{Euler's Formula}
\label{ssec:eulersFormula}

Euler's Formula provides a way to represent a complex number in terms of $\sin$ and $\cos$:
\begin{gather*}
  e^{i\theta} = \cos(\theta) + i\sin(\theta) \\
  \textup{or} \\
  re^{i\theta} = r\cos(\theta) + ri\sin(\theta),\ r \ge 0
\end{gather*}
As can be seen, Euler's Formula is simply just the polar coordinate transformation.

\subsection{Powers and Roots}
\label{ssec:powersAndRoots}

\subsubsection{Powers}
\label{sssec:powers}

To square $re^{i \theta} = r\cos(\theta) + ir\sin(\theta)$ using the previous description, it results in:
\begin{equation*}
  r^2e^{2i \theta} = r^2\left(\cos(2\theta) + i\sin(2\theta)\right)
\end{equation*}
Generally, for any integer $n$, the $n^{th}$ power of $re^{i\theta}$ is simply:
\begin{equation*}
  r^ne^{ni \theta} = r^n\left(\cos(n\theta) + i\sin(n\theta)\right)
\end{equation*}
Thus,
\begin{align*}
  (1+i)^4 &= \left(\sqrt{2}e^{i \frac{\pi}{4}}\right)^4 \\
          &= \left(\sqrt{2}\right)^4\left(e^{i \frac{\pi}{4}}\right)^4 \\
          &= 4e^{i\pi} \\
          &= -4
\end{align*}

\subsubsection{Roots}
\label{sssec:roots}

The roots are somewhat more complicated. Generally, for the $n^{th}$ root $\sqrt{z^{\frac{1}{n}}}$ has $n$ different candidates.

Let:
\begin{equation*}
  z=Re^{i \theta},\ w = \sqrt{z^{\frac{1}{n}}}
\end{equation*}
means that:
\begin{equation*}
 w^n = z
\end{equation*}

If
\begin{equation*}
  w = re^{i \alpha}
\end{equation*}
then,
\begin{equation*}
  r^ne^{in \alpha} = Re^{i \theta}
\end{equation*}
The amplitude $r$ is uniquely $R^{\frac{1}{n}}$, however, the phase $\alpha$ is \textit{not} unique. This is because $e^{in \alpha} = e^{i \theta}$ means that:
\begin{align*}
  n \alpha &= \theta + 2k\pi, k=0,\pm1,\pm2,\hdots \\
  \alpha   &= \frac{\theta + 2k\pi}{n}, k=0,\pm1,\pm2,\hdots
\end{align*}
\begin{example}
  \begin{equation*}
    1^{\frac{1}{3}} = \left(e^{i \cdot 2k\pi}\right)^{\frac{1}{3}}  =e^{i \frac{2k\pi}{3}}
  \end{equation*}
  Depending on the value of $k$, there are \textit{three} different possibilites for $e$: \\
  \begin{align*}
    &\textup{If}\ k=0,3,6,\hdots \Rightarrow e^{i0}=1 \\
    &\textup{If}\ k=1,4,7,\hdots \Rightarrow e^{\frac{2\pi}{3}i}=-\frac{1}{2}+\frac{\sqrt{3}}{2}i \\
    &\textup{If}\ k=2,5,8,\hdots \Rightarrow e^{\frac{3\pi}{3}i}=-\frac{1}{2}-\frac{\sqrt{3}}{2}i \\
  \end{align*}
\end{example}

\end{document}
