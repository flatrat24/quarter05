\documentclass[12pt]{article}

%%%% GRAPHICS %%%%
\usepackage{tikz}
\usepackage[siunitx, american, RPvoltages]{circuitikz}
\usetikzlibrary{arrows.meta}
\usepackage{tikz-3dplot}
\usepackage{graphicx}
\usepackage{pgfplots}
  \pgfplotsset{compat=1.18}
\usetikzlibrary{arrows}
\newcommand{\midarrow}{\tikz \draw[-triangle 90] (0,0) -- +(.1,0);}

%%%% FIGURES %%%%
\usepackage{subcaption}
\usepackage{wrapfig}
\usepackage{float}
\usepackage[skip=5pt, font=footnotesize]{caption}

%%%% FORMATTING %%%%
\usepackage{parskip}
\usepackage{tcolorbox}
\usepackage{ulem}

%%%% TABLE FORMATTING %%%%
\usepackage{tabularray}
\UseTblrLibrary{booktabs}

%%%% MATH AND LOGIC %%%%
\usepackage{xifthen}
\usepackage{amsmath}
\usepackage{amssymb}
\usepackage{amsfonts}

%%%% TEXT AND SYMBOLS %%%%
\usepackage[T1]{fontenc}
\usepackage{textcomp}
\usepackage{gensymb}

%%%% OTHER %%%%
\usepackage{standalone}

%%%% LOGIC SYMBOLS %%%%
\newcommand*\xor{\oplus}

%%%% STYLES %%%%

% Packages
\usepackage[paper=letterpaper,tmargin=45pt,bmargin=45pt,lmargin=45pt,rmargin=45pt]{geometry}
\usepackage{titlesec}
\usepackage[rgb]{xcolor}
\selectcolormodel{natural}
\usepackage{ninecolors}
\selectcolormodel{rgb}

% Colors
\definecolor{pg}{HTML}{24273A}
\definecolor{fg}{HTML}{FFFFFF}
\definecolor{bg}{HTML}{24273A}
\definecolor{re}{HTML}{F38BA8}
\definecolor{gr}{HTML}{A6E3A1}
\definecolor{ye}{HTML}{F9E2AF}
\definecolor{or}{HTML}{FAB387}
\definecolor{bl}{HTML}{89B4FA}
\definecolor{ma}{HTML}{CBA6F7}
\definecolor{cy}{HTML}{94E2D5}
\definecolor{pi}{HTML}{F2CDCD}

\definecolor{copper}{HTML}{B87333}

\usepackage{nameref}
\makeatletter
\newcommand*{\currentname}{\@currentlabelname}
\makeatother

\titleformat{\section}
  {\normalfont\scshape\Large\bfseries}
  {\thesection}
  {0.75em}
  {}

\titleformat{\subsection}
  {\normalfont\scshape\large\bfseries}
  {\thesubsection}
  {0.75em}
  {}

\titleformat{\subsubsection}
  {\normalfont\scshape\normalsize\bfseries}
  {\thesubsubsection}
  {0.75em}
  {}

% Formula
\newcounter{formula}[section]
\newenvironment{formula}[1]{
  \stepcounter{formula}
  \begin{tcolorbox}[
    standard jigsaw, % Allows opacity
    colframe={fg},
    boxrule=1px,
    colback=bg,
    opacityback=0,
    sharp corners,
    sidebyside,
    righthand width=18px,
    coltext={fg}
  ]
  \centering
  \textbf{\uline{#1}}
}{
  \tcblower
  \textbf{\thesection.\theformula}
  \end{tcolorbox}
}

% Definition
\newcounter{definition}[section]

\newenvironment{definition*}[1]{
  \begin{tcolorbox}[
    standard jigsaw, % Allows opacity
    colframe={fg},
    boxrule=1px,
    colback=bg,
    opacityback=0,
    sharp corners,
    coltext={fg}
  ]
  \textbf{#1 \hfill}
  \vspace{5px}
  \hrule
  \vspace{5px}
  \noindent
}{
  \end{tcolorbox}
}

\newenvironment{definition}[1]{
  \stepcounter{definition}
  \begin{tcolorbox}[
    standard jigsaw, % Allows opacity
    colframe={fg},
    boxrule=1px,
    colback=bg,
    opacityback=0,
    sharp corners,
    coltext={fg}
  ]
  \textbf{#1 \hfill \thesection.\thedefinition}
  \vspace{5px}
  \hrule
  \vspace{5px}
  \noindent
}{
  \end{tcolorbox}
}

% Example Problem
\newcounter{example}[section]
\newenvironment{example}{
  \stepcounter{example}
  \begin{tcolorbox}[
    standard jigsaw, % Allows opacity
    colframe={fg},
    boxrule=1px,
    colback=bg,
    opacityback=0,
    sharp corners,
    coltext={fg}
  ]
  \textbf{Example \hfill \thesection.\theexample}
  \vspace{5px}
  \hrule
  \vspace{5px}
  \noindent
}{
  \end{tcolorbox}
}

\tikzset{
  cubeBorder/.style=fg,
  cubeFilling/.style={fg!20!bg, opacity=0.25},
  gridLine/.style={very thin, gray},
  graphLine/.style={-latex, thick, fg},
}

\pgfplotsset{
  basicAxis/.style={
    grid,
    major grid style={line width=.2pt,draw=fg!50!bg},
    axis lines = box,
    axis line style = {line width = 1px},
  }
}

%%%% REFERENCES %%%%
\usepackage{hyperref}
\hypersetup{
  colorlinks  = true,
  linkcolor   = bl,
  anchorcolor = bl,
  citecolor   = bl,
  filecolor   = bl,
  menucolor   = bl,
  runcolor    = bl,
  urlcolor    = bl,
}

\author{Ethan Anthony}
\newcommand*{\equal}{=}


\title{Lecture 002}
\date{January 08, 2025}

\begin{document}

\newpage
\section{Design Considerations}
\label{sec:designConsiderations}

Determining the stresses in a body, in and of itself, serves no purpose. However, these methods of analysis can be used to inform design decisions and create sturdy structures and machines. An important family of information to an engineer is how a certain material will behave under various kinds of loads.

\subsection{Determining the Strength of Materials}
\label{ssec:determiningTheStrengthOfMaterials}

Properties, such as the \textbf{ultimate load}, of materials can be empirically tested by pushing that material to its limits and taking measurements. To find the ultimate load, a member is put under tensile stress until it cannot go any further. 

\begin{definition}{Ultimate Load}
  The largest axial force that may be applied to a material before it breaks, deforms, or begins to carry less load. Ultimate load is denoted by $P_U$.
\end{definition}

Since axial stress is uniformly distributed, the ultimate load divided by the cross-sectional area of the member gives the \textbf{ultimate normal stress} of the material.

\begin{formula}{Ultimate Normal Stress}
  \begin{equation*}
    \sigma_U = \frac{P_U}{A}
  \end{equation*}
  Also called the \textit{Ultimate Strength in Tension}, this measures the maximum axial stress a material can undergo before failing.
\end{formula}

Other tests can be performed to determine the \textbf{ultimate shearing stress} of a material. The same idea as prior is applied, however this time the load is applied is shear rather than axially.

\begin{formula}{Ultimate Shearing Stress}
  \begin{equation*}
    \tau_U = \frac{P_U}{A}
  \end{equation*}
  Also called the \textit{Ultimate Strength in Shear}, this measures the maximum shear stress a material can undergo before failing.
\end{formula}

\subsection{Factor of Safety}
\label{ssec:factorOfSafety}

Though a material can theoretically sustain stresses up to its ultimate load, in practice this is not the case. Since loads are unpredictable and vary, the maximum load a component is designed to sustain should only be a percentage of its ultimate load. This concept is reflected in the idea of a \textbf{factor of safety}.

\begin{formula}{Factor of Safety}
  \begin{equation*}
    \textup{Factor of Safety} = \textup{F.S.} = \frac{\textup{ultimate load}}{\textup{allowable load}} = \frac{\textup{ultimate stress}}{\textup{allowable stress}}
  \end{equation*}
  The factor of safety is calculated by finding the ratio between the ultimate load/stress and the allowable load/stress.
\end{formula}

The relationship between the F.S. as calculated by loads vs. stresses only holds up when a linear relationship exists between the load and the stress. However, most materials experience non-linear a relationship between these two values as the material approaches its ultimate load.

The F.S. to use in a design is a subjective choice, but is based on many factors such as:
\begin{enumerate}
  \itemsep-0.2em
  \item Variations that may occur in the properties of the member
  \item The number of loadings expected during the life of the structure or machine
  \item The type of loadings planned for in the design or that may occur in the future
  \item The type of failure (brittle and ductile materials will fail differently)
  \item Uncertainty due to methods of analysis
  \item Deterioration that may occur in the future because of poor maintenance or unpreventable natural causes
  \item The importance of a given member in the integrity of the entire structure
\end{enumerate}

\end{document}
