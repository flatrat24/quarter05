\documentclass[12pt]{article}

%%%% GRAPHICS %%%%
\usepackage{tikz}
\usepackage[siunitx, american, RPvoltages]{circuitikz}
\usetikzlibrary{arrows.meta}
\usepackage{tikz-3dplot}
\usepackage{graphicx}
\usepackage{pgfplots}
  \pgfplotsset{compat=1.18}
\usetikzlibrary{arrows}
\newcommand{\midarrow}{\tikz \draw[-triangle 90] (0,0) -- +(.1,0);}

%%%% FIGURES %%%%
\usepackage{subcaption}
\usepackage{wrapfig}
\usepackage{float}
\usepackage[skip=5pt, font=footnotesize]{caption}

%%%% FORMATTING %%%%
\usepackage{parskip}
\usepackage{tcolorbox}
\usepackage{ulem}

%%%% TABLE FORMATTING %%%%
\usepackage{tabularray}
\UseTblrLibrary{booktabs}

%%%% MATH AND LOGIC %%%%
\usepackage{xifthen}
\usepackage{amsmath}
\usepackage{amssymb}
\usepackage{amsfonts}

%%%% TEXT AND SYMBOLS %%%%
\usepackage[T1]{fontenc}
\usepackage{textcomp}
\usepackage{gensymb}

%%%% OTHER %%%%
\usepackage{standalone}

%%%% LOGIC SYMBOLS %%%%
\newcommand*\xor{\oplus}

%%%% STYLES %%%%

% Packages
\usepackage[paper=letterpaper,tmargin=45pt,bmargin=45pt,lmargin=45pt,rmargin=45pt]{geometry}
\usepackage{titlesec}
\usepackage[rgb]{xcolor}
\selectcolormodel{natural}
\usepackage{ninecolors}
\selectcolormodel{rgb}

% Colors
\definecolor{pg}{HTML}{24273A}
\definecolor{fg}{HTML}{FFFFFF}
\definecolor{bg}{HTML}{24273A}
\definecolor{re}{HTML}{F38BA8}
\definecolor{gr}{HTML}{A6E3A1}
\definecolor{ye}{HTML}{F9E2AF}
\definecolor{or}{HTML}{FAB387}
\definecolor{bl}{HTML}{89B4FA}
\definecolor{ma}{HTML}{CBA6F7}
\definecolor{cy}{HTML}{94E2D5}
\definecolor{pi}{HTML}{F2CDCD}

\definecolor{copper}{HTML}{B87333}

\usepackage{nameref}
\makeatletter
\newcommand*{\currentname}{\@currentlabelname}
\makeatother

\titleformat{\section}
  {\normalfont\scshape\Large\bfseries}
  {\thesection}
  {0.75em}
  {}

\titleformat{\subsection}
  {\normalfont\scshape\large\bfseries}
  {\thesubsection}
  {0.75em}
  {}

\titleformat{\subsubsection}
  {\normalfont\scshape\normalsize\bfseries}
  {\thesubsubsection}
  {0.75em}
  {}

% Formula
\newcounter{formula}[section]
\newenvironment{formula}[1]{
  \stepcounter{formula}
  \begin{tcolorbox}[
    standard jigsaw, % Allows opacity
    colframe={fg},
    boxrule=1px,
    colback=bg,
    opacityback=0,
    sharp corners,
    sidebyside,
    righthand width=18px,
    coltext={fg}
  ]
  \centering
  \textbf{\uline{#1}}
}{
  \tcblower
  \textbf{\thesection.\theformula}
  \end{tcolorbox}
}

% Definition
\newcounter{definition}[section]

\newenvironment{definition*}[1]{
  \begin{tcolorbox}[
    standard jigsaw, % Allows opacity
    colframe={fg},
    boxrule=1px,
    colback=bg,
    opacityback=0,
    sharp corners,
    coltext={fg}
  ]
  \textbf{#1 \hfill}
  \vspace{5px}
  \hrule
  \vspace{5px}
  \noindent
}{
  \end{tcolorbox}
}

\newenvironment{definition}[1]{
  \stepcounter{definition}
  \begin{tcolorbox}[
    standard jigsaw, % Allows opacity
    colframe={fg},
    boxrule=1px,
    colback=bg,
    opacityback=0,
    sharp corners,
    coltext={fg}
  ]
  \textbf{#1 \hfill \thesection.\thedefinition}
  \vspace{5px}
  \hrule
  \vspace{5px}
  \noindent
}{
  \end{tcolorbox}
}

% Example Problem
\newcounter{example}[section]
\newenvironment{example}{
  \stepcounter{example}
  \begin{tcolorbox}[
    standard jigsaw, % Allows opacity
    colframe={fg},
    boxrule=1px,
    colback=bg,
    opacityback=0,
    sharp corners,
    coltext={fg}
  ]
  \textbf{Example \hfill \thesection.\theexample}
  \vspace{5px}
  \hrule
  \vspace{5px}
  \noindent
}{
  \end{tcolorbox}
}

\tikzset{
  cubeBorder/.style=fg,
  cubeFilling/.style={fg!20!bg, opacity=0.25},
  gridLine/.style={very thin, gray},
  graphLine/.style={-latex, thick, fg},
}

\pgfplotsset{
  basicAxis/.style={
    grid,
    major grid style={line width=.2pt,draw=fg!50!bg},
    axis lines = box,
    axis line style = {line width = 1px},
  }
}

%%%% REFERENCES %%%%
\usepackage{hyperref}
\hypersetup{
  colorlinks  = true,
  linkcolor   = bl,
  anchorcolor = bl,
  citecolor   = bl,
  filecolor   = bl,
  menucolor   = bl,
  runcolor    = bl,
  urlcolor    = bl,
}

\author{Ethan Anthony}
\newcommand*{\equal}{=}


\title{Lecture 003}
\date{January 24, 2025}

\begin{document}
\newpage

\section{Torsion}
\label{sec:torsion}

\subsection{Introducing New Values}
\label{ssec:introducingNewValues}

Before covering concepts, there are several new values, each with its own notation. Referring to Figure \ref{fig:circularShaftInTorsion}, all these new values are labeled, and are as follows:

\begin{center}
  \begin{tblr}{ccc}
    \toprule
    Value & Symbol & Axial Counterpart \\
    \midrule
    Torque               & $T$    & Force ($P$) \\
    Distance From Center & $\rho$ & Force ($P$) \\
    Radius               & $C$    & Force ($P$) \\
    Angle of Twist       & $\phi$ & Force ($P$) \\
    Length of Rotation   & $a$    & Force ($P$) \\
    \bottomrule
  \end{tblr}
\end{center}

\begin{figure}[H]
  \begin{subfigure}[H]{0.45\textwidth}
    \centering
    \includestandalone{figures/fig_033}
    \caption{Side Profile}
    \label{fig:033}
  \end{subfigure}
  \begin{subfigure}[H]{0.45\textwidth}
    \centering
    \includestandalone{figures/fig_034}
    \caption{Front Profile}
    \label{fig:034}
  \end{subfigure}
  \centering
  \caption{Circular Shaft in Torsion}
  \label{fig:circularShaftInTorsion}
\end{figure}

Another important value is the angle formed by the line connecting two points some distance $\rho$ from the center in each face of the shaft. This is the \textbf{shear strain} ($\gamma$) experienced by the body. In Figure \ref{fig:037}, this angle can be seen between the points $n$, $m$, and $m'$.

\begin{figure}[H]
  \centering
  \includestandalone{figures/fig_037}
  \caption{Shear Strain}
  \label{fig:037}
\end{figure}

\subsection{Overview of Torsion}
\label{ssec:overviewOfTorsion}

\subsubsection{Torsion}
\label{sssec:torsion}

\begin{definition}{Torsion}
  The twisting of an object caused by a moment acting about the object's longitudinal (long) axis.
\end{definition}

The "force" that creates the moment around the axis is referred to as torque ($\tau$). When a shaft, such as the one if Figure \ref{fig:circularShaftInTorsion} is put into torsion, each cross section to rotate relative to the others. However, the face of each cross section \textit{remains plane}, which means that the cross sections remain flat and undistorted.

\begin{figure}[H]
  \centering
  \begin{subfigure}[H]{0.45\textwidth}
    \centering
    \includestandalone{figures/fig_035}
    \caption{Before Torsion}
    \label{fig:035}
  \end{subfigure}
  \begin{subfigure}[H]{0.45\textwidth}
    \centering
    \includestandalone{figures/fig_036}
    \caption{After Torsion}
    \label{fig:036}
  \end{subfigure}
  \caption{Shaft in Torsion}
  \label{fig:shaftInTorsion}
  \vspace{-22pt}
\end{figure}

\subsubsection{Angle of Twist}
\label{sssec:angleOfTwist}

In Figure \ref{fig:angleOfTwist}, the angle of twist ($\phi$) is shown. The angle of twist refers to the angle between an initial point $m$, the center of the face, and the distorted point $m'$. The angle of twist can be calculated as:

\begin{formula}{Angle of Twist}
  \begin{equation*}
    \phi = \frac{TL}{GJ}\ \ \ \textup{or}\ \ \ \int_{}^{} d\phi = \int_{}^{} \frac{T(x)}{J(x) \cdot G(x)} \,dx
  \end{equation*}
  \begin{tblr}{cc}
    \midrule
    $T$: Torque Applied & $L$: Length \\
    \midrule
    $G$: Modulus of Rigidity & $J$: Polar Moment of Inertia \\
    \midrule
  \end{tblr}
\end{formula}

\begin{figure}[H]
  \vspace{-10pt}
  \begin{subfigure}[H]{0.45\textwidth}
    \centering
    \includestandalone{figures/fig_038}
    \caption{Side Profile}
    \label{fig:038}
  \end{subfigure}
  \begin{subfigure}[H]{0.45\textwidth}
    \centering
    \includestandalone{figures/fig_039}
    \caption{Front Profile}
    \label{fig:039}
  \end{subfigure}
  \centering
  \caption{Angle of Twist}
  \label{fig:angleOfTwist}
  \vspace{-15pt}
\end{figure}

\subsubsection{Polar Moment of Inertia}
\label{sssec:polarMomentOfInertia}

\begin{wrapfigure}[4]{r}{0.4\textwidth}
  \centering
  \vspace{-30pt}
  \includestandalone{figures/fig_040}
  \caption{Geometry of a Cylinder}
  \label{fig:040}
\end{wrapfigure}

The polar moment of inertia of an object is a property of an object that described its resistance to torsion. This is a value based on the geometry of the body.

\begin{formula}{Polar Moment of Inertia}
  \begin{equation*}
    J = \frac{\pi}{2}\left(r^4_o-r^4_i\right)
  \end{equation*}
\end{formula}

\subsubsection{Shear Strain}
\label{sssec:shearStrain}

\begin{figure}[H]
  \centering
  \includestandalone{figures/fig_041}
  \caption{Front Profile}
  \label{fig:041}
\end{figure}

When an object undergoes shear strain one can imagine a grid across the surface of the cylinder. Each square in the grid will undergo a deformation. However, as has been said in Subsubsection \ref{sssec:torsion}, the circular shaft maintains it's cross-section faces' geometry intact. So, each square in the grid will have to maintain the same side lengths:

\begin{figure}[H]
  \centering
  \includestandalone{figures/fig_042}
\end{figure}

The strain caused by this can be calculated in terms of the angle of twist (\ref{sssec:angleOfTwist}) and either the outer radius of the bar ($C$) or some inner radius ($\rho$):
\begin{formula}{Shear Strain}
  \begin{equation*}
    \gamma = \frac{C \phi}{L} \ \ \ \textup{or}\ \ \ \gamma = \frac{\rho \phi}{L}
  \end{equation*}
\end{formula}

\newpage
\subsubsection{Shear Stress}
\label{sssec:shearStress}

\begin{wrapfigure}[2]{r}{0.45\textwidth}
  \centering
  \includestandalone{figures/fig_045}
  \caption{Calculating Shear Stress}
  \label{fig:045}
\end{wrapfigure}

To calculate the shear strain, consider Figure \ref{fig:045}.

\begin{formula}{Shear Stress}
  \begin{align*}
    T    &= \int_{0}^{r} \tau \rho \,dA \\
         &= \frac{\tau}{\rho}\int_{0}^{r} \rho^2 \,dA \\
         &= \frac{\tau}{\rho} \left(\frac{\pi}{2}\left(r_0^4-r_i^4\right)\right) \\
         &= \frac{\tau}{\rho} J
  \end{align*}
  \begin{equation*}
    \tau = \frac{T\rho}{J}\ \ \ \textup{or}\ \ \ \tau = G \cdot \gamma
  \end{equation*}
\end{formula}

\begin{figure}[H]
  \centering
  \begin{subfigure}[H]{0.45\textwidth}
    \centering
    \includestandalone{figures/fig_043}
    \caption{Distribution of Shear Stress}
    \label{fig:043}
  \end{subfigure}
  \begin{subfigure}[H]{0.45\textwidth}
    \centering
    \includestandalone{figures/fig_044}
    \caption{On a Hollow Object}
    \label{fig:044}
  \end{subfigure}
\end{figure}

The Shear stress is down the radius of the face of the body linearly. In Figure \ref{fig:044}, a hollow object is experiencing shear strain. Since it is hollow, it is more efficient at holding the shear strain.

\subsection{Multiple Torsions}
\label{ssec:multipleTorsions}

\begin{wrapfigure}[9]{l}{0.45\textwidth}
  \vspace{-10pt}
  \centering
  \includestandalone{figures/fig_046}
  \caption{Multiple Torques}
  \label{fig:046}
\end{wrapfigure}

Consider the member in Figure \ref{fig:046} experiencing multiple torques simultaneously. Very similarly to previous techniques used in statics to find internal forces. The shear stress and strain can be found in any region bounded by two torques (on the ends, there would be no stress or strain). Supposing that:
\begin{equation*}
  T_b > T_a > T_c
\end{equation*}
Then it could be found that region with the greatest shear and stress would be between $T_b$ and $T_a$.

A diagram displaying the internal torque as a function of $x$ would look like that in Figure \ref{fig:047}.

\begin{figure}[H]
  \centering
  \includestandalone{figures/fig_047}
  \caption{Internal Torque Diagram}
  \label{fig:047}
\end{figure}

\subsection{Brittle vs. Ductile}
\label{ssec:brittleVsDuctile}

\begin{figure}[H]
  \centering
  \begin{subfigure}[H]{0.45\textwidth}
    \centering
    \includestandalone{figures/fig_048}
    \caption{Brittle}
    \label{fig:048}
  \end{subfigure}
  \begin{subfigure}[H]{0.45\textwidth}
    \centering
    \includestandalone{figures/fig_049}
    \caption{Ductile}
    \label{fig:049}
  \end{subfigure}
  \caption{Brittle vs. Ductile}
  \label{fig:brittleVsDuctile}
\end{figure}

The brittle material in Figure \ref{fig:048} will fail at a $45\degree$ angle from the longitudinal axis. This is due to the brittle material tend to fail in tension, thus failing along the place of maximum tension. Conversely, the material in Figure \ref{fig:049} is ductile and will then fail along the plane of maximum shear.

\subsection{Torsional Stiffness}
\label{ssec:torsionalStiffness}

Similar to how an axial member will have a stiffness that helps in solving statically indeterminate problems, so too do members have torsional stiffnesses ($k$).

\begin{formula}{Torsional Stiffness}
  \begin{equation*}
    k = \frac{T}{\phi}
  \end{equation*}
\end{formula}

\subsection{Statically Indeterminate Torsion}
\label{ssec:staticallyIndeterminateTorsion}

\begin{wrapfigure}[7]{r}{0.45\textwidth}
  \vspace{-30pt}
  \centering
  \includestandalone{figures/fig_050}
  \caption{Statically Indeterminate Torsion}
  \label{fig:050}
\end{wrapfigure}

Just as there were situations in Subsection \ref{ssec:staticIndeterminacy} where the internal forces, and in turn the stresses and shears, couldn't be solved through statics alone, there are also situations in which the internal torsion cannot be solved through statics alone. In such cases, it is necessary to analyze the geometry of the problem to solve for internal torsion.

Consider Figure \ref{fig:050}. There is a beam fixed on each end to two supports. Furthermore, there is a torque applied $\frac{1}{3}$ of the way down the beam such that $L_{ab} = \frac{1}{2}L_{bc}$.

\begin{wrapfigure}[7]{l}{0.45\textwidth}
  \centering
  \includestandalone{figures/fig_051}
  \caption{Free Body Diagram}
  \label{fig:051}
\end{wrapfigure}

The member in Figure \ref{fig:050} can be translated into a free body diagram as shown in Figure \ref{fig:051} where each component of the beam is considered individually. However, this would result in an equation with two unknowns, thus showing this situation as statically indeterminate:
\begin{equation*}
  T_a + T_b = T
\end{equation*} \\
So, how should a statically indeterminate torsion problem be approached? First, it must be noticed that, since both ends are fixed in place, then the total angle of twist ($\phi_{t}$) should be zero.
\begin{equation*}
  \phi_{t} = 0
\end{equation*}
An analogous process to Subsection \ref{ssec:staticIndeterminacy} can be applied in which the shear of each component can be calculated individually to find the total angle of twist.
\begin{equation*}
  \phi_{t} = \sum_{n=1}^{N} \phi_{n}
\end{equation*}
Thus:
\begin{align*}
  \phi_{t} = 0                       &= \phi_{ab} + \phi_{bc} \\
                                     &= \frac{T_{ab}L_{ab}}{G_{ab}J_{ab}} + \frac{T_{bc}L_{bc}}{G_{bc}J_{bc}} \\ 
  -\frac{T_{ab}L_{ab}}{G_{ab}J_{ab}} &= \frac{T_{bc}L_{bc}}{G_{bc}J_{bc}} \\
\end{align*}
From here, different situations will give different unknowns and thus different values to solve for. In the current example, the only difference between the two sections is their lengths, and so:
\begin{align*}
  -\frac{T_{ab}L_{ab}}{GJ} &= \frac{T_{bc}L_{bc}}{GJ} \\
  -T_{ab}L_{ab} &= T_{bc}L_{bc} \\
  -\frac{L_{ab}}{L_{bc}} \cdot T_{ab} &= T_{bc}
\end{align*}
It can then be seen that the key to solving these statically indeterminate situations is to finding the ratio between the two torques based on the summation of the angles of rotation.

Consider the shaft in Figure \ref{fig:compositeStaticIndeterminateTorsion}. In this situation, the shaft is experiencing two equal and opposite torques ($T$). Additionally, the shaft is a composite of a two materials with different rigidities ($G$). How could the torque in each material be found?

\begin{figure}[H]
  \begin{subfigure}[H]{0.55\textwidth}
    \centering
    \includestandalone{figures/fig_052}
    \caption{Side Profile}
    \label{fig:052}
  \end{subfigure}
  \begin{subfigure}[H]{0.35\textwidth}
    \centering
    \includestandalone{figures/fig_053}
    \caption{Front Profile}
    \label{fig:053}
  \end{subfigure}
  \caption{Statically Indeterminate Composite Shaft}
  \label{fig:compositeStaticIndeterminateTorsion}
\end{figure}

In this case, the total angle of twist of the shaft \textit{won't} be zero. However, since the shaft is a composite, it is known that the angle of twist in each material must be the same.
\begin{equation*}
  \phi_i = \phi_o \Rightarrow \frac{T_iL}{G_iJ_i} = \frac{T_oL}{G_oJ_o} \Rightarrow \frac{T_i}{G_iJ_i} = \frac{T_o}{G_oJ_o}
\end{equation*}
From there, there exist two unknown values: $T_i$ and $T_o$. It may look impossible to solve, however, it's important to remember that the \textit{total} torque ($T$) is known. Thus, there are two equations and two unknowns:
\begin{equation*}
  \frac{T_i}{G_iJ_i} = \frac{T_o}{G_oJ_o}\ \ \ \textup{and}\ \ \ T_i + T_o = T
\end{equation*}

\subsection{Power and Rotation Speed}
\label{ssec:powerAndRotationSpeed}

For shafts that are designed to be rotated, there is obviously going to be a torque of some kind applied to the shaft to achieve that rotation. When designing these parts, it's important that the torque applied doesn't create shear beyond what is deemed acceptable in the specific context.

\begin{formula}{Power}
  \begin{equation*}
    P = T \omega = T(2 \pi f)
  \end{equation*}
\end{formula}

To determine the amount of torque generated by the power ($P$) applied at a certain frequency ($f$) to a shaft:
\begin{formula}{Torque Generated by Power and Frequency}
  \begin{equation*}
    T = \frac{P}{\omega} = \frac{P}{2\pi \cdot f}
  \end{equation*}
\end{formula}
Once this torque has been calculated, the engineer can use other methods and formulas from Section \ref{sec:torsion} to calculate the allowable power, frequency, etc., for the specific situation.

\subsection{Stress Concentrations in Circular Shafts}
\label{ssec:stressConcentrationsInCircularShafts}

Discontinuities in a circular shaft will cause different stress concentrations throughout that shaft. Fillets are often used to reduce drastic concentrations, so only filleted shafts will be considered in this section.
\begin{figure}[H]
  \centering
  \includestandalone{figures/fig_056}
  \caption{Filleted Shaft}
  \label{fig:056}
\end{figure}
The geometry of the filleted shaft will uniquely determine the maximum value of shear stress in terms of the average shear stress in the shaft. By analyzing the geometry, the \textbf{stress concentration factor} ($k$) can be calculated, thus giving the value of the maximum stress:
\begin{equation*}
  \tau_{max} = k \frac{T \rho}{J}
\end{equation*}
where $\frac{T \rho}{J}$ is calculated using the \textit{smaller} portion of the shaft.

The same as in Subsubsection \ref{sssec:maxStressForAFillet}, $k$ is calculated using the both the ratio of the larger to smaller diameter as well as the ratio between the radius of the fillet and the smaller diameter:
\begin{equation*}
  \frac{D}{d}\ \ \ \textup{and}\ \ \ \frac{r}{h}
\end{equation*}

\end{document}
