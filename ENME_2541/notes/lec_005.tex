\documentclass[12pt]{article}

%%%% GRAPHICS %%%%
\usepackage{tikz}
\usepackage[siunitx, american, RPvoltages]{circuitikz}
\usetikzlibrary{arrows.meta}
\usepackage{tikz-3dplot}
\usepackage{graphicx}
\usepackage{pgfplots}
  \pgfplotsset{compat=1.18}
\usetikzlibrary{arrows}
\newcommand{\midarrow}{\tikz \draw[-triangle 90] (0,0) -- +(.1,0);}

%%%% FIGURES %%%%
\usepackage{subcaption}
\usepackage{wrapfig}
\usepackage{float}
\usepackage[skip=5pt, font=footnotesize]{caption}

%%%% FORMATTING %%%%
\usepackage{parskip}
\usepackage{tcolorbox}
\usepackage{ulem}

%%%% TABLE FORMATTING %%%%
\usepackage{tabularray}
\UseTblrLibrary{booktabs}

%%%% MATH AND LOGIC %%%%
\usepackage{xifthen}
\usepackage{amsmath}
\usepackage{amssymb}
\usepackage{amsfonts}

%%%% TEXT AND SYMBOLS %%%%
\usepackage[T1]{fontenc}
\usepackage{textcomp}
\usepackage{gensymb}

%%%% OTHER %%%%
\usepackage{standalone}

%%%% LOGIC SYMBOLS %%%%
\newcommand*\xor{\oplus}

%%%% STYLES %%%%

% Packages
\usepackage[paper=letterpaper,tmargin=45pt,bmargin=45pt,lmargin=45pt,rmargin=45pt]{geometry}
\usepackage{titlesec}
\usepackage[rgb]{xcolor}
\selectcolormodel{natural}
\usepackage{ninecolors}
\selectcolormodel{rgb}

% Colors
\definecolor{pg}{HTML}{24273A}
\definecolor{fg}{HTML}{FFFFFF}
\definecolor{bg}{HTML}{24273A}
\definecolor{re}{HTML}{F38BA8}
\definecolor{gr}{HTML}{A6E3A1}
\definecolor{ye}{HTML}{F9E2AF}
\definecolor{or}{HTML}{FAB387}
\definecolor{bl}{HTML}{89B4FA}
\definecolor{ma}{HTML}{CBA6F7}
\definecolor{cy}{HTML}{94E2D5}
\definecolor{pi}{HTML}{F2CDCD}

\definecolor{copper}{HTML}{B87333}

\usepackage{nameref}
\makeatletter
\newcommand*{\currentname}{\@currentlabelname}
\makeatother

\titleformat{\section}
  {\normalfont\scshape\Large\bfseries}
  {\thesection}
  {0.75em}
  {}

\titleformat{\subsection}
  {\normalfont\scshape\large\bfseries}
  {\thesubsection}
  {0.75em}
  {}

\titleformat{\subsubsection}
  {\normalfont\scshape\normalsize\bfseries}
  {\thesubsubsection}
  {0.75em}
  {}

% Formula
\newcounter{formula}[section]
\newenvironment{formula}[1]{
  \stepcounter{formula}
  \begin{tcolorbox}[
    standard jigsaw, % Allows opacity
    colframe={fg},
    boxrule=1px,
    colback=bg,
    opacityback=0,
    sharp corners,
    sidebyside,
    righthand width=18px,
    coltext={fg}
  ]
  \centering
  \textbf{\uline{#1}}
}{
  \tcblower
  \textbf{\thesection.\theformula}
  \end{tcolorbox}
}

% Definition
\newcounter{definition}[section]

\newenvironment{definition*}[1]{
  \begin{tcolorbox}[
    standard jigsaw, % Allows opacity
    colframe={fg},
    boxrule=1px,
    colback=bg,
    opacityback=0,
    sharp corners,
    coltext={fg}
  ]
  \textbf{#1 \hfill}
  \vspace{5px}
  \hrule
  \vspace{5px}
  \noindent
}{
  \end{tcolorbox}
}

\newenvironment{definition}[1]{
  \stepcounter{definition}
  \begin{tcolorbox}[
    standard jigsaw, % Allows opacity
    colframe={fg},
    boxrule=1px,
    colback=bg,
    opacityback=0,
    sharp corners,
    coltext={fg}
  ]
  \textbf{#1 \hfill \thesection.\thedefinition}
  \vspace{5px}
  \hrule
  \vspace{5px}
  \noindent
}{
  \end{tcolorbox}
}

% Example Problem
\newcounter{example}[section]
\newenvironment{example}{
  \stepcounter{example}
  \begin{tcolorbox}[
    standard jigsaw, % Allows opacity
    colframe={fg},
    boxrule=1px,
    colback=bg,
    opacityback=0,
    sharp corners,
    coltext={fg}
  ]
  \textbf{Example \hfill \thesection.\theexample}
  \vspace{5px}
  \hrule
  \vspace{5px}
  \noindent
}{
  \end{tcolorbox}
}

\tikzset{
  cubeBorder/.style=fg,
  cubeFilling/.style={fg!20!bg, opacity=0.25},
  gridLine/.style={very thin, gray},
  graphLine/.style={-latex, thick, fg},
}

\pgfplotsset{
  basicAxis/.style={
    grid,
    major grid style={line width=.2pt,draw=fg!50!bg},
    axis lines = box,
    axis line style = {line width = 1px},
  }
}

%%%% REFERENCES %%%%
\usepackage{hyperref}
\hypersetup{
  colorlinks  = true,
  linkcolor   = bl,
  anchorcolor = bl,
  citecolor   = bl,
  filecolor   = bl,
  menucolor   = bl,
  runcolor    = bl,
  urlcolor    = bl,
}

\author{Ethan Anthony}
\newcommand*{\equal}{=}


\title{Lecture 005}
\date{February 03, 2025}

\begin{document}

\newpage
\section{Bending}
\label{sec:bending}

\subsection{Pure Bending}
\label{ssec:pureBending}

\begin{figure}[H]
  \centering
  \includestandalone{figures/fig_057}
  \caption{Member Experiencing Pure Bending}
  \label{fig:057}
\end{figure}

Consider the free body diagram in Figure \ref{fig:057}. Assuming static equilibrium, the forces acting upon it would create bending moments at points $B$ and $C$ as is shown in Figure \ref{fig:058}.

\begin{figure}[H]
  \centering
  \includestandalone{figures/fig_058}
  \caption{Bending Moments in Member}
  \label{fig:058}
\end{figure}

The pure symmetry of the bending of the bar ($M' = M$) results in \textbf{pure bending} in the member between $B$ and $C$. This is because there is \textit{only} bending in the member.

\begin{definition}{Pure Bending}
  A situation in which bending exists in a member without the presence of axial, shear, or torsional forces.
\end{definition}

If a cross section of the member were to be taken, the internal forces present would be a combination of a shear force acting in the vertical direction and a bending moment. Note these directions as correspondent to the \textit{negative} bending seen in Figure \ref{fig:058}.
\begin{figure}[H]
  \centering
  \begin{subfigure}[H]{0.45\textwidth}
    \centering
    \includestandalone{figures/fig_060}
    \caption{Bending Moment}
    \label{fig:060}
  \end{subfigure}
  \begin{subfigure}[H]{0.45\textwidth}
    \centering
    \includestandalone{figures/fig_059}
    \caption{Shear Force}
    \label{fig:059}
  \end{subfigure}
  \caption{Internal Forces of a Bending Member}
  \label{fig:internalForcesOfABendingMember}
\end{figure}

\subsubsection{Bending Strain}
\label{sssec:bendingStrain}

Consider the internal tension/compression forces seen in Figure \ref{fig:060}. Notice that, as the forces approach the \textit{centroid} of the beam, the magnitude of the forces decreases. To better understand this, consider the beam in Figure \ref{fig:061}. Along its centroid in green in the {\color{gr} neutral axis}. Along this axis, the beam experiences no elongation. However, along the dotted line, the beam does elongate, implying the existence of internal axial forces.

\begin{wrapfigure}[15]{r}{0.5\textwidth}
  \centering
  \includestandalone{figures/fig_061}
  \caption{Neutral Axis}
  \label{fig:061}
\end{wrapfigure}

Given the dimensions in Figure \ref{fig:061}, the length of the arc of the {\color{gr} neutral axis} within the beam can be calculated as:
\begin{equation*}
  l_{\textup{neutral arc}} = r \cdot \theta
\end{equation*}
Similarly, the length of the dashed arc within the beam is:
\begin{equation*}
  l_{\textup{dashed}} = (r+y) \cdot \theta
\end{equation*}
The strain at any point can then be calculated as the ratio between these two values:
\begin{align*}
  \epsilon = \frac{\Delta L}{L_0} = \frac{\left((r+y) \cdot \theta\right) - r \cdot \theta}{r \cdot \theta}
\end{align*}
This can then be simplified into:
\begin{formula}{Bending Strain}
  \begin{equation*}
    \epsilon = \frac{y}{r}
  \end{equation*}
\end{formula}

\subsection{Bending Stress}
\label{ssec:bendingStress}

During bending that remains within the elastic, non-plastic, range, Hooke's Law applies. This means that the bending stress can be found by multiplying the strain by some elastic constant:
\begin{formula}{Bending Stress}
  \begin{equation*}
    \sigma = E \epsilon = E \frac{y}{r}
  \end{equation*}
\end{formula}

\begin{wrapfigure}[5]{r}{0.3\textwidth}
  \vspace{-20pt}
  \centering
  \includestandalone{figures/fig_060}
  \caption{Bending Moment}
  \label{fig:060_b}
\end{wrapfigure}

Since the internal moment is created by internal forces, the resultant forces of all of these internal forces must equal the moment. Thus, by integrating the internal forces, the moment can be found.
\begin{equation*}
  M = \int_{A}^{} \sigma y \,dA = \int_{A}^{} E\frac{y}{r} y \,dA = \frac{E}{r} \int_{A}^{} y^2 \,dA
\end{equation*}

\subsection{Flexure Formula}
\label{ssec:flexureFormula}

The \textbf{bending moment of inertia} ($I$) of an object quantifies how much a beam resists bending. This value is calculated as:
\begin{equation*}
  \int_{A}^{} y^2 \,dA
\end{equation*}
Thus:
\begin{equation*}
  M = \frac{E}{r} \int_{A}^{} y^2 \,dA = \frac{EI}{r}
\end{equation*}
\clearpage
Combining this equation for $M$ with the previous equation for $\sigma$ results in the \textbf{Flexure Formula}.
\begin{formula}{Flexure Formula}
  \begin{equation*}
    \sigma = \frac{My}{I}
  \end{equation*}
  where $I_{rectangle} = \frac{bh^3}{12}$, $I_{circle} = \frac{\pi\left(r_o^4-r_i^4\right)}{4}$ is the \textbf{area moment of inertia}\\
  or
  \begin{equation*}
    \sigma_{max} = \frac{M}{S}
  \end{equation*}
  where $S=\frac{I}{y_{max}}=\frac{I}{C}$ is the \textbf{Section Modulus}
\end{formula}

\subsection{Moment of Inertia}
\label{ssec:momentOfInertia}

The moment of inertia for some common shapes can be calculated geometrically (see Table \ref{tbl:sectionModulusOfSeveralCrossSections}). However, the general process of calculating a moment of inertia is as follows:
\begin{enumerate}
  \itemsep0em
  \item Calculate the centroid of the shape ($\overline{y}$)
  \item Calculate moment of inertia for each rectangular segment independently using the \textbf{parallel axis theorem}
\end{enumerate}

\subsection{Impure Bending}
\label{ssec:impureBending}

Just as pure bending happens when there are no shear forces present during bending, impure bending is when shear \textit{and} bending coexist. Luckily, the flexure formula from Subsection \ref{ssec:flexureFormula} holds over all cases of pure bending, and most cases of impure bending.

\begin{wrapfigure}[9]{l}{0.35\textwidth}
  \vspace{-20pt}
  \centering
  \includestandalone{figures/fig_062}
  \caption{Bending Shear}
  \label{fig:062}
\end{wrapfigure}

The average shear across some cross-sectional face, such as that in Figure \ref{fig:062}, is simply calculated as:
\begin{equation*}
  \tau = \frac{V}{A}
\end{equation*}
However, as can be seen, the shear forces are not distributed evenly throughout the face. So, how can the maximum shear stress be found?
\begin{formula}{Shear Stress}
  \begin{equation*}
    \tau(x,y) = \frac{V(x) \cdot Q(y)}{I \cdot b(y)}
  \end{equation*}
\end{formula}
Where $V$ is the shear force, $I$ is the area moment of intertia, $b$ is the width of the face, and $Q$ is the \textbf{first moment of area}.
\begin{formula}{First Moment of Area}
  \begin{equation*}
    Q = \textup{{\color{gr} area outside of axis}} \cdot \textup{{\color{bl} distance to centroid from neutral axis}}
  \end{equation*}
  \begin{equation*}
    Q = {\color{gr} b\left( \frac{h}{2}-y \right)} \cdot {\color{bl} y + \frac{\frac{h}{2}-y}{2}} = \frac{b}{2}\left( \frac{h^2}{4}-y^2 \right)
  \end{equation*}
\end{formula}
In the case of Figure \ref{fig:062}, the width is constant along the height of the cross-section just as the shear force $V$ is constant throughout the face. With these considerations, the equation of shear stress can be expressed as:
\begin{wrapfigure}[4]{l}{0.35\textwidth}
  \centering
  \includestandalone{figures/fig_063}
  \caption{First Moment of Area}
  \label{fig:063}
\end{wrapfigure}
\begin{formula}{Shear Stress}
  \begin{equation*}
    \tau = \frac{V}{2I}\left(\frac{h^2}{4}-y^2\right)
  \end{equation*}
\end{formula}
By inspecting this formula, it can be seen that the \textit{maximum} shear stress would be found at the neutral axis of the face (where $y=0$). By substituting in $y=0$ to the shear stress formula, a formula for the max shear stress ($\tau_{max}$) can be found:
\begin{formula}{Maximum Shear Stress}
  \begin{gather*}
    \tau_{\textup{max,rectangle}} = \frac{V}{2I}\left(\frac{h^2}{4}-0^2\right) = \frac{V}{2I}\frac{h^2}{4} = \frac{3}{2}\frac{V}{A} \\
    \tau_{\textup{max,circle}} = \frac{4}{3}\frac{V}{A}
  \end{gather*}
\end{formula}

\subsection{Stress Concentrations}
\label{ssec:bendingStressConcentrations}

Just as it was in Subsection \ref{ssec:axialStressConcentrations}, there are also stress concentrations when it comes to non-uniform members experiencing bending. Generally:
\begin{formula}{Maximum Stress}
  \begin{equation*}
    \sigma_{max} = k \frac{Mc}{I} = k \sigma_{avg}
  \end{equation*}
\end{formula}
Where $k$ is the experimentally derived stress concentration factor. To find $k$ for any given member, again, only the geometry of the member must be analyzed. The relevant geometric properties for two types of non-uniform shapes are shown in Figure \ref{fig:stressConcentrationsInBending}.
\begin{figure}[H]
  \vspace{-15pt}
  \centering
  \begin{subfigure}[H]{0.48\textwidth}
    \centering
    \includestandalone{figures/fig_064}
  \end{subfigure}
  \begin{subfigure}[H]{0.48\textwidth}
    \centering
    \includestandalone{figures/fig_065}
  \end{subfigure}
  \caption{Stress Concentrations in Bending}
  \label{fig:stressConcentrationsInBending}
\end{figure}

\subsection{Eccentric Bending}
\label{ssec:eccentricBending}

\subsubsection{Symmetric Eccentric Bending}
\label{sssec:symmetricEccentricBending}

In all the previous axial loading analysis, it was assumed that the force creating the axial stresses was acting through the centroid of the cross-section of the member. In other words, it was assumed that the forces were perfectly centered. In those cases, only axial stresses occurred. However, what happens in a member is loaded eccentrically (not through its centroid)? This results in \textbf{eccentric loading}.
\begin{definition}{Eccentric Loading}
  Eccentric loading occurs when a member is loaded such that the forces creating the load do not have a line of action through the centroid of the member's cross-section.
\end{definition}
\begin{figure}[H]
  \centering
  \includestandalone{figures/fig_066}
  \caption{Eccentric Loading}
  \label{fig:066}
\end{figure}
When eccentric loading occurs, bending is created. Consider Figure \ref{fig:066}. It might be intuitive to see that these forces $P$ will create some sort of bending through the section $BD$. This can be seen better if an imaginary cut through the object were to be made as in Figure \ref{fig:067}.
\begin{figure}[H]
  \centering
  \includestandalone{figures/fig_067}
  \caption{Eccentric Loading}
  \label{fig:067}
\end{figure}
Further dissecting this member, it can be seen in Figure \ref{fig:068} that the section $BD$ is experiencing both centric and bending stresses.
\begin{figure}[H]
  \centering
  \includestandalone{figures/fig_068}
  \caption{Eccentric Loading}
  \label{fig:068}
\end{figure}
From here, the internal stresses can be expressed as the sum of the internal stresses caused be each of the two loads: the centric axial load and the bending moments.
\begin{figure}[H]
  \centering
  \begin{subfigure}[H]{0.3\textwidth}
    \centering
    \includestandalone{figures/fig_069}
    \caption{Centric Load}
    \label{fig:069}
  \end{subfigure}
  {\Large \textbf{$+$}}
  \begin{subfigure}[H]{0.3\textwidth}
    \centering
    \includestandalone{figures/fig_070}
    \caption{Bending Moment}
    \label{fig:070}
  \end{subfigure}
  {\Large \textbf{$=$}}
  \begin{subfigure}[H]{0.3\textwidth}
    \centering
    \includestandalone{figures/fig_071}
    \caption{Total Stress}
    \label{fig:071}
  \end{subfigure}
\end{figure}
\begin{equation*}
  \sigma = \frac{P}{A} - \frac{My}{I}
\end{equation*}

\subsubsection{General Eccentric Bending}
\label{sssec:generalEccentricBending}

Applying the same principles from Subsubsection \ref{sssec:symmetricEccentricBending}, a more generalized approach to eccentric bending that isn't symmetric on any plane with the member can be formulated. Consider Figure \ref{fig:072}.
\begin{figure}[H]
  \centering
  \includestandalone{figures/fig_072}
  \caption{Asymmetric Eccentric Bending}
  \label{fig:072}
\end{figure}
These axial forces $P$ and $P'$ can be substituted for three component loads applied centrically: a moment $M_z$, a moment $M_y$, and a force $P$. Thus, a more generalized formula can be expressed as:
\begin{formula}{Axial Stress Under Eccentric Loading}
  \begin{equation*}
    \sigma = \frac{P}{A} - \frac{M_zy}{I_z} + \frac{M_yz}{I_y}
  \end{equation*}
\end{formula}

\subsection{Prismatic Beams}
\label{ssec:prismaticBeams}

The maximum absolute value of the bending moment ($|M|_{max}$) generally guides the design process for a beam. The maximum normal stress ($\sigma_{max}$) is found at the surface where $|M|_{max}$ occurs. Thus, to find the maximum stress, $|M|_{max}$ can be substituted into the flexure formula:
\begin{equation*}
  \sigma_{max} = \frac{|M|_{max}y}{I} = \frac{|M|_{max}}{S}
\end{equation*}
Considering that the maximum stress cannot exceed the allowable stress ($\sigma_{all}$) at any point in the beam, by substituting $\sigma_{all}$ for $\sigma_{max}$, the minimum section modulus ($S_{min}$) can be solved for:
\begin{equation*}
  \sigma_{all} = \frac{|M|_{max}}{S_{min}} \Rightarrow S_{min} = \frac{|M|_{max}}{\sigma_{all}}
\end{equation*}
These considerations will guide the design of efficient and structurally sound beams.

The section modulus of many cross-sections can be calculated geometrically. For common shapes, see Table \ref{tbl:sectionModulusOfSeveralCrossSections}. The {\color{gr} neutral axis} is highlighted in green for each figure.

\begin{figure}[H]
  \centering
  \begin{tblr}{ccc}
    \toprule
    Geometry & Moment of Inertia & Section Modulus \\
    \midrule
    \includestandalone{figures/fig_073} & $temp$                                     & $S = \frac{BH^2}{6}-\frac{bh^3}{6H}$ \\
    \midrule
    \includestandalone{figures/fig_074} & $temp$                                     & $S = \frac{B^2\left(H-h\right)}{6} + \frac{(B-b)^3 \cdot h}{6B}$ \\
    \midrule
    \includestandalone{figures/fig_075} & $I = \frac{1}{12}\left(BH^3 - bh^3\right)$ & $S = \frac{BH^2}{6} - \frac{bh^3}{6H}$ \\
    \midrule
    \includestandalone{figures/fig_077} & $I = \frac{pi}{4} \left( R^4-r^4 \right)$  & $S = \frac{pi}{4} \cdot \frac{R^4 - r^4}{R}$ \\
    \midrule
    \includestandalone{figures/fig_076} & $temp$                                     & $S = \frac{BH^2}{24}$ \\
    \midrule
    \includestandalone{figures/fig_078} & $temp$                                     & $S = \frac{BH^2}{6} - \frac{bh^3}{6H}$ \\
    \bottomrule
  \end{tblr}
  \caption{Section Modulus of Several Cross-Sections}
  \label{tbl:sectionModulusOfSeveralCrossSections}
\end{figure}

\end{document}
