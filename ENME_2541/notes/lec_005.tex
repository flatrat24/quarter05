\documentclass[12pt]{article}

\input{../../xlatex/imports/preamble}

\title{Lecture 005}
\date{February 03, 2025}

\begin{document}

\newpage
\section{Bending}
\label{sec:bending}

\subsection{Pure Bending}
\label{ssec:pureBending}

\begin{figure}[H]
  \centering
  \includestandalone{figures/fig_057}
  \caption{Member Experiencing Pure Bending}
  \label{fig:057}
\end{figure}

Consider the free body diagram in Figure \ref{fig:057}. Assuming static equilibrium, the forces acting upon it would create bending moments at points $B$ and $C$ as is shown in Figure \ref{fig:058}.

\begin{figure}[H]
  \centering
  \includestandalone{figures/fig_058}
  \caption{Bending Moments in Member}
  \label{fig:058}
\end{figure}

The pure symmetry of the bending of the bar ($M' = M$) results in \textbf{pure bending} in the member between $B$ and $C$. This is because the shear force along this section is zero.

\begin{definition}{Pure Bending}
  A situation in which bending exists in a member without the presence of axial, shear, or torsional forces.
\end{definition}

If a cross section of the member were to be taken, the internal forces present would be a combination of a shear force acting in the vertical direction and a bending moment.

\begin{figure}[H]
  \centering
  \begin{subfigure}[H]{0.45\textwidth}
    \centering
    \includestandalone{figures/fig_059}
    \caption{Shear Force}
    \label{fig:059}
  \end{subfigure}
  \begin{subfigure}[H]{0.45\textwidth}
    \centering
    \includestandalone{figures/fig_060}
    \caption{Bending Moment}
    \label{fig:060}
  \end{subfigure}
  \caption{Internal Forces of a Bending Member}
  \label{fig:internalForcesOfABendingMember}
\end{figure}

\end{document}
