\documentclass[12pt]{article}

%%%% GRAPHICS %%%%
\usepackage{tikz}
\usepackage[siunitx, american, RPvoltages]{circuitikz}
\usetikzlibrary{arrows.meta}
\usepackage{tikz-3dplot}
\usepackage{graphicx}
\usepackage{pgfplots}
  \pgfplotsset{compat=1.18}
\usetikzlibrary{arrows}
\newcommand{\midarrow}{\tikz \draw[-triangle 90] (0,0) -- +(.1,0);}

%%%% FIGURES %%%%
\usepackage{subcaption}
\usepackage{wrapfig}
\usepackage{float}
\usepackage[skip=5pt, font=footnotesize]{caption}

%%%% FORMATTING %%%%
\usepackage{parskip}
\usepackage{tcolorbox}
\usepackage{ulem}

%%%% TABLE FORMATTING %%%%
\usepackage{tabularray}
\UseTblrLibrary{booktabs}

%%%% MATH AND LOGIC %%%%
\usepackage{xifthen}
\usepackage{amsmath}
\usepackage{amssymb}
\usepackage{amsfonts}

%%%% TEXT AND SYMBOLS %%%%
\usepackage[T1]{fontenc}
\usepackage{textcomp}
\usepackage{gensymb}

%%%% OTHER %%%%
\usepackage{standalone}

%%%% LOGIC SYMBOLS %%%%
\newcommand*\xor{\oplus}

%%%% STYLES %%%%

% Packages
\usepackage[paper=letterpaper,tmargin=45pt,bmargin=45pt,lmargin=45pt,rmargin=45pt]{geometry}
\usepackage{titlesec}
\usepackage[rgb]{xcolor}
\selectcolormodel{natural}
\usepackage{ninecolors}
\selectcolormodel{rgb}

% Colors
\definecolor{pg}{HTML}{24273A}
\definecolor{fg}{HTML}{FFFFFF}
\definecolor{bg}{HTML}{24273A}
\definecolor{re}{HTML}{F38BA8}
\definecolor{gr}{HTML}{A6E3A1}
\definecolor{ye}{HTML}{F9E2AF}
\definecolor{or}{HTML}{FAB387}
\definecolor{bl}{HTML}{89B4FA}
\definecolor{ma}{HTML}{CBA6F7}
\definecolor{cy}{HTML}{94E2D5}
\definecolor{pi}{HTML}{F2CDCD}

\definecolor{copper}{HTML}{B87333}

\usepackage{nameref}
\makeatletter
\newcommand*{\currentname}{\@currentlabelname}
\makeatother

\titleformat{\section}
  {\normalfont\scshape\Large\bfseries}
  {\thesection}
  {0.75em}
  {}

\titleformat{\subsection}
  {\normalfont\scshape\large\bfseries}
  {\thesubsection}
  {0.75em}
  {}

\titleformat{\subsubsection}
  {\normalfont\scshape\normalsize\bfseries}
  {\thesubsubsection}
  {0.75em}
  {}

% Formula
\newcounter{formula}[section]
\newenvironment{formula}[1]{
  \stepcounter{formula}
  \begin{tcolorbox}[
    standard jigsaw, % Allows opacity
    colframe={fg},
    boxrule=1px,
    colback=bg,
    opacityback=0,
    sharp corners,
    sidebyside,
    righthand width=18px,
    coltext={fg}
  ]
  \centering
  \textbf{\uline{#1}}
}{
  \tcblower
  \textbf{\thesection.\theformula}
  \end{tcolorbox}
}

% Definition
\newcounter{definition}[section]

\newenvironment{definition*}[1]{
  \begin{tcolorbox}[
    standard jigsaw, % Allows opacity
    colframe={fg},
    boxrule=1px,
    colback=bg,
    opacityback=0,
    sharp corners,
    coltext={fg}
  ]
  \textbf{#1 \hfill}
  \vspace{5px}
  \hrule
  \vspace{5px}
  \noindent
}{
  \end{tcolorbox}
}

\newenvironment{definition}[1]{
  \stepcounter{definition}
  \begin{tcolorbox}[
    standard jigsaw, % Allows opacity
    colframe={fg},
    boxrule=1px,
    colback=bg,
    opacityback=0,
    sharp corners,
    coltext={fg}
  ]
  \textbf{#1 \hfill \thesection.\thedefinition}
  \vspace{5px}
  \hrule
  \vspace{5px}
  \noindent
}{
  \end{tcolorbox}
}

% Example Problem
\newcounter{example}[section]
\newenvironment{example}{
  \stepcounter{example}
  \begin{tcolorbox}[
    standard jigsaw, % Allows opacity
    colframe={fg},
    boxrule=1px,
    colback=bg,
    opacityback=0,
    sharp corners,
    coltext={fg}
  ]
  \textbf{Example \hfill \thesection.\theexample}
  \vspace{5px}
  \hrule
  \vspace{5px}
  \noindent
}{
  \end{tcolorbox}
}

\tikzset{
  cubeBorder/.style=fg,
  cubeFilling/.style={fg!20!bg, opacity=0.25},
  gridLine/.style={very thin, gray},
  graphLine/.style={-latex, thick, fg},
}

\pgfplotsset{
  basicAxis/.style={
    grid,
    major grid style={line width=.2pt,draw=fg!50!bg},
    axis lines = box,
    axis line style = {line width = 1px},
  }
}

%%%% REFERENCES %%%%
\usepackage{hyperref}
\hypersetup{
  colorlinks  = true,
  linkcolor   = bl,
  anchorcolor = bl,
  citecolor   = bl,
  filecolor   = bl,
  menucolor   = bl,
  runcolor    = bl,
  urlcolor    = bl,
}

\author{Ethan Anthony}
\newcommand*{\equal}{=}


\title{Lecture 001}
\date{January 06, 2025}

\begin{document}

\section{Types of Stress}
\label{sec:typesOfStress}

\subsection{Axial Stress}
\label{ssec:axialStress}

\subsubsection{Overview}
\label{sssec:overview}

Mechanics of materials provides a means to analyze the effects of \textit{stresses} and \textit{deformations}. Statics covered finding a balance of forces, including internal forces such as \textit{shear}, \textit{bending}, and \textit{tension/compression}. Finding these internal forces is imperative to be able to determine the integrity of a structure. In addition to the internal forces, the integrity of a structure is also partially determined by the \textit{dimensions} and \textit{materials} of that structure.

In the analysis of a rod, for example, the ability for that rod to withstand the internal forces (its structural integrity) is determined by the cross-sectional area and the material of the rod.

\begin{figure}[H]
  \centering
  \begin{subfigure}[H]{0.45\textwidth}
    \centering
    \includestandalone{figures/fig_001}
    \caption{Resultant Force}
    \label{fig:001}
  \end{subfigure}
  \begin{subfigure}[H]{0.45\textwidth}
    \centering
    \includestandalone{figures/fig_002}
    \caption{Elementary Forces}
    \label{fig:002}
  \end{subfigure}
  \caption{Axial force is the resultant of distributed elementary forces}
  \label{fig:axialVsElementary}
\end{figure}

\subsubsection{Analyzing Axial Stress}
\label{sssec:analyzingAxialStress}

In Figure \ref{fig:axialVsElementary}, the \textit{stress} being experiences by the member is the \textbf{force per unit area}, denoted by the Greek letter sigma ($\sigma$).

\begin{formula}{Axial Stress}

  \vspace{10pt}
  Stress can be calculated by dividing the total axial force by the cross-sectional area: 
  \begin{equation*}
    \sigma = \frac{P}{A}
  \end{equation*}
\end{formula}

This formula gives the \textit{average} axial stress over the cross section of a member. This stress can be assumed to be \textit{uniform} throughout the cross section.

\vspace{12pt}
\hrule
\vspace{2pt}

By convention, a positive force indicates \textbf{tensile stress} while a negative force indicates \textbf{compressive stress}.

\begin{figure}[H]
  \centering
  \begin{subfigure}[H]{0.45\textwidth}
    \centering
    \includestandalone{figures/fig_003}
    \caption{Tensile Stress}
    \label{fig:003}
  \end{subfigure}
  \begin{subfigure}[H]{0.45\textwidth}
    \centering
    \includestandalone{figures/fig_004}
    \caption{Compressive Stress}
    \label{fig:004}
  \end{subfigure}
  \caption{Tensile vs. Compressive Force}
  \label{fig:tensileVsCompressiveForce}
\end{figure}

\hrule
\vspace{5pt}

The cross section, as seen in Figure \ref{fig:005}, is perpendicular to the axial forces. The corresponding stress in the member in described as \textit{normal stress}.

\begin{figure}[H]
  \vspace{-20pt}
  \centering
  \includestandalone{figures/fig_005}
  \caption{Normal Stress}
  \label{fig:005}
\end{figure}

Thus, the formula of $\sigma = \frac{P}{A}$ gives the normal stress of an member under axial loading.

\subsubsection{Stress Points}
\label{sssec:stressPoints}

The formula of $\sigma = \frac{P}{A}$ is only useful for \textit{averages} or \textit{ranges}. This can calculate the average value of the stress over the entire cross section. However, what about calculating at specific points?

\begin{wrapfigure}[6]{l}{0.4\textwidth}
  \centering
  \includestandalone{figures/fig_006}
  \caption{Stress Points}
  \label{fig:006}
\end{wrapfigure}

In Figure \ref{fig:006}, to find the stress of the highlighted area, the working equation can still be applied, just now with a smaller area. To find the stress at a single point, rather than just a smaller area, stress must be calculated as the area approaches zero.

\begin{formula}{Stress at a Single Point}
  \begin{equation*}
    \sigma = \lim_{\Delta A \to 0} \frac{\Delta F}{\Delta A}
  \end{equation*}
\end{formula}

\subsection{Shearing Stress}
\label{ssec:shearingStress}

\subsubsection{Overview}
\label{sssec:overview2}

In Section \ref{ssec:axialStress}, the internal forces and corresponding stresses were \textit{normal} to the member.

\begin{definition}{Shear Stress}
  Internal stresses caused by forces with parallel and opposite components vectors.
\end{definition}

\begin{wrapfigure}[12]{r}{0.35\textwidth}
  \vspace{-20pt}
  \centering
  \begin{subfigure}[H]{0.35\textwidth}
    \centering
    \includestandalone{figures/fig_007}
    \caption{Opposing Forces Creating Shear}
    \label{fig:007}
  \end{subfigure}
  \begin{subfigure}[H]{0.35\textwidth}
    \centering
    \includestandalone{figures/fig_008}
    \caption{Opposing Forces Creating Shear}
    \label{fig:008}
  \end{subfigure}
  \caption{Shear Stress}
  \label{fig:shearingStress}
\end{wrapfigure}

In Figure \ref{fig:shearingStress}, a member has two opposing forces, $P_a$ and $P_b$, acting upon it. These two forces create internal shear in the section between them as seen in Figure \ref{fig:008}.

By analyzing the section at $C$ in the member, it can be seen that the internal force created by the two forces in the section is equal to $P_b$. This resultant force is called a \textbf{shear force ($P$)}. Dividing $P$ by the area ($A$) results in the \textbf{average shearing stress} in the section.

\begin{formula}{Average Shearing Stress}
  \begin{equation*}
    \tau_{avg} = \frac{P}{A}
  \end{equation*}
\end{formula}

The average shearing stress cannot be assumed to be uniform throughout the section. Though not the full story, the shearing stress is generally distributed throughout the section such that it is zero at the surface and greater than the average near the center.

\subsubsection{Single and Double Shear}
\label{sssec:singleAndDoubleShear}

The member in Figure \ref{fig:shearingStress} is said to be in \textbf{single shear} because there is only a single section in shear. However, if forces were to be applied to a member as in Figure \ref{fig:009}, there are now two sections and thus the member is said to be in \textbf{double shear}.

\begin{figure}[H]
  \centering
  \includestandalone{figures/fig_009}
  \caption{Double Shear}
  \label{fig:009}
\end{figure}

\subsubsection{Bearing Stress}
\label{sssec:bearingStress}

Often, rather than singular forces creating shearing stress, the surfaces of multiple objects will apply distributed forces to each other, thus creating \textbf{bearing shear}.

\begin{definition}{Bearing Shear}
  Caused by distributed forces creating shear, bearing stress is the average shear across a region of a member.
\end{definition}

\begin{wrapfigure}[8]{l}{0.4\textwidth}
  \centering
  \includestandalone{figures/fig_010}
  \caption{Bearing Shear}
  \label{fig:010}
\end{wrapfigure}

In Figure \ref{fig:010}, two objects are applying reciprocal forces to each other. Throughout the area the two objects interface over, the distribution of the shear is not even. So, in practice, the average nominal value of the stress ($\sigma_b$) called the \textbf{bearing stress} is used.

This value is obtained by dividing the load $P$ by the area of the surface of contact.

\begin{formula}{Bearing Stress}
  \begin{equation*}
    \sigma_b = \frac{P}{A} = \frac{P}{td}
  \end{equation*}
\end{formula}



\end{document}
