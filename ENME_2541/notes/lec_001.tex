\documentclass[12pt]{article}

\input{../../xlatex/imports/preamble}

\title{Lecture 001}
\date{January 06, 2025}

\begin{document}

\section{Axial Stress}
\label{sec:axialStress}

\subsection{Overview}
\label{ssec:overview}

Mechanics of materials provides a means to analyze the effects of \textit{stresses} and \textit{deformations}. Statics covered finding a balance of forces, including internal forces such as \textit{shear}, \textit{bending}, and \textit{tension/compression}. Finding these internal forces is imperative to be able to determine the integrity of a structure. In addition to the internal forces, the integrity of a structure is also partially determined by the \textit{dimensions} and \textit{materials} of that structure.

In the analysis of a rod, for example, the ability for that rod to withstand the internal forces (its structural integrity) is determined by the cross-sectional area and the material of the rod.

\begin{figure}[H]
  \centering
  \begin{subfigure}[H]{0.45\textwidth}
    \centering
    \includestandalone{figures/fig_001}
    \caption{Resultant Force}
    \label{fig:001}
  \end{subfigure}
  \begin{subfigure}[H]{0.45\textwidth}
    \centering
    \includestandalone{figures/fig_002}
    \caption{Elementary Forces}
    \label{fig:002}
  \end{subfigure}
  \caption{Axial force is the resultant of distributed elementary forces}
  \label{fig:axialVsElementary}
\end{figure}

\subsection{Analyzing Axial Stress}
\label{ssec:analyzingAxialStress}

In Figure \ref{fig:axialVsElementary}, the \textit{stress} being experiences by the object is the \textbf{force per unit area}, denoted by the Greek letter sigma ($\sigma$).

\begin{formula}{Stress}

  \vspace{10pt}
  Stress can be calculated by dividing the total axial force by the cross-sectional area: 
  \begin{equation*}
    \sigma = \frac{P}{A}
  \end{equation*}
\end{formula}

\vspace{12pt}
\hrule
\vspace{2pt}

By convention, a positive force indicates \textbf{tensile stress} while a negative force indicates \textbf{compressive stress}.

\begin{figure}[H]
  \centering
  \begin{subfigure}[H]{0.45\textwidth}
    \centering
    \includestandalone{figures/fig_003}
    \caption{Tensile Stress}
    \label{fig:003}
  \end{subfigure}
  \begin{subfigure}[H]{0.45\textwidth}
    \centering
    \includestandalone{figures/fig_004}
    \caption{Compressive Stress}
    \label{fig:004}
  \end{subfigure}
  \caption{Tensile vs. Compressive Force}
  \label{fig:tensileVsCompressiveForce}
\end{figure}

\hrule
\vspace{5pt}

\begin{wrapfigure}[5]{r}{0.4\textwidth}
  \vspace{-20pt}
  \centering
  \includestandalone{figures/fig_005}
  \caption{Normal Stress}
  \label{fig:005}
\end{wrapfigure}

The cross section, as seen in Figure \ref{fig:005}, is perpendicular to the axial forces. The corresponding stress in the object in described as \textit{normal stress}.

Thus, the formula of $\sigma = \frac{P}{A}$ gives the normal stress of an object under axial loading.

\subsection{Stress Points}
\label{ssec:stressPoints}

The formula of $\sigma = \frac{P}{A}$ is only useful for \textit{averages} or \textit{ranges}. This can calculate the average value of the stress over the entire cross section. However, what about calculating at specific points?

\begin{wrapfigure}[]{l}{0.4\textwidth}
  \centering
  \includestandalone{figures/fig_006}
  \caption{Stress Points}
  \label{fig:006}
\end{wrapfigure}

In Figure \ref{fig:006}, to find the stress of the highlighted area, the working equation can still be applied, just now with a smaller area. To find the stress at a single point, rather than just a smaller area, stress must be calculated as the area approaches zero.

\begin{formula}{Stress at a Single Point}
  \begin{equation*}
    \sigma = \lim_{\Delta A \to 0} \frac{\Delta F}{\Delta A}
  \end{equation*}
\end{formula}



\end{document}
