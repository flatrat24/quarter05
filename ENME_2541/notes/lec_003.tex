\documentclass[12pt]{article}
\usepackage{physics}

%%%% GRAPHICS %%%%
\usepackage{tikz}
\usepackage[siunitx, american, RPvoltages]{circuitikz}
\usetikzlibrary{arrows.meta}
\usepackage{tikz-3dplot}
\usepackage{graphicx}
\usepackage{pgfplots}
  \pgfplotsset{compat=1.18}
\usetikzlibrary{arrows}
\newcommand{\midarrow}{\tikz \draw[-triangle 90] (0,0) -- +(.1,0);}

%%%% FIGURES %%%%
\usepackage{subcaption}
\usepackage{wrapfig}
\usepackage{float}
\usepackage[skip=5pt, font=footnotesize]{caption}

%%%% FORMATTING %%%%
\usepackage{parskip}
\usepackage{tcolorbox}
\usepackage{ulem}

%%%% TABLE FORMATTING %%%%
\usepackage{tabularray}
\UseTblrLibrary{booktabs}

%%%% MATH AND LOGIC %%%%
\usepackage{xifthen}
\usepackage{amsmath}
\usepackage{amssymb}
\usepackage{amsfonts}

%%%% TEXT AND SYMBOLS %%%%
\usepackage[T1]{fontenc}
\usepackage{textcomp}
\usepackage{gensymb}

%%%% OTHER %%%%
\usepackage{standalone}

%%%% LOGIC SYMBOLS %%%%
\newcommand*\xor{\oplus}

%%%% STYLES %%%%

% Packages
\usepackage[paper=letterpaper,tmargin=45pt,bmargin=45pt,lmargin=45pt,rmargin=45pt]{geometry}
\usepackage{titlesec}
\usepackage[rgb]{xcolor}
\selectcolormodel{natural}
\usepackage{ninecolors}
\selectcolormodel{rgb}

% Colors
\definecolor{pg}{HTML}{24273A}
\definecolor{fg}{HTML}{FFFFFF}
\definecolor{bg}{HTML}{24273A}
\definecolor{re}{HTML}{F38BA8}
\definecolor{gr}{HTML}{A6E3A1}
\definecolor{ye}{HTML}{F9E2AF}
\definecolor{or}{HTML}{FAB387}
\definecolor{bl}{HTML}{89B4FA}
\definecolor{ma}{HTML}{CBA6F7}
\definecolor{cy}{HTML}{94E2D5}
\definecolor{pi}{HTML}{F2CDCD}

\definecolor{copper}{HTML}{B87333}

\usepackage{nameref}
\makeatletter
\newcommand*{\currentname}{\@currentlabelname}
\makeatother

\titleformat{\section}
  {\normalfont\scshape\Large\bfseries}
  {\thesection}
  {0.75em}
  {}

\titleformat{\subsection}
  {\normalfont\scshape\large\bfseries}
  {\thesubsection}
  {0.75em}
  {}

\titleformat{\subsubsection}
  {\normalfont\scshape\normalsize\bfseries}
  {\thesubsubsection}
  {0.75em}
  {}

% Formula
\newcounter{formula}[section]
\newenvironment{formula}[1]{
  \stepcounter{formula}
  \begin{tcolorbox}[
    standard jigsaw, % Allows opacity
    colframe={fg},
    boxrule=1px,
    colback=bg,
    opacityback=0,
    sharp corners,
    sidebyside,
    righthand width=18px,
    coltext={fg}
  ]
  \centering
  \textbf{\uline{#1}}
}{
  \tcblower
  \textbf{\thesection.\theformula}
  \end{tcolorbox}
}

% Definition
\newcounter{definition}[section]

\newenvironment{definition*}[1]{
  \begin{tcolorbox}[
    standard jigsaw, % Allows opacity
    colframe={fg},
    boxrule=1px,
    colback=bg,
    opacityback=0,
    sharp corners,
    coltext={fg}
  ]
  \textbf{#1 \hfill}
  \vspace{5px}
  \hrule
  \vspace{5px}
  \noindent
}{
  \end{tcolorbox}
}

\newenvironment{definition}[1]{
  \stepcounter{definition}
  \begin{tcolorbox}[
    standard jigsaw, % Allows opacity
    colframe={fg},
    boxrule=1px,
    colback=bg,
    opacityback=0,
    sharp corners,
    coltext={fg}
  ]
  \textbf{#1 \hfill \thesection.\thedefinition}
  \vspace{5px}
  \hrule
  \vspace{5px}
  \noindent
}{
  \end{tcolorbox}
}

% Example Problem
\newcounter{example}[section]
\newenvironment{example}{
  \stepcounter{example}
  \begin{tcolorbox}[
    standard jigsaw, % Allows opacity
    colframe={fg},
    boxrule=1px,
    colback=bg,
    opacityback=0,
    sharp corners,
    coltext={fg}
  ]
  \textbf{Example \hfill \thesection.\theexample}
  \vspace{5px}
  \hrule
  \vspace{5px}
  \noindent
}{
  \end{tcolorbox}
}

\tikzset{
  cubeBorder/.style=fg,
  cubeFilling/.style={fg!20!bg, opacity=0.25},
  gridLine/.style={very thin, gray},
  graphLine/.style={-latex, thick, fg},
}

\pgfplotsset{
  basicAxis/.style={
    grid,
    major grid style={line width=.2pt,draw=fg!50!bg},
    axis lines = box,
    axis line style = {line width = 1px},
  }
}

%%%% REFERENCES %%%%
\usepackage{hyperref}
\hypersetup{
  colorlinks  = true,
  linkcolor   = bl,
  anchorcolor = bl,
  citecolor   = bl,
  filecolor   = bl,
  menucolor   = bl,
  runcolor    = bl,
  urlcolor    = bl,
}

\author{Ethan Anthony}
\newcommand*{\equal}{=}


\title{Lecture 003}
\date{January 10, 2025}

\begin{document}
\newpage

\section{Deformation}
\label{sec:deformation}

\subsection{Normal Strain}
\label{ssec:normalStrain}

When studying the mechanics of materials, analyzing them entirely in static equilibrium would yield little to no information about how a material changes when ample load is applied. Consider the beam in Figure \ref{fig:014}. After having endured an axial load, the beam deforms be elongating.

\begin{wrapfigure}[6]{r}{0.25\textwidth}
  \centering
  \includestandalone{figures/fig_014}
  \caption{Deformation Due to Axial Stress}
  \label{fig:014}
\end{wrapfigure}

In this case, the elongation undergone by the beam is considered the \textbf{strain} endured by the beam. In other situations, strain can look different, such as the bending or compression of a material.

\begin{definition}{Strain}
  Deformation experienced by an object as the result of stresses exceeding the materials ability to maintain shape.
\end{definition}

In the case of Figure \ref{fig:014}, the strain is specifically \textbf{normal strain}.

\begin{definition}{Normal Strain}
  Strain specifically relating to axial/normal stress. It is specifically defined as the \textit{deformation per unit length}. Normal strain is denoted by the Greek letter epsilon ($\epsilon$).
\end{definition}

Since normal strain is the deformation per unit length, to find the value of $\epsilon$, one must divide the deformation by the \textit{original} length of the member.

\begin{formula}{Normal Strain}
  \begin{equation*}
    \epsilon = \frac{\delta}{L}
  \end{equation*}
\end{formula}

Though indirectly caused by an axial load, the factors in determining the amount of normal strain experienced by a material are: the materials inherent properties and the axial stress on the member.

\begin{figure}[H]
  \centering
  \begin{subfigure}[h]{0.48\textwidth}
    \centering
    \includestandalone{figures/fig_015}
  \end{subfigure}
  \vrule
  \begin{subfigure}[h]{0.48\textwidth}
    \centering
    \includestandalone{figures/fig_016}
  \end{subfigure}
  \caption{Strain For Two Rods of Different Cross Sections}
  \label{fig:015}
\end{figure}

Assuming the same material, two members of different dimensions, but undergoing the same stress, will strain the same amount. In Figure \ref{fig:015}, this exact situation is visualized. Since stress is just a ratio of force to area, a member double the area experiencing an axial load double of another member will strain the same.

\begin{figure}[H]
  \centering
  \begin{subfigure}[H]{\textwidth}
    \centering
    \includestandalone{figures/fig_017}
  \end{subfigure}
  \begin{subfigure}[H]{\textwidth}
    \centering
    \includestandalone{figures/fig_018}
  \end{subfigure}
  \caption{Strain For Two Rods of Different Lengths}
  \label{fig:strainForTwoRodsOfDifferentLengths}
\end{figure}

Similarly, since strain is a ratio between the original length and the final length, a body double the length of another will experience the same strain assuming all else is the same. See Figure \ref{fig:strainForTwoRodsOfDifferentLengths}.

Strain so far has only been considered for uniform bodies: the same material, same cross-section, etc, and only for members loaded axially on their ends. What happens when these conditions aren't met? Conveniently, strain can be added for subsections of a member, and summed together to find the total strain:
\begin{equation*}
  \delta = \sum_{i}^{} \frac{P_iL_i}{A_iE_i}
\end{equation*}

\subsection{Stress Strain Diagrams}
\label{ssec:stressStrainDiagrams}

Having seen that strain is independent of the dimensions of a material, and rather depends on the stress experienced by the material, it follows that plotting strain as a function of stress would result in a diagram generally applicable to a specific material. This curve, called a \textbf{stress-strain diagram} characterizes the properties of a material.

Though all materials behave differently when examined through a stress-strain diagram, there are two broad categories of materials: \textit{brittle} and \textit{ductile}.

\textbf{Ductile materials} are able to yield without necessarily failing entirely. Their elongation initially increases linearly with stress until a some value $\sigma_Y$ where suddenly undergoes a large deformation with a relatively small increase in stress.

\begin{formula}{Measuring Ductility}
  \begin{equation*}
    \textup{Percent Elongation} = 100 \cdot \frac{\textup{Initial Length} - \textup{Length at Failure}}{\textup{Initial Length}} = 100 \cdot \frac{L_B - L_0}{L_0}
  \end{equation*}
  \begin{equation*}
    \textup{Percent Reduction in Area} = 100 \cdot \frac{\textup{Initial Area} - \textup{Area at Failure}}{\textup{Initial Area}} = 100 \cdot \frac{A_0 - A_B}{A_0}
  \end{equation*}
\end{formula}

\textbf{Brittle materials}, on the other hand, experience a very small amount of yield, after which they tend to fail suddenly. There is a distinct lack of necking in brittle materials.

\subsection{True vs. Engineering Stress and Strain}
\label{ssec:trueVsEngineeringStressAndStrain}

When an object undergoes elongation, the cross sectional area of that object will change. Thus introduces a question: should the original or current cross-section be used as the area when calculating stress? Rather than having an answer, there are just two types of stress which each reflect one of the options: \textbf{engineering stress} and \textbf{true stress}.

\begin{definition}{True vs. Engineering Stress}
  \textbf{Engineering stress} is stress experienced by an object calculated using the original cross-section of the body.
  \begin{equation*}
  \sigma_e = \frac{P}{A_{initial}}
  \end{equation*}
  \textbf{True stress} is stress as calculated by using the current cross-section of the body.
  \begin{equation*}
    \sigma_t = \frac{P}{A_{current}}
  \end{equation*}
\end{definition}

A similar situation occurs when measuring the \textit{strain} of a body. Using the formula for strain as introduced in \ref{ssec:normalStrain} produces the \textbf{engineering strain} experienced by a body. However, if rather than measuring a single value for the length and deformation of the body, one were to subdivide the body into several subsections and measure the deformation for each subsection, something closer to the \textbf{true strain} would be produced. This is seen in Figure \ref{fig:019}.

\begin{figure}[H]
  \centering
  \begin{subfigure}[H]{0.45\textwidth}
    \centering
    \includestandalone{figures/fig_020}
    \caption{Engineering Strain}
    \label{fig:020}
  \end{subfigure}
  \begin{subfigure}[H]{0.45\textwidth}
    \centering
    \includestandalone{figures/fig_019}
    \caption{True(ish) Strain}
    \label{fig:019}
  \end{subfigure}
  \caption{Engineering vs. True Strain}
  \label{fig:engineeringVsTrueStrain}
\end{figure}

Taking this subdividing to its limit, the \textbf{True Strain} will be a perfectly continuous measurement of the change in deformation over some change in length:
\begin{equation*}
  \epsilon_{true} = \int_{L_0}^{L} \frac{dL}{L} \, = \ln \frac{L}{L_0}
\end{equation*}

\subsection{Hooke's Law and the Modulus of Elasticity}
\label{ssec:HookesLawAndTheModulusOfElasticity}

Generally, structures are designed to keep any deformations within the linear portion of the stress-strain diagram. Given that some body is kept within that range, the stress ($\sigma$) is directly proportional to the strain ($\epsilon$):
\begin{equation*}
  \sigma = E \epsilon
\end{equation*}
Where $E$ is the \textbf{modulus of elasticity} of the material.
\begin{definition}{Modulus of Elasticity}
  An inherent value to a material that reflects the proportionality between the stress and strain experienced by a body.
\end{definition}
This linear relationship, governed by a constant coefficient ($E$) is known as \textbf{Hooke's Law}. This law only applies until the material reaches its \textbf{proportional limit}.
\begin{definition}{Proportional Limit}
  The upper bound of the range over which Hooke's Law applies to some material.
\end{definition}

\subsection{Repeated Loadings and Fatigue}
\label{ssec:repeatedLoadingsAndFatigue}

So far, only single instances of loading have been considered for the materials. What happens after thousands, or millions, of loads have been repeatedly applied to a material? In these cases, the material is said to have undergone \textbf{fatigue}.
\begin{definition}{Fatigue}
  The result of many repeated loadings, fatigue is a general term for the effect that these repeated loadings have on a material.
\end{definition}
Failures due to fatigue are generally more brittle than they are ductile. Since fatigue not only increases the chances of a material failing, but also makes it such that there is less of a yield period, it is imperative to consider fatigue when designing structures and machines that are expected to support time-varying loads.

To measure the fatigue of a material, a stress to loading curve can be made. This measures the number of repeated loads required to cause failure at different load-sizes for a given material. A generic one can be seen in Figure \ref{fig:021}, illustrating the typical shape of such a curve.

\begin{figure}[H]
  \centering
  \includestandalone{figures/fig_021}
  \caption{Stress to Loading Curve}
  \label{fig:021}
\end{figure}

The flat part at the end of the blue material in Figure \ref{fig:021} marks the \textbf{endurance limit} of the material. This is where, regardless of the number of time a load is applied, the material will never fail. Any force value below the endurance limit on the $y$-axis (stress) will never cause a material to fail.

However, for non-ferrous materials, there is no endurance limit. Rather, the curve just continues to go down. This is seen in the red curve in Figure \ref{fig:021}. This behavior is caused by microscopic failure after each consecutive loading, eventually compounding into total failure.

\subsection{Static Indeterminacy}
\label{ssec:staticIndeterminacy}

Thus far, only objects in static equilibrium have been considered. However, situations in which statics alone can't determine internal forces are very common. In such cases, the geometry must be used alongside equilibrium equations. These cases are referred to as \textbf{statically indeterminate}.

\begin{definition}{Static Indeterminate}
  Statically indeterminate structures are those which have reactions and internal forces that cannot be found through statics alone, and thus must derive results based on the material composition of the structure.
\end{definition}

Consider the structure on Figure \ref{fig:023}. In this body, there is a force being axially applied, but not at the ends of the structure. It is known that the structure is in static equilibrium:
\begin{equation*}
  R_A+R_B=P
\end{equation*}
However, there are two unknowns in this equation: $R_A$ and $R_B$. In order to solve for these unknowns, it is necessary, then, to use a compatibility equation. However, before understanding that, it is important to understand the \textbf{Principal of Superposition}.
\begin{definition}{Principal of Superposition}
  In materials governed by Hooke's Law, the deformation of a structure is equal to the deformation of any number of arbitrary sections of that structure:
  \begin{equation*}
    \delta_{total} = \sum_{n}^{N} \delta_n
  \end{equation*}
\end{definition}

\begin{figure}[H]
  \centering
  \includestandalone{figures/fig_023}
  \caption{Basic Example}
  \label{fig:023}
\end{figure}

By the principal of superposition, one is able to decompose a structure into component parts, solve for the deformation in each part, then sum the deformations to find the total deformation of the structure. 

Applying the Principal of Superposition, it is possible to relate the deformation of the structure based on the constraints. This process results in a compatibility equation.

\begin{figure}[H]
  \centering
  \includestandalone{figures/fig_024}
  \caption{Sections of Figure \ref{fig:023}}
  \label{fig:024}
\end{figure}

Decomposing the structure (as seen in Figure \ref{fig:024}), the deformation of each individual component can be analyzed independently, and then be compared through a \textbf{compatibility equation} to determine the true values of each portion:
\begin{equation*}
  \delta_{top} + \delta_{bottom} = 0
\end{equation*}
This can then be expanded out to:
\begin{equation*}
  \frac{P_tL_t}{A_tE_t} = \frac{P_bL_b}{A_bE_b}
\end{equation*}
Assuming the same material throughout as well as the same cross-section, this can be taken further by canceling out the area ($A$) and the modulus of elasticity ($E$) of each side, resulting in:
\begin{align*}
  P_tL_t &= P_bL_b \\
  \frac{L_t}{L_b}\cdot P_t &= P_b
\end{align*}
In situations like these, it can be clearly seen that the load on each component is directly proportional to the length of that component.

\subsection{Poisson's Ratio}
\label{ssec:poissonsRatio}

In all engineering materials, the relationship between the elongation of body due to axial stress is directly proportional to contraction in any transverse direction. This relationship is codified with a constant that relates the contraction with the elongation, this constant being a given material's \textbf{Poisson's Ratio}.
\begin{formula}{Poisson's Ratio}
  \begin{equation*}
    \nu = -\frac{\textup{lateral strain}}{\textup{axial strain}} = -\frac{\epsilon_y}{\epsilon_x}
  \end{equation*}
\end{formula}

\subsection{Stress Concentrations}
\label{ssec:stressConcentrations}

Consider Figure \ref{fig:stressDistributionNearDiscontinuity}. Due to the discontinuity in the center, the stress is distributed differently than normal. Considering the location with the smallest cross section (the exact center), the stress is distributed closer to the center of the discontinuity.

\begin{figure}[H]
  \centering
  \begin{subfigure}[H]{0.48\textwidth}
    \centering
    \includestandalone{figures/fig_025}
  \end{subfigure}
  \begin{subfigure}[H]{0.48\textwidth}
    \centering
    \includestandalone{figures/fig_026}
  \end{subfigure}
  \caption{Stress Distribution Near Discontinuity}
  \label{fig:stressDistributionNearDiscontinuity}
\end{figure}

Through rigorous experimentation, it's been determined that analyzing these structures can be done solely based on the geometry of the discontinuities. The main concern of this analysis is to find whether the allowable stress will be exceeded. Since the maximum stress ($\sigma_{max}$) will be at the narrowest cross-section, only this cross section needs to be analyzed.

To complete this analysis, an engineer must find the average stress in the narrowest cross-section:
\begin{equation*}
  \sigma_{avg} = \frac{P}{A}
\end{equation*}
and the ratio between the average and maximum stress at that cross-section:
\begin{equation*}
  K = \frac{\sigma_{max}}{\sigma_{avg}}
\end{equation*}
This ratio is referred to as the \textbf{stress concentration factor}. As claimed earlier, this stress concentration factor can be computed based on solely the geometry of the discontinuity. More specifically, there exist \href{https://research.iaun.ac.ir/pd/jjfesharaki/pdfs/UploadFile_9038.pdf}{repositories of information} on how to plot the ratio of the radius and min height ($\frac{r}{h}$) and the ratio between the max and min heights ($\frac{H}{h}$)

\subsection{Thermal Strain}
\label{ssec:thermalStrain}

When changes in shape due to changes in temperature occur, new stresses are created due to these thermal deformations. Generally, these are somewhat small. However, with large bodies or large changes in temperature these thermal strains create large enough stresses to cause a material to rupture.

\begin{figure}[H]
  \centering
  \begin{subfigure}[H]{0.45\textwidth}
    \centering
    \includestandalone{figures/fig_028}
    \caption{Initial Temperature $T_i$}
    \label{fig:028}
  \end{subfigure}
  \begin{subfigure}[H]{0.45\textwidth}
    \centering
    \includestandalone{figures/fig_029}
    \caption{Final Temperature $T_f > T_i$}
    \label{fig:029}
  \end{subfigure}
  \caption{Expansions Due to Increased Temperature}
  \label{fig:expansionsDueToIncreasedTemperature}
\end{figure}

\begin{formula}{Thermal Strain}
  \begin{equation*}
    \epsilon_{thermal} = \alpha \cdot \Delta T
  \end{equation*}
  \begin{tblr}{c|c}
    \midrule
    $\alpha = $ coefficient of thermal expansion & $\Delta T = $ change in temperature \\
  \end{tblr}
\end{formula}
Notice that there is no stress associated with the thermal strain. The strain is directly caused by the change in temperature of the material as well as the material's inherent properties.

Similar to the axial elongation, the change in length of a body can be defined by multiplying the thermal strain ($\epsilon_T$) by the original length of the member ($L$).
\begin{formula}{Change in Length}
  \begin{equation*}
    \delta_{thermal} = \epsilon_{T} \cdot L = \alpha \cdot \Delta T \cdot L
  \end{equation*}
\end{formula}

\begin{wrapfigure}[]{l}{0.25\textwidth}
  \centering
  \includestandalone{figures/fig_027}
  \caption{Constrained Body}
  \label{fig:027}
\end{wrapfigure}

These two formulas model the thermal strain/thermal expansion undergone by unconstrained bodies. However, how do constrained bodies (such as the one in Figure \ref{fig:027}) react to changes in temperature?

In these situations, a \textbf{thermal stress} is induced into this bar. This thermal stress can be calculated in a similar way to statically indeterminate structures (Subsection \ref{ssec:staticIndeterminacy}).

The deformation due to the thermal expansion and the deformation due to the constraints can be summed to the total deformation experienced. In the example of Figure \ref{fig:027}, since the total deformation is zero, the compatibility equation is:
\begin{align*}
  \delta_{thermal} + \delta_{constraint} &= 0 \\
  \alpha \cdot \Delta T \cdot L &= - \delta_C \\
  \alpha \cdot \Delta T \cdot L &= -\frac{P_CL}{AE}
\end{align*}
Then using the values known about the member and its composition, the force applied by the constraints can be found, leading directly to then being able to solve for the stress. This compatibility equation can be seen in steps in Figure \ref{fig:constrainedThermalExpansion}.

\begin{figure}[H]
  \centering
  \begin{subfigure}[t!]{0.3\textwidth}
    \centering
    \includestandalone{figures/fig_030}
    \caption{Initial Body}
    \label{fig:030}
  \end{subfigure}
  \begin{subfigure}[t!]{0.3\textwidth}
    \centering
    \includestandalone{figures/fig_031}
    \caption{Thermal Expansion}
    \label{fig:031}
  \end{subfigure}
  \begin{subfigure}[t!]{0.3\textwidth}
    \centering
    \includestandalone{figures/fig_032}
    \caption{Constraint}
    \label{fig:032}
  \end{subfigure}
  \caption{Constrained Thermal Expansion}
  \label{fig:constrainedThermalExpansion}
\end{figure}

\end{document}
