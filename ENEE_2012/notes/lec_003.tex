\documentclass[12pt]{article}

\input{../../xlatex/imports/preamble}

\title{Lecture 002}
\date{January 21, 2025}

\begin{document}
\newpage

\section{Basic Laws}
\label{sec:basicLaws}

\subsection{Ohm's Law}
\label{ssec:ohmsLaw}

\textbf{Ohm's Law} states that the voltage $V$ is \textit{directly} proportional to the current $I$ and resistance $R$ of a circuit.

\begin{formula}{Ohm's Law}
  \begin{equation*}
    V = IR
  \end{equation*}
\end{formula}

When there is current flowing through a wire with resistance approaching zero, a \textbf{short circuit} is created. Conversely, an \textbf{open circuit} is where the resistance in a circuit approaches infinity.

Whereas resistance measures how much something impedes the flow of current, \textbf{conductance} is the ability of an element of conduct electric current; it is measured in mhos ($\mho$) or siemens ($S$).

\subsubsection{Resistivity}
\label{sssec:resistivity}

\begin{wrapfigure}[]{r}{0.3\textwidth}
  \vspace{-45pt}
  \centering
  \includestandalone{figures/fig_023}
  \caption{Resistivity in a Wire}
  \label{fig:023}
  \vspace{-10pt}
\end{wrapfigure}

The resistance $\si{R}$ of a wire is a value determined by a wires dimensions as well as the properties of the material the wire is made of. The inherent properties of a material as they relate to resistance are codified as a material's \textbf{resistivity}, which is a quantification of how much a material impedes charge passing through it. The resistivity of some common materials can be seen in Table \ref{tbl:resistivitiesOfCommonMaterials}.

\begin{definition}{Resistivity ($\rho$)}
  A value representing the amount a material conducts electricity. Most metals tend to have high conductivity, while rubber has a low conductivity.
  \begin{equation*}
    R = \rho \cdot \frac{l}{A}
  \end{equation*}
\end{definition}

\begin{figure}[H]
  \centering
  \begin{tblr}{ccc}
    \toprule
    \textbf{Material} & \textbf{Resistivity} ($\ohm \cdot m$) & \textbf{Usage} \\
    \midrule
    Silver    & $1.64 \cdot 10^{-8}$ & Conductor \\
    Copper    & $1.72 \cdot 10^{-8}$ & Conductor \\
    Aluminum  & $2.80 \cdot 10^{-8}$ & Conductor \\
    Carbon    & $4    \cdot 10^{-5}$ & Semiconductor \\
    Germanium & $47   \cdot 10^{-2}$ & Semiconductor \\
    Silicon   & $6.4  \cdot 10^{2 }$ & Semiconductor \\
    Paper     & $           10^{10}$ & Insulator \\
    Glass     & $10   \cdot 10^{12}$ & Insulator \\
    Teflon    & $3    \cdot 10^{12}$ & Insulator \\
    \bottomrule
  \end{tblr}
  \caption{Resistivities of Common Materials}
  \label{tbl:resistivitiesOfCommonMaterials}
\end{figure}

\subsection{Nodes, Branches, and Loops}
\label{ssec:nodesBranchesLoops}

\begin{definition}{Branch $(b)$}
  Represents a single element in a circuit such as a resistor or power supply.
\end{definition}

\begin{wrapfigure}[7]{l}{0.35\textwidth}
  \centering
  \includestandalone{figures/fig_007}
  \caption{Nodes and Branches}
  \label{fig:007}
\end{wrapfigure}

In Figure \ref{fig:007}, there exist \textit{five} branches in the circuit. Specifically, there are three resistors, one voltage source, and one current source:
\begin{equation*}
  3 + 1 + 1 = 5
\end{equation*}

\begin{definition}{Node ($n$)}
  The point of connection between two or more branches.
\end{definition}

In the same circuit (\ref{fig:007}), there are three nodes. Each node is highlighted in a different color. Notice that there are no branches within a node, thus each node is the largest area possible without crossing a branch. Furthermore, the voltage throughout an ideal node is zero.

\begin{definition}{Loop ($l$)}
  A loop is any \textit{closed} path in a circuit. Generally, loops are defined as the smallest possible path.
\end{definition}

\begin{figure}[H]
  \centering
  \includestandalone{figures/fig_008}
  \caption{Loops}
  \label{fig:008}
\end{figure}
\vspace{-10pt}
Each shown in a different color in Figure \ref{fig:008}, loops are easy to visualize as the area enclosed by any closed series of components.

\begin{formula}{Network Topology}
  \begin{equation*}
    b = l + n - 1
  \end{equation*}
  \vspace{-17pt}
\end{formula}

\subsection{Parallel vs. Series}
\label{ssec:parallelVsSeries}

\begin{definition}{Series}
  Two or more elements are in series if they exclusively share a single node and consequently carry the same current.
\end{definition}

Continuing with the same circuit, Figure \ref{fig:009} highlights elements in series. The only two elements in series are the voltage source and the top resistor.

Figure \ref{fig:010} highlights the two shared nodes between three of the branches in the circuit. Since these branches are sharing the same two nodes, they are said to be in parallel.

\begin{figure}[H]
  \centering
  \begin{subfigure}[H]{0.45\textwidth}
    \centering
    \includestandalone{figures/fig_009}
    \caption{Loops}
    \label{fig:009}
  \end{subfigure}
  \begin{subfigure}[H]{0.45\textwidth}
    \centering
    \includestandalone{figures/fig_010}
    \caption{Loops}
    \label{fig:010}
  \end{subfigure}
  \caption{In Parallel and In Series}
  \label{fig:inParallelAndInSeries}
\end{figure}

\begin{definition}{Parallel}
  Two or more elements are in parallel if they are connected to the same two notes and consequently have the same voltage across them.
\end{definition}

\subsection{Kirchhoff's Laws}
\label{ssec:kirchoffsLaws}

\subsubsection{Kirchoff's Current Law}
\label{sssec:kirchoffsCurrentLaw}

\begin{definition}{Kirchhoff's Current Law (KCL)}
  Kirchhoff's Current Law states that the algebraic sum of currents entering and exiting a node is zero:
  \begin{equation*}
    \sum_{n=1}^{N_{branch}} i_n = 0
  \end{equation*}
\end{definition}

\begin{wrapfigure}[]{r}{0.5\textwidth}
  \vspace{-15pt}
  \centering
  \includestandalone{figures/fig_026}
  \caption{KCL}
  \label{fig:026}
\end{wrapfigure}

Consider the circuit in Figure \ref{fig:026}. In this circuit, there are two labeled nodes $n_1$ and $n_2$. Applying KCL at $n_1$, the sum of currents would be expressed as:
\begin{align*}
  I_1 - I_2 - I_3 - I_a = 0
\end{align*}
Since only $I_1$ is flowing \textit{into} $n_1$, with the other tree currents flowing out. At $n_2$, the currents would be:
\begin{equation*}
  -I_1 + I_2 + I_3 + I_a = 0
\end{equation*}
Since the opposite is true: only $I_1$ is flowing \textit{out} of $n_2$, with the rest flowing in.

\subsubsection{Kirchoff's Voltage Law}
\label{sssec:kirchoffsVoltageLaw}

\begin{definition}{Kirchhoff's Voltage Law (KVL)}
  Kirchhoff's Voltage Law states that the algebraic sum of all voltages around a closed path (loop) is zero:
  \begin{equation*}
    \sum_{m=1}^{M_{branch}} v_m = 0
  \end{equation*}
\end{definition}

\begin{wrapfigure}[11]{l}{0.3\textwidth}
  \vspace{-25pt}
  \centering
  \includestandalone{figures/fig_027}
  \caption{KVL}
  \label{fig:027}
\end{wrapfigure}

Consider the circuit in Figure \ref{fig:027}. In this circuit, there is a single loop. According to KVL, the voltage gained and lost through each branch of this loop should sum to zero.

Following the direction of $I$ in this circuit, there is an \textit{increase} in voltage over the voltage source and two consecutive losses in voltage over the resistors. Thus:
\begin{equation*}
  -V_a + V_1 + V_2 = 0\ \ \ \textup{or}\ \ \ V_a = V_1 + V_2
\end{equation*}
This balance of the voltage rises across some components in a loop and the voltage drops across other components is what KVL is.

\subsection{Voltage Divider}
\label{ssec:voltageDivider}

When resistors are connected in series, the equivalent resistance of the resistors is equal to the sum of the individual resistances:
\begin{equation*}
  R_{eq} = R_1 + R_2 + R_3 + \hdots + R_n = \sum_{n=1}^{N} R_n
\end{equation*}

\begin{figure}[H]
  \vspace{-20pt}
  \centering
  \includestandalone{figures/fig_028}
  \caption{Resistors in Series}
  \label{fig:resistorsInSeries}
\end{figure}

By applying Ohm's Law to this concept of adding resistances together, the total current of the circuit can be found as:
\begin{equation*}
  I = \frac{V}{R_1+R_2+R_3} = \frac{V}{R_{eq}}
\end{equation*}
The to find the voltage drop across any single resistor, Ohm's Law can be applied again with the current $I$:
\begin{equation*}
  V_n = I \cdot R_n = V \cdot \frac{R_n}{R_{eq}}
\end{equation*}
Thus giving the formula for a voltage divider. This voltage divider can only be applied to elements that have the same current across them all, such as these resistors \textit{in series}.
\begin{formula}{Voltage Divider}
  \begin{equation*}
    V_n = V \cdot \frac{R_n}{R_{eq}}
  \end{equation*}
\end{formula}

\newpage
\subsection{Current Divider}
\label{ssec:currentDivider}

\begin{wrapfigure}[]{r}{0.3\textwidth}
  \vspace{-20pt}
  \centering
  \includestandalone{figures/fig_029}
  \caption{Resistors in Parallel}
  \label{fig:resistorsInParallel}
\end{wrapfigure}

When resistors are connected in parallel, the equivalent resistance of the resistors is calculated as:
\begin{equation*}
  R_{eq} = \left(\frac{1}{R_1} + \frac{1}{R_2} + \frac{1}{R_3} + \hdots + \frac{1}{R_n}\right)^{-1} = \left( \sum_{n=1}^{N} \frac{1}{R_n} \right)^{-1}
\end{equation*}

By applying Ohm's Law to this concept resistors in parallel, the total current of the circuit can be found as:
\begin{equation*}
  I_t = \frac{V}{\left(\frac{1}{R_1} + \frac{1}{R_2} + \frac{1}{R_3} + \hdots + \frac{1}{R_n}\right)^{-1}} = V \cdot R_{eq}
\end{equation*}
Then to find the current across any single resistor, Ohm's Law can be applied again with the total current $I_t$:
\begin{equation*}
  I_n = I_t \cdot \frac{R_{eq}}{R_n}
\end{equation*}
Thus giving the formula for a current divider. This current divider can only be applied to elements in parallel (elements that share both nodes with each other).
\begin{formula}{Current Divider}
  \begin{equation*}
    I_n = I_t \cdot \frac{R_{eq}}{R_n}
  \end{equation*}
\end{formula}

\end{document}
