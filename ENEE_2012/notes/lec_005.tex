\documentclass[12pt]{article}

%%%% GRAPHICS %%%%
\usepackage{tikz}
\usepackage[siunitx, american, RPvoltages]{circuitikz}
\usetikzlibrary{arrows.meta}
\usepackage{tikz-3dplot}
\usepackage{graphicx}
\usepackage{pgfplots}
  \pgfplotsset{compat=1.18}
\usetikzlibrary{arrows}
\newcommand{\midarrow}{\tikz \draw[-triangle 90] (0,0) -- +(.1,0);}

%%%% FIGURES %%%%
\usepackage{subcaption}
\usepackage{wrapfig}
\usepackage{float}
\usepackage[skip=5pt, font=footnotesize]{caption}

%%%% FORMATTING %%%%
\usepackage{parskip}
\usepackage{tcolorbox}
\usepackage{ulem}

%%%% TABLE FORMATTING %%%%
\usepackage{tabularray}
\UseTblrLibrary{booktabs}

%%%% MATH AND LOGIC %%%%
\usepackage{xifthen}
\usepackage{amsmath}
\usepackage{amssymb}
\usepackage{amsfonts}

%%%% TEXT AND SYMBOLS %%%%
\usepackage[T1]{fontenc}
\usepackage{textcomp}
\usepackage{gensymb}

%%%% OTHER %%%%
\usepackage{standalone}

%%%% LOGIC SYMBOLS %%%%
\newcommand*\xor{\oplus}

%%%% STYLES %%%%

% Packages
\usepackage[paper=letterpaper,tmargin=45pt,bmargin=45pt,lmargin=45pt,rmargin=45pt]{geometry}
\usepackage{titlesec}
\usepackage[rgb]{xcolor}
\selectcolormodel{natural}
\usepackage{ninecolors}
\selectcolormodel{rgb}

% Colors
\definecolor{pg}{HTML}{24273A}
\definecolor{fg}{HTML}{FFFFFF}
\definecolor{bg}{HTML}{24273A}
\definecolor{re}{HTML}{F38BA8}
\definecolor{gr}{HTML}{A6E3A1}
\definecolor{ye}{HTML}{F9E2AF}
\definecolor{or}{HTML}{FAB387}
\definecolor{bl}{HTML}{89B4FA}
\definecolor{ma}{HTML}{CBA6F7}
\definecolor{cy}{HTML}{94E2D5}
\definecolor{pi}{HTML}{F2CDCD}

\definecolor{copper}{HTML}{B87333}

\usepackage{nameref}
\makeatletter
\newcommand*{\currentname}{\@currentlabelname}
\makeatother

\titleformat{\section}
  {\normalfont\scshape\Large\bfseries}
  {\thesection}
  {0.75em}
  {}

\titleformat{\subsection}
  {\normalfont\scshape\large\bfseries}
  {\thesubsection}
  {0.75em}
  {}

\titleformat{\subsubsection}
  {\normalfont\scshape\normalsize\bfseries}
  {\thesubsubsection}
  {0.75em}
  {}

% Formula
\newcounter{formula}[section]
\newenvironment{formula}[1]{
  \stepcounter{formula}
  \begin{tcolorbox}[
    standard jigsaw, % Allows opacity
    colframe={fg},
    boxrule=1px,
    colback=bg,
    opacityback=0,
    sharp corners,
    sidebyside,
    righthand width=18px,
    coltext={fg}
  ]
  \centering
  \textbf{\uline{#1}}
}{
  \tcblower
  \textbf{\thesection.\theformula}
  \end{tcolorbox}
}

% Definition
\newcounter{definition}[section]

\newenvironment{definition*}[1]{
  \begin{tcolorbox}[
    standard jigsaw, % Allows opacity
    colframe={fg},
    boxrule=1px,
    colback=bg,
    opacityback=0,
    sharp corners,
    coltext={fg}
  ]
  \textbf{#1 \hfill}
  \vspace{5px}
  \hrule
  \vspace{5px}
  \noindent
}{
  \end{tcolorbox}
}

\newenvironment{definition}[1]{
  \stepcounter{definition}
  \begin{tcolorbox}[
    standard jigsaw, % Allows opacity
    colframe={fg},
    boxrule=1px,
    colback=bg,
    opacityback=0,
    sharp corners,
    coltext={fg}
  ]
  \textbf{#1 \hfill \thesection.\thedefinition}
  \vspace{5px}
  \hrule
  \vspace{5px}
  \noindent
}{
  \end{tcolorbox}
}

% Example Problem
\newcounter{example}[section]
\newenvironment{example}{
  \stepcounter{example}
  \begin{tcolorbox}[
    standard jigsaw, % Allows opacity
    colframe={fg},
    boxrule=1px,
    colback=bg,
    opacityback=0,
    sharp corners,
    coltext={fg}
  ]
  \textbf{Example \hfill \thesection.\theexample}
  \vspace{5px}
  \hrule
  \vspace{5px}
  \noindent
}{
  \end{tcolorbox}
}

\tikzset{
  cubeBorder/.style=fg,
  cubeFilling/.style={fg!20!bg, opacity=0.25},
  gridLine/.style={very thin, gray},
  graphLine/.style={-latex, thick, fg},
}

\pgfplotsset{
  basicAxis/.style={
    grid,
    major grid style={line width=.2pt,draw=fg!50!bg},
    axis lines = box,
    axis line style = {line width = 1px},
  }
}

%%%% REFERENCES %%%%
\usepackage{hyperref}
\hypersetup{
  colorlinks  = true,
  linkcolor   = bl,
  anchorcolor = bl,
  citecolor   = bl,
  filecolor   = bl,
  menucolor   = bl,
  runcolor    = bl,
  urlcolor    = bl,
}

\author{Ethan Anthony}
\newcommand*{\equal}{=}


\title{Lecture 004}
\date{February 03, 2025}

\begin{document}

\newpage
\section{Circuit Theorems}
\label{sec:circuitTheorems}

% \subsection{Linearity}
% \label{ssec:linearity}
%
% \begin{definition}{Linear Circuit}
%   A linear circuit is one whose output is linearly related to its input.
% \end{definition}
%
% More specifically, a linear circuit must satisfy the homogeneity (scaling) property as well as well as the additive property. Homogeneity means
%
% \begin{figure}[H]
%   
%   \caption{Linear}
%   \label{fig:}
% \end{figure}
%
% \textit{Power is non-linear}.

\subsection{Superposition}
\label{ssec:superposition}

\begin{definition}{Superposition Principle}
  The superposition principle states that the voltage across an element \textit{in a linear} circuit is the algebraic sum of the voltages across that element due to each independent source \textit{acting alone}.
\end{definition}

\begin{figure}[H]
  \centering
  \includestandalone{figures/fig_011}
  \caption{Circuit}
  \label{fig:011}
\end{figure}
Consider the circuit in Figure \ref{fig:011}. This circuit has two sources, and while it can be analyzed through mesh and nodal analysis, the principle of superposition can also be applied to it. By doing so, two similar circuits can be constructed, each taking only a single source at once.
\begin{figure}[H]
  \centering
  \begin{subfigure}[H]{0.45\textwidth}
    \centering
    \includestandalone{figures/fig_012}
    \caption{Voltage Source}
    \label{fig:012}
  \end{subfigure}
  \begin{subfigure}[H]{0.45\textwidth}
    \centering
    \includestandalone{figures/fig_013}
    \caption{Current Source}
    \label{fig:013}
  \end{subfigure}
  \caption{Sub-Circuits}
  \label{fig:subcircuits}
\end{figure}
From here, each circuit can be analyzed independently. The values found for voltages and currents throughout the circuits can be summed to find the values from the original circuit in Figure \ref{fig:011}.

\subsection{Source Transformation}
\label{ssec:sourceTransformation}

\begin{definition}{Source Transformation}
  A source transformation is the process of replacing a voltage source $V_s$ in series with a resistor $R$ with a current source $I_s$ in parallel with a resistor or vice versa.
\end{definition}

\begin{figure}[H]
  \vspace{-10pt}
  \centering
  \includestandalone{figures/fig_014}
  \caption{Source Transformation}
  \label{fig:014}
  \vspace{-10pt}
\end{figure}

In Figure \ref{fig:014}, the voltage source and current source can be switched back and forth by applying source transformation.

To transform a voltage source to a current source, the voltage source must be \textit{in series} with a resistor. The transformation will then yield a current source \textit{in parallel} with the resistor, and a current source value calculated from Ohm's Law ($I = \frac{V}{R}$). Consider the circuits in Figure \ref{fig:voltageTransformations}. In each circuit, the voltage sources that \textit{can} be transformed into current sources are highlighted in {\color{gr} green}.

\begin{figure}[H]
  \centering
  \begin{subfigure}[H]{0.3\textwidth}
    \centering
    \includestandalone{figures/fig_034}
    \caption{}
    \label{fig:034}
  \end{subfigure}
  \begin{subfigure}[H]{0.3\textwidth}
    \centering
    \includestandalone{figures/fig_035}
    \caption{}
    \label{fig:035}
  \end{subfigure}
  \begin{subfigure}[H]{0.3\textwidth}
    \centering
    \includestandalone{figures/fig_036}
    \caption{}
    \label{fig:036}
  \end{subfigure}
  \caption{Voltage Transformations}
  \label{fig:voltageTransformations}
\end{figure}

The applied source transformations to each circuit can be seen in Figure \ref{fig:voltageTransformationsAfter}.

\begin{figure}[H]
  \centering
  \begin{subfigure}[H]{0.3\textwidth}
    \centering
    \includestandalone{figures/fig_037}
    \caption{}
    \label{fig:037}
  \end{subfigure}
  \begin{subfigure}[H]{0.3\textwidth}
    \centering
    \includestandalone{figures/fig_038}
    \caption{}
    \label{fig:038}
  \end{subfigure}
  \begin{subfigure}[H]{0.3\textwidth}
    \centering
    \includestandalone{figures/fig_039}
    \caption{}
    \label{fig:039}
  \end{subfigure}
  \caption{Voltage Transformations}
  \label{fig:voltageTransformationsAfter}
\end{figure}

The same process, just reversed, can be done to transform a current source into a voltage source. If a current source exists \textit{in parallel} with a resistor, it can be transformed into a voltage source \textit{in series} with that resistor. The voltage on the voltage source is determined by Ohm's Law ($V = IR$).

\subsection{Thevenin's Theorem}
\label{ssec:theveninsTheorem}

\end{document}
