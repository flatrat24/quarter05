\documentclass[12pt]{article}

%%%% GRAPHICS %%%%
\usepackage{tikz}
\usepackage[siunitx, american, RPvoltages]{circuitikz}
\usetikzlibrary{arrows.meta}
\usepackage{tikz-3dplot}
\usepackage{graphicx}
\usepackage{pgfplots}
  \pgfplotsset{compat=1.18}
\usetikzlibrary{arrows}
\newcommand{\midarrow}{\tikz \draw[-triangle 90] (0,0) -- +(.1,0);}

%%%% FIGURES %%%%
\usepackage{subcaption}
\usepackage{wrapfig}
\usepackage{float}
\usepackage[skip=5pt, font=footnotesize]{caption}

%%%% FORMATTING %%%%
\usepackage{parskip}
\usepackage{tcolorbox}
\usepackage{ulem}

%%%% TABLE FORMATTING %%%%
\usepackage{tabularray}
\UseTblrLibrary{booktabs}

%%%% MATH AND LOGIC %%%%
\usepackage{xifthen}
\usepackage{amsmath}
\usepackage{amssymb}
\usepackage{amsfonts}

%%%% TEXT AND SYMBOLS %%%%
\usepackage[T1]{fontenc}
\usepackage{textcomp}
\usepackage{gensymb}

%%%% OTHER %%%%
\usepackage{standalone}

%%%% LOGIC SYMBOLS %%%%
\newcommand*\xor{\oplus}

%%%% STYLES %%%%

% Packages
\usepackage[paper=letterpaper,tmargin=45pt,bmargin=45pt,lmargin=45pt,rmargin=45pt]{geometry}
\usepackage{titlesec}
\usepackage[rgb]{xcolor}
\selectcolormodel{natural}
\usepackage{ninecolors}
\selectcolormodel{rgb}

% Colors
\definecolor{pg}{HTML}{24273A}
\definecolor{fg}{HTML}{FFFFFF}
\definecolor{bg}{HTML}{24273A}
\definecolor{re}{HTML}{F38BA8}
\definecolor{gr}{HTML}{A6E3A1}
\definecolor{ye}{HTML}{F9E2AF}
\definecolor{or}{HTML}{FAB387}
\definecolor{bl}{HTML}{89B4FA}
\definecolor{ma}{HTML}{CBA6F7}
\definecolor{cy}{HTML}{94E2D5}
\definecolor{pi}{HTML}{F2CDCD}

\definecolor{copper}{HTML}{B87333}

\usepackage{nameref}
\makeatletter
\newcommand*{\currentname}{\@currentlabelname}
\makeatother

\titleformat{\section}
  {\normalfont\scshape\Large\bfseries}
  {\thesection}
  {0.75em}
  {}

\titleformat{\subsection}
  {\normalfont\scshape\large\bfseries}
  {\thesubsection}
  {0.75em}
  {}

\titleformat{\subsubsection}
  {\normalfont\scshape\normalsize\bfseries}
  {\thesubsubsection}
  {0.75em}
  {}

% Formula
\newcounter{formula}[section]
\newenvironment{formula}[1]{
  \stepcounter{formula}
  \begin{tcolorbox}[
    standard jigsaw, % Allows opacity
    colframe={fg},
    boxrule=1px,
    colback=bg,
    opacityback=0,
    sharp corners,
    sidebyside,
    righthand width=18px,
    coltext={fg}
  ]
  \centering
  \textbf{\uline{#1}}
}{
  \tcblower
  \textbf{\thesection.\theformula}
  \end{tcolorbox}
}

% Definition
\newcounter{definition}[section]

\newenvironment{definition*}[1]{
  \begin{tcolorbox}[
    standard jigsaw, % Allows opacity
    colframe={fg},
    boxrule=1px,
    colback=bg,
    opacityback=0,
    sharp corners,
    coltext={fg}
  ]
  \textbf{#1 \hfill}
  \vspace{5px}
  \hrule
  \vspace{5px}
  \noindent
}{
  \end{tcolorbox}
}

\newenvironment{definition}[1]{
  \stepcounter{definition}
  \begin{tcolorbox}[
    standard jigsaw, % Allows opacity
    colframe={fg},
    boxrule=1px,
    colback=bg,
    opacityback=0,
    sharp corners,
    coltext={fg}
  ]
  \textbf{#1 \hfill \thesection.\thedefinition}
  \vspace{5px}
  \hrule
  \vspace{5px}
  \noindent
}{
  \end{tcolorbox}
}

% Example Problem
\newcounter{example}[section]
\newenvironment{example}{
  \stepcounter{example}
  \begin{tcolorbox}[
    standard jigsaw, % Allows opacity
    colframe={fg},
    boxrule=1px,
    colback=bg,
    opacityback=0,
    sharp corners,
    coltext={fg}
  ]
  \textbf{Example \hfill \thesection.\theexample}
  \vspace{5px}
  \hrule
  \vspace{5px}
  \noindent
}{
  \end{tcolorbox}
}

\tikzset{
  cubeBorder/.style=fg,
  cubeFilling/.style={fg!20!bg, opacity=0.25},
  gridLine/.style={very thin, gray},
  graphLine/.style={-latex, thick, fg},
}

\pgfplotsset{
  basicAxis/.style={
    grid,
    major grid style={line width=.2pt,draw=fg!50!bg},
    axis lines = box,
    axis line style = {line width = 1px},
  }
}

%%%% REFERENCES %%%%
\usepackage{hyperref}
\hypersetup{
  colorlinks  = true,
  linkcolor   = bl,
  anchorcolor = bl,
  citecolor   = bl,
  filecolor   = bl,
  menucolor   = bl,
  runcolor    = bl,
  urlcolor    = bl,
}

\author{Ethan Anthony}
\newcommand*{\equal}{=}


\title{Lecture 004}
\date{February 03, 2025}

\begin{document}

\newpage
\section{Circuit Theorems}
\label{sec:circuitTheorems}

\subsection{Superposition}
\label{ssec:superposition}

\begin{definition}{Superposition Principle}
  The superposition principle states that the voltage across an element \textit{in a linear} circuit is the algebraic sum of the voltages across that element due to each independent source \textit{acting alone}.
\end{definition}

\begin{figure}[H]
  \centering
  \includestandalone{figures/fig_011}
  \caption{Circuit}
  \label{fig:011}
\end{figure}
Consider the circuit in Figure \ref{fig:011}. This circuit has two sources, and while it can be analyzed through mesh and nodal analysis, the principle of superposition can also be applied to it. By doing so, two similar circuits can be constructed, each taking only a single source at once.
\begin{figure}[H]
  \centering
  \begin{subfigure}[H]{0.45\textwidth}
    \centering
    \includestandalone{figures/fig_012}
    \caption{Voltage Source}
    \label{fig:012}
  \end{subfigure}
  \begin{subfigure}[H]{0.45\textwidth}
    \centering
    \includestandalone{figures/fig_013}
    \caption{Current Source}
    \label{fig:013}
  \end{subfigure}
  \caption{Sub-Circuits}
  \label{fig:subcircuits}
\end{figure}
From here, each circuit can be analyzed independently. The values found for voltages and currents throughout the circuits can be summed to find the values from the original circuit in Figure \ref{fig:011}.

\subsection{Source Transformation}
\label{ssec:sourceTransformation}

\begin{definition}{Source Transformation}
  A source transformation is the process of replacing a voltage source $V_s$ in series with a resistor $R$ with a current source $I_s$ in parallel with a resistor or vice versa.
\end{definition}

\begin{figure}[H]
  \vspace{-10pt}
  \centering
  \includestandalone{figures/fig_014}
  \caption{Source Transformation}
  \label{fig:014}
  \vspace{-10pt}
\end{figure}

In Figure \ref{fig:014}, the voltage source and current source can be switched back and forth by applying source transformation.

To transform a voltage source to a current source, the voltage source must be \textit{in series} with a resistor. The transformation will then yield a current source \textit{in parallel} with the resistor, and a current source value calculated from Ohm's Law ($I = \frac{V}{R}$). Consider the circuits in Figure \ref{fig:voltageTransformations}. In each circuit, the voltage sources that \textit{can} be transformed into current sources are highlighted in {\color{gr} green}.

\begin{figure}[H]
  \centering
  \begin{subfigure}[H]{0.3\textwidth}
    \centering
    \includestandalone{figures/fig_034}
    \caption{}
    \label{fig:034}
  \end{subfigure}
  \begin{subfigure}[H]{0.3\textwidth}
    \centering
    \includestandalone{figures/fig_035}
    \caption{}
    \label{fig:035}
  \end{subfigure}
  \begin{subfigure}[H]{0.3\textwidth}
    \centering
    \includestandalone{figures/fig_036}
    \caption{}
    \label{fig:036}
  \end{subfigure}
  \caption{Voltage Transformations}
  \label{fig:voltageTransformations}
\end{figure}

The applied source transformations to each circuit can be seen in Figure \ref{fig:voltageTransformationsAfter}.

\begin{figure}[H]
  \centering
  \begin{subfigure}[H]{0.3\textwidth}
    \centering
    \includestandalone{figures/fig_037}
    \caption{}
    \label{fig:037}
  \end{subfigure}
  \begin{subfigure}[H]{0.3\textwidth}
    \centering
    \includestandalone{figures/fig_038}
    \caption{}
    \label{fig:038}
  \end{subfigure}
  \begin{subfigure}[H]{0.3\textwidth}
    \centering
    \includestandalone{figures/fig_039}
    \caption{}
    \label{fig:039}
  \end{subfigure}
  \caption{Voltage Transformations}
  \label{fig:voltageTransformationsAfter}
  \vspace{-10pt}
\end{figure}

The same process, just reversed, can be done to transform a current source into a voltage source. If a current source exists \textit{in parallel} with a resistor, it can be transformed into a voltage source \textit{in series} with that resistor. The voltage on the voltage source is determined by Ohm's Law ($V = IR$).

\subsection{Thevenin's Theorem}
\label{ssec:theveninsTheorem}

\begin{definition}{Thevenin's Theorem}
  States that a linear two-terminal circuit can be replaced by an equivalent circuit consisting of a voltage source ($V_{th}$) in series with a resistor ($R_{th}$) where $V_{th}$ is the \textbf{open-circuit voltage} and $R_{th}$ is the equivalent resistance at the terminals when the independent sources are turned off.
\end{definition}

Consider the circuit in Figure \ref{fig:042}.
\begin{figure}[H]
  \centering
  \begin{subfigure}[H]{0.45\textwidth}
    \centering
    \includestandalone{figures/fig_042}
    \caption{Circuit}
    \label{fig:042}
  \end{subfigure}
  \begin{subfigure}[H]{0.45\textwidth}
    \centering
    \includestandalone{figures/fig_043}
    \caption{Thevenin Equivalent}
    \label{fig:043}
  \end{subfigure}
  \caption{Thevenin's Theorem}
  \label{fig:theveninsTheoremOne}
  \vspace{-10pt}
\end{figure}
Thevenin's Theorem states that this circuit, seeing that it is indeed linear, can be modeled as an equivalent circuit between terminals $a$ and $b$ as simply a resistor and voltage source in series.

To find the Thevenin equivalent circuit, $V_{th}$ and $R_{th}$ must be found. To find $V_{th}$, as in Figure \ref{fig:040}, is found by opening the circuit between $a$ and $b$. To find $R_{th}$, the resistance of the circuit must be analyzed by shorting voltage sources and opening current sources.
\begin{figure}[H]
  \centering
  \begin{subfigure}[H]{0.45\textwidth}
    \centering
    \includestandalone{figures/fig_040}
    \caption{Open Circuit Voltage}
    \label{fig:040}
  \end{subfigure}
  \begin{subfigure}[H]{0.45\textwidth}
    \centering
    \includestandalone{figures/fig_041}
    \caption{Dead Circuit Resistance}
    \label{fig:041}
  \end{subfigure}
  \caption{Finding Thevenin's Theorem}
  \label{fig:theveninsTheorem}
\end{figure}
Some circuits, such as the one in Figure \ref{fig:044}, aren't possible to find the $R_{th}$ of just through adding resistance values in series and parallel.
\begin{figure}[H]
  \centering
  \includestandalone{figures/fig_044}
  \caption{Finding $R_{th}$}
  \label{fig:044}
\end{figure}
In such cases, by inserting some dummy voltage source between the terminals $a$ and $b$ and solving for the current over that branch, Ohm's Law can then be applied to calculate the resistance:
\begin{equation*}
  R_{th} = \frac{V_{dummy}}{I_{calculated}}
\end{equation*}

\subsection{Norton's Theorem}
\label{ssec:nortonsTheorem}

\begin{definition}{Norton's Theorem}
  States that a linear two-terminal circuit can be replaced by an equivalent circuit consisting of a current source ($I_{N}$) in series with a resistor ($R_{N}$) where $I_{N}$ is the \textbf{short-circuit current} and $R_{N}$ is the equivalent resistance at the terminals when the independent sources are turned off.
\end{definition}

Consider the circuit in Figure \ref{fig:047}.
\begin{figure}[H]
  \centering
  \begin{subfigure}[H]{0.45\textwidth}
    \centering
    \includestandalone{figures/fig_047}
    \caption{Circuit}
    \label{fig:047}
  \end{subfigure}
  \begin{subfigure}[H]{0.45\textwidth}
    \centering
    \includestandalone{figures/fig_048}
    \caption{Norton Equivalent}
    \label{fig:048}
  \end{subfigure}
  \caption{Norton's Theorem}
  \label{fig:NortonTheoremOne}
  \vspace{-10pt}
\end{figure}
Similar to Thevenin's Theorem, since this circuit still is linear, it can also be expressed as a simple circuit with just a current source and resistor in parallel. Notice that a Thevenin equivalent and a Norton equivalent circuit are simply source transformed versions of the other (see Figure \ref{fig:014}).

To solve for a Norton equivalent circuit, $I_N$ and $R_N$ must be found. $R_N$ can be found the same as previously in Subsection \ref{ssec:theveninsTheorem}. $I_N$, however, is solved by shorting terminals $a$ and $b$, and finding the current over the wire.

\begin{figure}[H]
  \centering
  \begin{subfigure}[H]{0.45\textwidth}
    \centering
    \includestandalone{figures/fig_045}
    \caption{Short Circuit Current}
    \label{fig:045}
  \end{subfigure}
  \begin{subfigure}[H]{0.45\textwidth}
    \centering
    \includestandalone{figures/fig_046}
    \caption{Dead Circuit Resistance}
    \label{fig:046}
  \end{subfigure}
  \caption{Norton's Theorem}
  \label{fig:nortonsTheorem}
  \vspace{-10pt}
\end{figure}

\subsection{Power Transfer}
\label{ssec:powerTransfer}

In a circuit reduced to the form of a Thevenin or Norton circuit, the next step of the analysis might then to be determine the amount of \textbf{power} consumed by some load connected between the terminals $a$ and $b$. As in Subsection \ref{ssec:powerAndEnergy}, the power can be calculated as:
\begin{equation*}
  P = I \cdot V
\end{equation*}
\begin{wrapfigure}[5]{r}{0.3\textwidth}
  \vspace{-30pt}
  \centering
  \includestandalone{figures/fig_049}
  \caption{Thevenin Equivalent}
  \label{fig:049}
\end{wrapfigure}
For a Thevenin circuit, this could take the form of multiplying the current in the circuit by the voltage drop across the load (with a voltage divider):
\begin{equation*}
  P = {\color{gr} I} \cdot {\color{re} V} = \left[{\color{gr} \frac{V_{th}}{R_{th}+R_{L}}}\right] \cdot \left[{\color{re} V_{th}\frac{R_{L}}{R_{L}+R_{th}}}\right] = {\color{bl} R_L \cdot \left(\frac{V_{th}}{R_L+R_{th}}\right)^2}
\end{equation*}
Thus, giving the equation for the power consumed by the load in terms of $R_L$, $R_{th}$, and $V_{th}$.

\begin{wrapfigure}[6]{l}{0.3\textwidth}
  \vspace{-20pt}
  \centering
  \includestandalone{figures/fig_050}
  \caption{Norton Equivalent}
  \label{fig:050}
\end{wrapfigure}

For a Norton circuit, it might be simpler to see it as a {\color{ma} current divider} for $I$ multiplied with an {\color{or} Ohm's Law} to find $V$:
\begin{equation*}
  P = {\color{ma} I} \cdot {\color{or} V} = \left[{\color{ma} \frac{I_{N}}{\left(R_{N}^{-1}+R_{L}^{-1}\right)R_{L}}}\right] \cdot \left[{\color{or} \frac{I_{N}}{R_{N}^{-1}+R_{L}^{-1}}}\right] = {\color{cy} \left(\frac{I_N}{R_{N}^{-1}+R_{L}^{-1}}\right)^2 \cdot \frac{1}{R_L}}
\end{equation*}
Thus giving the power consumed in terms of $R_L$, $R_N$ and $I_{N}$.

Considering the fact that $V_{th}=I_N \cdot R_{th/N}$ and $I_N = \frac{V_{th}}{R_{th/N}}$ by Ohm's Law, these two equations can be shown to be equivalent:
\begin{gather*}
  \left[{\color{bl} R_L \cdot \left(\frac{V_{th}}{R_L+R_{th}}\right)^2} = {\color{cy} \left(\frac{I_N}{R_{N}^{-1}+R_{L}^{-1}}\right)^2 \cdot \frac{1}{R_L}}\right] \rightarrow \left[R_{\color{fg} L}^{\color{re} 2} \cdot \left(\frac{V_{th}}{R_L+R_{th}}\right)^{\color{re} 2} = \left(\frac{I_N}{R_{N}^{-1}+R_{L}^{-1}}\right)^{\color{re} 2}\right] \rightarrow \\
  \left[R_L \cdot \frac{{\color{gr} V_{th}}}{R_L+R_{th}} = \frac{I_N}{R_{N}^{-1}+R_{L}^{-1}}\right] \rightarrow \left[R_L \cdot \frac{{\color{gr} {\color{re} I_N} \cdot R_{th/N}}}{R_L+R_{th}} = \frac{{\color{re} I_N}}{R_{N}^{-1}+R_{L}^{-1}}\right] \rightarrow \left[\frac{R_L \cdot R_{th/N}}{R_L+R_{th}} = \frac{{\color{re} 1}}{R_{N}^{-1}+R_{L}^{-1}}\right] \rightarrow \\
  \left[\frac{R_L+R_{th}}{R_L \cdot R_{th/N}} = \frac{1}{R_N} + \frac{1}{R_L}\right] \rightarrow \left[\frac{R_L}{R_L \cdot R_{th/N}} + \frac{R_{th}}{R_L \cdot R_{th/N}} = \frac{1}{R_N} + \frac{1}{R_L}\right] \rightarrow \left[\frac{1}{R_N} + \frac{1}{R_L} = \frac{1}{R_N} + \frac{1}{R_L}\right]
\end{gather*}

\subsubsection{Maximum Power Transfer}
\label{sssec:maximumPowerTransfer}

Often, maximizing the power consumed by a load is desirable in the design process. Using calculus, and specifically optimization, the load resistance that yields the maximum power consumed by the load can be solved for. Consider the Thevenin equation for power:
\begin{equation*}
  P = R_L \cdot \left(\frac{V_{th}}{R_L+R_{th}}\right)^2
\end{equation*}
The first step of optimization is to derive the equation. Deriving with respect to $R_L$:
\begin{equation*}
  \frac{dP}{dR_L} = V_{th}^2 \cdot \left(\frac{R_{th}-R_L}{\left(R_{th}+R_{L}\right)^3}\right)
\end{equation*}
Then, set $\frac{dP}{dR_L}=0$ to find critical points. It is only at these points where an extrema (whether max or min) may occur:
\begin{equation*}
  V_{th}^2 \cdot \left(\frac{R_{th}-R_L}{\left(R_{th}+R_{L}\right)^3}\right) = 0 \rightarrow R_{th}-R_L = 0 \rightarrow R_{th}=R_L
\end{equation*}
Thus, when $R_L=R_{th}$, the maximum amount of power consumption over the load occurs. This process can be done analogously for the Norton euqation for power, and will give the same answer.
\begin{definition}{Maximum Power}
  The maximum power is transferred to the load when the load resistance equals the Thevenin/Norton resistance as seen from the load.
  \begin{equation*}
    R_L = R_{th/N}
  \end{equation*}
\end{definition}

\end{document}
