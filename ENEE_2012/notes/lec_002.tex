\documentclass[12pt]{article}

%%%% GRAPHICS %%%%
\usepackage{tikz}
\usepackage[siunitx, american, RPvoltages]{circuitikz}
\usetikzlibrary{arrows.meta}
\usepackage{tikz-3dplot}
\usepackage{graphicx}
\usepackage{pgfplots}
  \pgfplotsset{compat=1.18}
\usetikzlibrary{arrows}
\newcommand{\midarrow}{\tikz \draw[-triangle 90] (0,0) -- +(.1,0);}

%%%% FIGURES %%%%
\usepackage{subcaption}
\usepackage{wrapfig}
\usepackage{float}
\usepackage[skip=5pt, font=footnotesize]{caption}

%%%% FORMATTING %%%%
\usepackage{parskip}
\usepackage{tcolorbox}
\usepackage{ulem}

%%%% TABLE FORMATTING %%%%
\usepackage{tabularray}
\UseTblrLibrary{booktabs}

%%%% MATH AND LOGIC %%%%
\usepackage{xifthen}
\usepackage{amsmath}
\usepackage{amssymb}
\usepackage{amsfonts}

%%%% TEXT AND SYMBOLS %%%%
\usepackage[T1]{fontenc}
\usepackage{textcomp}
\usepackage{gensymb}

%%%% OTHER %%%%
\usepackage{standalone}

%%%% LOGIC SYMBOLS %%%%
\newcommand*\xor{\oplus}

%%%% STYLES %%%%

% Packages
\usepackage[paper=letterpaper,tmargin=45pt,bmargin=45pt,lmargin=45pt,rmargin=45pt]{geometry}
\usepackage{titlesec}
\usepackage[rgb]{xcolor}
\selectcolormodel{natural}
\usepackage{ninecolors}
\selectcolormodel{rgb}

% Colors
\definecolor{pg}{HTML}{24273A}
\definecolor{fg}{HTML}{FFFFFF}
\definecolor{bg}{HTML}{24273A}
\definecolor{re}{HTML}{F38BA8}
\definecolor{gr}{HTML}{A6E3A1}
\definecolor{ye}{HTML}{F9E2AF}
\definecolor{or}{HTML}{FAB387}
\definecolor{bl}{HTML}{89B4FA}
\definecolor{ma}{HTML}{CBA6F7}
\definecolor{cy}{HTML}{94E2D5}
\definecolor{pi}{HTML}{F2CDCD}

\definecolor{copper}{HTML}{B87333}

\usepackage{nameref}
\makeatletter
\newcommand*{\currentname}{\@currentlabelname}
\makeatother

\titleformat{\section}
  {\normalfont\scshape\Large\bfseries}
  {\thesection}
  {0.75em}
  {}

\titleformat{\subsection}
  {\normalfont\scshape\large\bfseries}
  {\thesubsection}
  {0.75em}
  {}

\titleformat{\subsubsection}
  {\normalfont\scshape\normalsize\bfseries}
  {\thesubsubsection}
  {0.75em}
  {}

% Formula
\newcounter{formula}[section]
\newenvironment{formula}[1]{
  \stepcounter{formula}
  \begin{tcolorbox}[
    standard jigsaw, % Allows opacity
    colframe={fg},
    boxrule=1px,
    colback=bg,
    opacityback=0,
    sharp corners,
    sidebyside,
    righthand width=18px,
    coltext={fg}
  ]
  \centering
  \textbf{\uline{#1}}
}{
  \tcblower
  \textbf{\thesection.\theformula}
  \end{tcolorbox}
}

% Definition
\newcounter{definition}[section]

\newenvironment{definition*}[1]{
  \begin{tcolorbox}[
    standard jigsaw, % Allows opacity
    colframe={fg},
    boxrule=1px,
    colback=bg,
    opacityback=0,
    sharp corners,
    coltext={fg}
  ]
  \textbf{#1 \hfill}
  \vspace{5px}
  \hrule
  \vspace{5px}
  \noindent
}{
  \end{tcolorbox}
}

\newenvironment{definition}[1]{
  \stepcounter{definition}
  \begin{tcolorbox}[
    standard jigsaw, % Allows opacity
    colframe={fg},
    boxrule=1px,
    colback=bg,
    opacityback=0,
    sharp corners,
    coltext={fg}
  ]
  \textbf{#1 \hfill \thesection.\thedefinition}
  \vspace{5px}
  \hrule
  \vspace{5px}
  \noindent
}{
  \end{tcolorbox}
}

% Example Problem
\newcounter{example}[section]
\newenvironment{example}{
  \stepcounter{example}
  \begin{tcolorbox}[
    standard jigsaw, % Allows opacity
    colframe={fg},
    boxrule=1px,
    colback=bg,
    opacityback=0,
    sharp corners,
    coltext={fg}
  ]
  \textbf{Example \hfill \thesection.\theexample}
  \vspace{5px}
  \hrule
  \vspace{5px}
  \noindent
}{
  \end{tcolorbox}
}

\tikzset{
  cubeBorder/.style=fg,
  cubeFilling/.style={fg!20!bg, opacity=0.25},
  gridLine/.style={very thin, gray},
  graphLine/.style={-latex, thick, fg},
}

\pgfplotsset{
  basicAxis/.style={
    grid,
    major grid style={line width=.2pt,draw=fg!50!bg},
    axis lines = box,
    axis line style = {line width = 1px},
  }
}

%%%% REFERENCES %%%%
\usepackage{hyperref}
\hypersetup{
  colorlinks  = true,
  linkcolor   = bl,
  anchorcolor = bl,
  citecolor   = bl,
  filecolor   = bl,
  menucolor   = bl,
  runcolor    = bl,
  urlcolor    = bl,
}

\author{Ethan Anthony}
\newcommand*{\equal}{=}


\title{Lecture 002}
\date{January 21, 2025}

\begin{document}
\newpage

\section{Basic Laws}
\label{sec:basicLaws}

\subsection{Ohm's Law}
\label{ssec:ohmsLaw}

\textbf{Ohm's Law} states that the voltage $V$ is \textit{directly} proportional to the current $I$ and resistance $R$ of a circuit.

\begin{formula}{Ohm's Law}
  \begin{equation*}
    V = IR
  \end{equation*}
  Where:
  \begin{equation*}
    R = \rho \cdot \frac{l}{A}
  \end{equation*}
\end{formula}

When there is current flowing through a wire with resistance approaching zero, a \textbf{short circuit} is created. Conversely, an \textbf{open circuit} is where the resistance in a circuit approaches infinity.

Whereas resistance measures how much something impedes the flow of current, \textbf{conductance} is the ability of an element of conduct electric current; it is measured in mhos ($\mho$) or siemens ($S$).

\subsubsection{Resistivity}
\label{sssec:resistivity}

\begin{definition}{Resistivity ($\rho$)}
  A value representing the amount a material conducts electricity. Most metals tend to have high conductivity, while rubber has a low conductivity.
\end{definition}

The resistance $\si{R}$ of a wire, for example, is a value to varies with the inherent properties as well as the dimensions of the wire and its materials. The inherent properties of a material as they relate to resistance are codified as a material's \textbf{resistivity}.

\subsection{Nodes, Branches, and Loops}
\label{ssec:nodesBranchesLoops}

\begin{definition}{Branch $(b)$}
  Represents a single element in a circuit such as a resistor or power supply.
\end{definition}

\begin{wrapfigure}[7]{l}{0.35\textwidth}
  \centering
  \includestandalone{figures/fig_007}
  \caption{Nodes and Branches}
  \label{fig:007}
\end{wrapfigure}

In Figure \ref{fig:007}, there exist \textit{five} branches in the circuit. Specifically, there are three resistors, one voltage source, and one current source:
\begin{equation*}
  3 + 1 + 1 = 5
\end{equation*}

\begin{definition}{Node ($n$)}
  The point of connection between two or more branches.
\end{definition}

In the same circuit (\ref{fig:007}), there are three nodes. Each node is highlighted in a different color. Notice that there are no branches within a node, thus each node is the largest area possible without crossing a branch. Furthermore, the voltage throughout an ideal node is zero.

\begin{definition}{Loop ($l$)}
  A loop is any \textit{closed} path in a circuit. Generally, loops are defined as the smallest possible path.
\end{definition}

\begin{figure}[H]
  \centering
  \includestandalone{figures/fig_008}
  \caption{Loops}
  \label{fig:008}
\end{figure}
\vspace{-10pt}
Each shown in a different color in Figure \ref{fig:008}, loops are easy to visualize as the area enclosed by any closed series of components.

\begin{formula}{Network Topology}
  \begin{equation*}
    b = l + n - 1
  \end{equation*}
  \vspace{-17pt}
\end{formula}

\subsubsection{In Parallel vs. Series}
\label{sssec:inParallelVsSeries}

\begin{definition}{Series}
  Two or more elements are in series if they exclusively share a single node and consequently carry the same current.
\end{definition}

\begin{figure}[H]
  \centering
  \begin{subfigure}[H]{0.45\textwidth}
    \centering
    \includestandalone{figures/fig_009}
    \caption{Loops}
    \label{fig:009}
  \end{subfigure}
  \begin{subfigure}[H]{0.45\textwidth}
    \centering
    \includestandalone{figures/fig_010}
    \caption{Loops}
    \label{fig:010}
  \end{subfigure}
  \caption{In Parallel and In Series}
  \label{fig:inParallelAndInSeries}
\end{figure}

Continuing with the same circuit, Figure \ref{fig:009} highlights elements in series. The only two elements in series are the voltage source and the top resistor.

Figure \ref{fig:010} highlights two of the three sets of parallel components. The third set would be the leftmost vertical resistor and the current source.

\begin{definition}{In Parallel}
  Two or more elements are in parallel if they are connected to the same two notes and consequently have the same voltage across them.
\end{definition}

\subsection{Kirchhoff's Laws}
\label{ssec:kirchoffsLaws}

\begin{definition}{Kirchhoff's Current Law}
  Kirchhoff's Current Law (KCL) states that the algebraic sum of currents entering a node (or a closed boundary) is zero:
  \begin{equation*}
    \sum_{n=1}^{N_{branch}} i_n = 0
  \end{equation*}
\end{definition}

\begin{definition}{Kirchhoff's Voltage Law}
  Kirchhoff's Voltage Law (KVL) states that the algebraic sum of all voltages around a closed path is zero:
  \begin{equation*}
    \sum_{m=1}^{M_{branch}} v_m = 0
  \end{equation*}
\end{definition}

\end{document}
