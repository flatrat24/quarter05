\documentclass[12pt]{article}

\input{../../xlatex/imports/preamble}

\title{Lecture 002}
\date{February 15, 2025}

\begin{document}

\newpage
\section{Circuit Elements}
\label{sec:circuitElements}

\subsection{Sources}
\label{ssec:sources}

The most basic kind of source is an \textbf{ideal independent source}. These can be in the form of a voltage or current source, and supply some fixed amount of either voltage or current.

\begin{definition}{Ideal Independent Source}
  An ideal independent source is an active element that provides specified voltage or current that is completely independent of other circuit elements.
\end{definition}

\begin{figure}[H]
  \centering
  \begin{subfigure}[H]{0.45\textwidth}
    \centering
    \begin{subfigure}[H]{0.45\textwidth}
      \centering
      \includestandalone{figures/fig_015}
      \caption{Voltage Source}
      \label{fig:015}
    \end{subfigure}
    \begin{subfigure}[H]{0.45\textwidth}
      \centering
      \includestandalone{figures/fig_016}
      \caption{Current Source}
      \label{fig:016}
    \end{subfigure}
    \caption{Independent Sources}
    \label{fig:independentSources}
  \end{subfigure}
  \vrule
  \begin{subfigure}[H]{0.45\textwidth}
    \centering
    \begin{subfigure}[H]{0.45\textwidth}
      \centering
      \includestandalone{figures/fig_017}
      \caption{Voltage Source}
      \label{fig:017}
    \end{subfigure}
    \begin{subfigure}[H]{0.45\textwidth}
      \centering
      \includestandalone{figures/fig_018}
      \caption{Current Source}
      \label{fig:018}
    \end{subfigure}
    \caption{Controlled Sources}
    \label{fig:controlledSources}
  \end{subfigure}
  \caption{Sources}
  \label{fig:sources}
\end{figure}

In addition to ideal independent sources, there also exist \textbf{ideal controlled sources}. Again, they can provide either voltage or current, but the amount they provide is dependent on the circuit they are a part of.

\begin{definition}{Ideal Controlled Source}
  An ideal controlled source is an active element in which the source quantity is controlled by another voltage or current.
\end{definition}

\begin{wrapfigure}[11]{r}{0.45\textwidth}
  \vspace{-10pt}
  \begin{subfigure}[H]{0.45\textwidth}
    \begin{subfigure}[H]{0.45\textwidth}
      \includestandalone{figures/fig_019}
      \caption{VCVS}
      \label{fig:019}
    \end{subfigure}
    \begin{subfigure}[H]{0.45\textwidth}
      \includestandalone{figures/fig_020}
      \caption{CCVS}
      \label{fig:020}
    \end{subfigure}
  \end{subfigure}
  \begin{subfigure}[H]{0.45\textwidth}
    \begin{subfigure}[H]{0.45\textwidth}
      \includestandalone{figures/fig_021}
      \caption{VCCS}
      \label{fig:021}
    \end{subfigure}
    \begin{subfigure}[H]{0.45\textwidth}
      \includestandalone{figures/fig_022}
      \caption{CCCS}
      \label{fig:022}
    \end{subfigure}
  \end{subfigure}
  \caption{Controlled Source Types}
  \label{fig:controlledSourceTypes}
\end{wrapfigure}

As stated earlier, controlled sources can provide either current or voltage. Furthermore, they can be \textit{dependent} on either some voltage ($V_x$) or some current ($I_x$) in a circuit. Thus, there are four kinds of controlled sources:

\begin{itemize}
  \itemsep0em
  \item \textbf{V}oltage \textbf{C}ontrolled \textbf{V}oltage \textbf{S}ource (\ref{fig:019})
  \item \textbf{C}urrent \textbf{C}ontrolled \textbf{V}oltage \textbf{S}ource (\ref{fig:020})
  \item \textbf{V}oltage \textbf{C}ontrolled \textbf{C}urrent \textbf{S}ource (\ref{fig:021})
  \item \textbf{C}urrent \textbf{C}ontrolled \textbf{C}urrent \textbf{S}ource (\ref{fig:022})
\end{itemize}

\subsection{Resistors}
\label{ssec:resistors}

The most basic kind of resistor is just a fixed resistor (as seen in Figure \ref{fig:024}). These resistors will have some set value of $R$ that does not change.

\begin{figure}[H]
  \centering
  \includestandalone{figures/fig_024}
  \caption{Fixed Resistor}
  \label{fig:024}
\end{figure}

There also exist variable resistors (also called potentiometers). These resistors can by manipulated manually, or through the design of a circuit, to have variable resistances. (Figure \ref{fig:025})

\begin{figure}[H]
  \centering
  \includestandalone{figures/fig_025}
  \caption{Variable Resistor}
  \label{fig:025}
\end{figure}

\end{document}
