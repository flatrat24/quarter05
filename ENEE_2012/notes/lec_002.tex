\documentclass[12pt]{article}

%%%% GRAPHICS %%%%
\usepackage{tikz}
\usepackage[siunitx, american, RPvoltages]{circuitikz}
\usetikzlibrary{arrows.meta}
\usepackage{tikz-3dplot}
\usepackage{graphicx}
\usepackage{pgfplots}
  \pgfplotsset{compat=1.18}
\usetikzlibrary{arrows}
\newcommand{\midarrow}{\tikz \draw[-triangle 90] (0,0) -- +(.1,0);}

%%%% FIGURES %%%%
\usepackage{subcaption}
\usepackage{wrapfig}
\usepackage{float}
\usepackage[skip=5pt, font=footnotesize]{caption}

%%%% FORMATTING %%%%
\usepackage{parskip}
\usepackage{tcolorbox}
\usepackage{ulem}

%%%% TABLE FORMATTING %%%%
\usepackage{tabularray}
\UseTblrLibrary{booktabs}

%%%% MATH AND LOGIC %%%%
\usepackage{xifthen}
\usepackage{amsmath}
\usepackage{amssymb}
\usepackage{amsfonts}

%%%% TEXT AND SYMBOLS %%%%
\usepackage[T1]{fontenc}
\usepackage{textcomp}
\usepackage{gensymb}

%%%% OTHER %%%%
\usepackage{standalone}

%%%% LOGIC SYMBOLS %%%%
\newcommand*\xor{\oplus}

%%%% STYLES %%%%

% Packages
\usepackage[paper=letterpaper,tmargin=45pt,bmargin=45pt,lmargin=45pt,rmargin=45pt]{geometry}
\usepackage{titlesec}
\usepackage[rgb]{xcolor}
\selectcolormodel{natural}
\usepackage{ninecolors}
\selectcolormodel{rgb}

% Colors
\definecolor{pg}{HTML}{24273A}
\definecolor{fg}{HTML}{FFFFFF}
\definecolor{bg}{HTML}{24273A}
\definecolor{re}{HTML}{F38BA8}
\definecolor{gr}{HTML}{A6E3A1}
\definecolor{ye}{HTML}{F9E2AF}
\definecolor{or}{HTML}{FAB387}
\definecolor{bl}{HTML}{89B4FA}
\definecolor{ma}{HTML}{CBA6F7}
\definecolor{cy}{HTML}{94E2D5}
\definecolor{pi}{HTML}{F2CDCD}

\definecolor{copper}{HTML}{B87333}

\usepackage{nameref}
\makeatletter
\newcommand*{\currentname}{\@currentlabelname}
\makeatother

\titleformat{\section}
  {\normalfont\scshape\Large\bfseries}
  {\thesection}
  {0.75em}
  {}

\titleformat{\subsection}
  {\normalfont\scshape\large\bfseries}
  {\thesubsection}
  {0.75em}
  {}

\titleformat{\subsubsection}
  {\normalfont\scshape\normalsize\bfseries}
  {\thesubsubsection}
  {0.75em}
  {}

% Formula
\newcounter{formula}[section]
\newenvironment{formula}[1]{
  \stepcounter{formula}
  \begin{tcolorbox}[
    standard jigsaw, % Allows opacity
    colframe={fg},
    boxrule=1px,
    colback=bg,
    opacityback=0,
    sharp corners,
    sidebyside,
    righthand width=18px,
    coltext={fg}
  ]
  \centering
  \textbf{\uline{#1}}
}{
  \tcblower
  \textbf{\thesection.\theformula}
  \end{tcolorbox}
}

% Definition
\newcounter{definition}[section]

\newenvironment{definition*}[1]{
  \begin{tcolorbox}[
    standard jigsaw, % Allows opacity
    colframe={fg},
    boxrule=1px,
    colback=bg,
    opacityback=0,
    sharp corners,
    coltext={fg}
  ]
  \textbf{#1 \hfill}
  \vspace{5px}
  \hrule
  \vspace{5px}
  \noindent
}{
  \end{tcolorbox}
}

\newenvironment{definition}[1]{
  \stepcounter{definition}
  \begin{tcolorbox}[
    standard jigsaw, % Allows opacity
    colframe={fg},
    boxrule=1px,
    colback=bg,
    opacityback=0,
    sharp corners,
    coltext={fg}
  ]
  \textbf{#1 \hfill \thesection.\thedefinition}
  \vspace{5px}
  \hrule
  \vspace{5px}
  \noindent
}{
  \end{tcolorbox}
}

% Example Problem
\newcounter{example}[section]
\newenvironment{example}{
  \stepcounter{example}
  \begin{tcolorbox}[
    standard jigsaw, % Allows opacity
    colframe={fg},
    boxrule=1px,
    colback=bg,
    opacityback=0,
    sharp corners,
    coltext={fg}
  ]
  \textbf{Example \hfill \thesection.\theexample}
  \vspace{5px}
  \hrule
  \vspace{5px}
  \noindent
}{
  \end{tcolorbox}
}

\tikzset{
  cubeBorder/.style=fg,
  cubeFilling/.style={fg!20!bg, opacity=0.25},
  gridLine/.style={very thin, gray},
  graphLine/.style={-latex, thick, fg},
}

\pgfplotsset{
  basicAxis/.style={
    grid,
    major grid style={line width=.2pt,draw=fg!50!bg},
    axis lines = box,
    axis line style = {line width = 1px},
  }
}

%%%% REFERENCES %%%%
\usepackage{hyperref}
\hypersetup{
  colorlinks  = true,
  linkcolor   = bl,
  anchorcolor = bl,
  citecolor   = bl,
  filecolor   = bl,
  menucolor   = bl,
  runcolor    = bl,
  urlcolor    = bl,
}

\author{Ethan Anthony}
\newcommand*{\equal}{=}


\title{Lecture 002}
\date{February 15, 2025}

\begin{document}

\newpage
\section{Circuit Elements}
\label{sec:circuitElements}

\subsection{Sources}
\label{ssec:sources}

The most basic kind of source is an \textbf{ideal independent source}. These can be in the form of a voltage or current source, and supply some fixed amount of either voltage or current.

\begin{definition}{Ideal Independent Source}
  An ideal independent source is an active element that provides specified voltage or current that is completely independent of other circuit elements.
\end{definition}

\begin{figure}[H]
  \centering
  \begin{subfigure}[H]{0.45\textwidth}
    \centering
    \begin{subfigure}[H]{0.45\textwidth}
      \centering
      \includestandalone{figures/fig_015}
      \caption{Voltage Source}
      \label{fig:015}
    \end{subfigure}
    \begin{subfigure}[H]{0.45\textwidth}
      \centering
      \includestandalone{figures/fig_016}
      \caption{Current Source}
      \label{fig:016}
    \end{subfigure}
    \caption{Independent Sources}
    \label{fig:independentSources}
  \end{subfigure}
  \vrule
  \begin{subfigure}[H]{0.45\textwidth}
    \centering
    \begin{subfigure}[H]{0.45\textwidth}
      \centering
      \includestandalone{figures/fig_017}
      \caption{Voltage Source}
      \label{fig:017}
    \end{subfigure}
    \begin{subfigure}[H]{0.45\textwidth}
      \centering
      \includestandalone{figures/fig_018}
      \caption{Current Source}
      \label{fig:018}
    \end{subfigure}
    \caption{Controlled Sources}
    \label{fig:controlledSources}
  \end{subfigure}
  \caption{Sources}
  \label{fig:sources}
\end{figure}

In addition to ideal independent sources, there also exist \textbf{ideal controlled sources}. Again, they can provide either voltage or current, but the amount they provide is dependent on the circuit they are a part of.

\begin{definition}{Ideal Controlled Source}
  An ideal controlled source is an active element in which the source quantity is controlled by another voltage or current.
\end{definition}

\begin{wrapfigure}[11]{r}{0.45\textwidth}
  \vspace{-10pt}
  \begin{subfigure}[H]{0.45\textwidth}
    \begin{subfigure}[H]{0.45\textwidth}
      \includestandalone{figures/fig_019}
      \caption{VCVS}
      \label{fig:019}
    \end{subfigure}
    \begin{subfigure}[H]{0.45\textwidth}
      \includestandalone{figures/fig_020}
      \caption{CCVS}
      \label{fig:020}
    \end{subfigure}
  \end{subfigure}
  \begin{subfigure}[H]{0.45\textwidth}
    \begin{subfigure}[H]{0.45\textwidth}
      \includestandalone{figures/fig_021}
      \caption{VCCS}
      \label{fig:021}
    \end{subfigure}
    \begin{subfigure}[H]{0.45\textwidth}
      \includestandalone{figures/fig_022}
      \caption{CCCS}
      \label{fig:022}
    \end{subfigure}
  \end{subfigure}
  \caption{Controlled Source Types}
  \label{fig:controlledSourceTypes}
\end{wrapfigure}

As stated earlier, controlled sources can provide either current or voltage. Furthermore, they can be \textit{dependent} on either some voltage ($V_x$) or some current ($I_x$) in a circuit. Thus, there are four kinds of controlled sources:

\begin{itemize}
  \itemsep0em
  \item \textbf{V}oltage \textbf{C}ontrolled \textbf{V}oltage \textbf{S}ource (\ref{fig:019})
  \item \textbf{C}urrent \textbf{C}ontrolled \textbf{V}oltage \textbf{S}ource (\ref{fig:020})
  \item \textbf{V}oltage \textbf{C}ontrolled \textbf{C}urrent \textbf{S}ource (\ref{fig:021})
  \item \textbf{C}urrent \textbf{C}ontrolled \textbf{C}urrent \textbf{S}ource (\ref{fig:022})
\end{itemize}

\subsection{Resistors}
\label{ssec:resistors}

The most basic kind of resistor is just a fixed resistor (as seen in Figure \ref{fig:024}). These resistors will have some set value of $R$ that does not change.

\begin{figure}[H]
  \centering
  \includestandalone{figures/fig_024}
  \caption{Fixed Resistor}
  \label{fig:024}
\end{figure}

There also exist variable resistors (also called potentiometers). These resistors can by manipulated manually, or through the design of a circuit, to have variable resistances. (Figure \ref{fig:025})

\begin{figure}[H]
  \centering
  \includestandalone{figures/fig_025}
  \caption{Variable Resistor}
  \label{fig:025}
\end{figure}

\end{document}
