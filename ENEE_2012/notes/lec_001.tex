\documentclass[12pt]{article}

\input{../../xlatex/imports/preamble}

\title{Lecture 001}
\date{January 06, 2025}

\begin{document}

\section{Basic Concepts}
\label{sec:basicConcepts}

\subsection{Charge and Current}
\label{ssec:chargeAndCurrent}

\begin{definition}{Charge}
  Charge ($Q$) is an electrical property of the atomic particles of which matter consists, measures in coulombs ($C$).
\end{definition}

Throughout this course, it is generally the charge of an electron  ($e$) that will be considered.
\begin{equation*}
  e = -1.602 \cdot 10^{-19}\ C
\end{equation*}
Thus, if there is some known quantity of electrons $n$, the total charge of those electrons can be calculated:
\begin{equation*}
  Q = n \cdot e
\end{equation*}

Current relates closely with charge, being a measurement of the movement of charge (or electrons).

\begin{definition}{Current}
  Electric current ($I$) is the rate of change of charge, measured in amperes ($A$).
\end{definition}

Since current is a rate of change (over time) of charge, current and charge relate as each other's (anti)derivative.

\begin{equation*}
  I(t) = \frac{dQ}{dt}\ A \ \ \ \ \ \ \ \ \ \ \ \ \ \ \ Q(t) = \int_{}^{} I(t) \,dt \ C
\end{equation*}

\subsection{Direct vs. Alternating Current}
\label{ssec:directVsAlternatingCurrent}

\subsubsection{Direct Current}
\label{sssec:directCurrent}

\begin{definition}{Direct Current}
  A direct current (dc) flows only in one direction and can be constant or time varying.
\end{definition}

\begin{wrapfigure}[5]{r}{0.4\textwidth}
  \centering
  \begin{subfigure}[H]{0.15\textwidth}
    \centering
    \includestandalone{figures/fig_001}
    \caption{Conventional}
    \label{fig:001}
  \end{subfigure}
  \begin{subfigure}[H]{0.15\textwidth}
    \centering
    \includestandalone{figures/fig_002}
    \caption{Electron}
    \label{fig:002}
  \end{subfigure}
  \caption{Electron Flow in Direct Currents}
  \label{fig:electronFlowInDirectCurrent}
\end{wrapfigure}

There are two ways of describing the \textit{direction} in which the electrons flow in a direct current: \textbf{conventional flow} and \textbf{electron flow}. Both are shown in Figure \ref{fig:electronFlowInDirectCurrent}.

\subsubsection{Alternating Current}
\label{sssec:alternatingCurrent}

Direct current flow isn't the only way current can flow. Some currents utilize \textbf{Alternating Current (AC)}.

\begin{definition}{Alternating Current}
  An alternating current (ac) is a current that changes direction with respect to time.
\end{definition}

\subsection{Voltage}
\label{ssec:voltage}

\begin{definition}{Voltage}
  Voltage (or \textit{potential difference}) is the \textit{energy required} to move a unit charge from a reference point (-) to another point (+), measured in Volts ($V$).
\end{definition}

The potential different that voltage measures is a literal different in potential between two points in a circuit. As seen in Figure \ref{fig:potentialDifference}, the voltage from $a$ to $b$ is different than the voltage from $b$ to $a$.
\begin{figure}[H]
  \centering
  \begin{subfigure}[H]{0.45\textwidth}
    \centering
    \includestandalone{figures/fig_004}
  \end{subfigure}
  \begin{subfigure}[H]{0.45\textwidth}
    \centering
    \includestandalone{figures/fig_005}
  \end{subfigure}
  \caption{Potential Difference}
  \label{fig:potentialDifference}
\end{figure}
The energy required to move an object is expressed in Joules ($J$), and remains consistent with measurements of energy to move regular objects like a elevator up a shaft. Since voltage is the energy per unit charge, it can be expressed as:
\begin{equation*}
  v(t) = v_b - v_b = \frac{dw}{dQ} = \frac{dE}{dQ}
\end{equation*}
Where $w$ is work and $E$ is energy.

\subsection{Power and Energy}
\label{ssec:powerAndEnergy}

\begin{definition}{Power}
  Power ($P$) is the rate of change of energy measured in watts ($W$).
\end{definition}
Previously, it's been seen that current is the rate of change of charge (\ref{ssec:chargeAndCurrent}), and voltage is the amount of energy required to move charge (\ref{ssec:voltage}). Putting these two ideas together, it follows that power can be expressed as the product of current and voltage:
\begin{equation*}
  P = I \cdot V
\end{equation*}
Power is the rate of change of charge multiplied by the amount of energy required to move some charge. Another way of expressing this in terms of calculus is:
\begin{align*}
  P(t) &= \frac{dE}{dt} = \frac{dE \cdot dQ}{dt \cdot dQ} = \frac{dE}{dQ} \cdot \frac{dQ}{dt} \\
\end{align*}
Where
\begin{equation*}
  V(t) = \frac{dE}{dQ}\ \ \ \ \ \textup{and}\ \ \ \ \ I(t) = \frac{dQ}{dt}
\end{equation*}

\begin{definition}{Energy}
  Energy is the capacity to do work measured in Joules ($J$).
  \begin{equation*}
    E(t) = \int_{}^{} P(t) \, dt
  \end{equation*}
\end{definition}

Currents follow the \textbf{Law of Conservation of Energy}. This means that the total change in energy within a closed circuit must sum to zero:
\begin{equation*}
  \sum_{}^{} P = 0
\end{equation*}
Thus, the total power supplied to a circuit must be equal to the total power absorbed by that circuit.

The difference between supplying and absorbing energy is a matter of convention and does not matter given that it remains consistent throughout the full analysis of a circuit. Generally, the \textbf{passive sign convention} is used.

\begin{definition}{Passive Sign Convention}
  Passive sign convention is satisfied when the current enterts through the positive terminal of an element and $P=+V \cdot I$. If the current enters through the negative terminal, $P = -V \cdot I$.
\end{definition}

\subsection{Categorization of Components}
\label{ssec:categorizationOfComponents}

\subsubsection{Passive and Active Components}
\label{sssec:passiveAndActiveComponents}

Rather than trying to understand full circuits at once, it is easier to break them into two groups: \textbf{passive} components and \textbf{active} components.

\begin{definition}{Passive Components}
  Passive components are ones that don't require any power supply to operate. For example, a resistor or a capacitor are both passive components.
\end{definition}

\begin{definition}{Active Circuits}
  Active components require power to operate. In other words, they need to be connected to a power supply to function. Logic gates (74LSXX) are active since they require a power supply.
\end{definition}

\subsubsection{Linear vs. Non-Linear Components}
\label{sssec:linearVsNonlinearComponents}

Another way to divide components is between \textbf{linear} and \textbf{non-linear}.

\begin{definition}{Linear Components}
  Linear components are... They also can be subdivided into components that store energy (capacitors and inductors) and components that dissipate energy (resistors).
\end{definition}

\begin{definition}{Non-Linear Components}
  ...
\end{definition}
\subsection{Circuit Elements}
\label{ssec:circuitElements}

\begin{definition}{Ideal Independent Source}
  An ideal independent source is an active element that provides a specified voltage or current that is completely independent of other circuit elements.
\end{definition}

\begin{definition}{Ideal Dependent Source}
  Also called an \textit{Ideal Controlled Source}, this is an active element in which the source quantity is controlled by another voltage or current.
\end{definition}

% TODO: include figures of each type as wrapfig

There are four types of dependent sources:
\begin{enumerate}
  \itemsep0em
  \item Voltage Controlled Voltage Source (VCVS)
  \item Voltage Controlled Current Source (VCCS)
  \item Current Controlled Voltage Source (CCVS)
  \item Current Controlled Current Source (CCCS)
\end{enumerate}

\subsection{Basic Laws}
\label{ssec:basicLaws}

\subsubsection{Ohm's Law}
\label{sssec:ohmsLaw}

\textbf{Ohm's Law} states that the voltage $V$ across a resistor is \textit{directly} proportional to the current $I$ flowing through the resistor.

\begin{formula}{Ohm's Law}
  \begin{equation*}
    R = \rho \cdot \frac{l}{A}
  \end{equation*}
\end{formula}

When there is current flowing through a wire with resistance approaching zero, a \textbf{short circuit} is created. Conversely, an \textbf{open circuit} is where the resistance in a circuit approaches infinity.

Resistance and conductance are inversely related. It is the ability of an element of conduct electric current; it is measured in mhos or siemens ($S$).

\subsubsection{Nodes, Branches, and Loops}
\label{sssec:nodesBranchesLoops}

\begin{definition}{Branch}
  Represents a single element in a circuit such as a resistor or power supply.
\end{definition}

\begin{definition}{Node}
  The point of connection between two or more branches.
\end{definition}

\begin{definition}{Loop}
  A loop is any \textit{closed} path in a circuit. Generally, loops are defined as the smallest possible path.
\end{definition}

By analyzing the nodes connection branches of a circuit, elements can be found to be in parallel or in series.

\begin{definition}{Parallel}
  Elements are in parallel if they share two nodes.
\end{definition}

\begin{definition}{Series}
  Elements are in series if they \textit{exclusively} share a node.
\end{definition}

It is possible for elements to be neither in series or in parallel. These situations aren't problematic, but require somewhat different techniques to analyze.

\subsubsection{Kirchhoff's Laws}
\label{sssec:kirchoffsLaws}

\begin{definition}{Kirchhoff's Current Law}
  Kirchhoff's Current Law (KCL) states that the algebraic sum of currents entering a node (or a closed boundary) is zero:
  \begin{equation*}
    \sum_{n=1}^{N_{branch}} i_n = 0
  \end{equation*}
\end{definition}

\begin{definition}{Kirchhoff's Voltage Law}
  Kirchhoff's Voltage Law (KVL) states that the algebraic sum of all voltages around a closed path is zero:
  \begin{equation*}
    \sum_{m=1}^{M_{branch}} v_m = 0
  \end{equation*}
\end{definition}

\end{document}
