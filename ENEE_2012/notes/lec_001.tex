\documentclass[12pt]{article}

%%%% GRAPHICS %%%%
\usepackage{tikz}
\usepackage[siunitx, american, RPvoltages]{circuitikz}
\usetikzlibrary{arrows.meta}
\usepackage{tikz-3dplot}
\usepackage{graphicx}
\usepackage{pgfplots}
  \pgfplotsset{compat=1.18}
\usetikzlibrary{arrows}
\newcommand{\midarrow}{\tikz \draw[-triangle 90] (0,0) -- +(.1,0);}

%%%% FIGURES %%%%
\usepackage{subcaption}
\usepackage{wrapfig}
\usepackage{float}
\usepackage[skip=5pt, font=footnotesize]{caption}

%%%% FORMATTING %%%%
\usepackage{parskip}
\usepackage{tcolorbox}
\usepackage{ulem}

%%%% TABLE FORMATTING %%%%
\usepackage{tabularray}
\UseTblrLibrary{booktabs}

%%%% MATH AND LOGIC %%%%
\usepackage{xifthen}
\usepackage{amsmath}
\usepackage{amssymb}
\usepackage{amsfonts}

%%%% TEXT AND SYMBOLS %%%%
\usepackage[T1]{fontenc}
\usepackage{textcomp}
\usepackage{gensymb}

%%%% OTHER %%%%
\usepackage{standalone}

%%%% LOGIC SYMBOLS %%%%
\newcommand*\xor{\oplus}

%%%% STYLES %%%%

% Packages
\usepackage[paper=letterpaper,tmargin=45pt,bmargin=45pt,lmargin=45pt,rmargin=45pt]{geometry}
\usepackage{titlesec}
\usepackage[rgb]{xcolor}
\selectcolormodel{natural}
\usepackage{ninecolors}
\selectcolormodel{rgb}

% Colors
\definecolor{pg}{HTML}{24273A}
\definecolor{fg}{HTML}{FFFFFF}
\definecolor{bg}{HTML}{24273A}
\definecolor{re}{HTML}{F38BA8}
\definecolor{gr}{HTML}{A6E3A1}
\definecolor{ye}{HTML}{F9E2AF}
\definecolor{or}{HTML}{FAB387}
\definecolor{bl}{HTML}{89B4FA}
\definecolor{ma}{HTML}{CBA6F7}
\definecolor{cy}{HTML}{94E2D5}
\definecolor{pi}{HTML}{F2CDCD}

\definecolor{copper}{HTML}{B87333}

\usepackage{nameref}
\makeatletter
\newcommand*{\currentname}{\@currentlabelname}
\makeatother

\titleformat{\section}
  {\normalfont\scshape\Large\bfseries}
  {\thesection}
  {0.75em}
  {}

\titleformat{\subsection}
  {\normalfont\scshape\large\bfseries}
  {\thesubsection}
  {0.75em}
  {}

\titleformat{\subsubsection}
  {\normalfont\scshape\normalsize\bfseries}
  {\thesubsubsection}
  {0.75em}
  {}

% Formula
\newcounter{formula}[section]
\newenvironment{formula}[1]{
  \stepcounter{formula}
  \begin{tcolorbox}[
    standard jigsaw, % Allows opacity
    colframe={fg},
    boxrule=1px,
    colback=bg,
    opacityback=0,
    sharp corners,
    sidebyside,
    righthand width=18px,
    coltext={fg}
  ]
  \centering
  \textbf{\uline{#1}}
}{
  \tcblower
  \textbf{\thesection.\theformula}
  \end{tcolorbox}
}

% Definition
\newcounter{definition}[section]

\newenvironment{definition*}[1]{
  \begin{tcolorbox}[
    standard jigsaw, % Allows opacity
    colframe={fg},
    boxrule=1px,
    colback=bg,
    opacityback=0,
    sharp corners,
    coltext={fg}
  ]
  \textbf{#1 \hfill}
  \vspace{5px}
  \hrule
  \vspace{5px}
  \noindent
}{
  \end{tcolorbox}
}

\newenvironment{definition}[1]{
  \stepcounter{definition}
  \begin{tcolorbox}[
    standard jigsaw, % Allows opacity
    colframe={fg},
    boxrule=1px,
    colback=bg,
    opacityback=0,
    sharp corners,
    coltext={fg}
  ]
  \textbf{#1 \hfill \thesection.\thedefinition}
  \vspace{5px}
  \hrule
  \vspace{5px}
  \noindent
}{
  \end{tcolorbox}
}

% Example Problem
\newcounter{example}[section]
\newenvironment{example}{
  \stepcounter{example}
  \begin{tcolorbox}[
    standard jigsaw, % Allows opacity
    colframe={fg},
    boxrule=1px,
    colback=bg,
    opacityback=0,
    sharp corners,
    coltext={fg}
  ]
  \textbf{Example \hfill \thesection.\theexample}
  \vspace{5px}
  \hrule
  \vspace{5px}
  \noindent
}{
  \end{tcolorbox}
}

\tikzset{
  cubeBorder/.style=fg,
  cubeFilling/.style={fg!20!bg, opacity=0.25},
  gridLine/.style={very thin, gray},
  graphLine/.style={-latex, thick, fg},
}

\pgfplotsset{
  basicAxis/.style={
    grid,
    major grid style={line width=.2pt,draw=fg!50!bg},
    axis lines = box,
    axis line style = {line width = 1px},
  }
}

%%%% REFERENCES %%%%
\usepackage{hyperref}
\hypersetup{
  colorlinks  = true,
  linkcolor   = bl,
  anchorcolor = bl,
  citecolor   = bl,
  filecolor   = bl,
  menucolor   = bl,
  runcolor    = bl,
  urlcolor    = bl,
}

\author{Ethan Anthony}
\newcommand*{\equal}{=}


\title{Lecture 001}
\date{January 06, 2025}

\begin{document}

\section{Introduction}
\label{sec:introduction}

\subsection{Categorization of Components}
\label{ssec:categorizationOfComponents}

Rather than trying to understand full circuits at once, it is easier to break them into two groups: \textbf{passive} components and \textbf{active} components.

\begin{definition}{Passive Components}
  Passive components are ones that don't require any power supply to operate. For example, a resistor or a capacitor are both passive components.
\end{definition}

\begin{definition}{Active Circuits}
  Active components require power to operate. In other words, they need to be connected to a power supply to function. Logic gates (74LSXX) are active since they require a power supply.
\end{definition}

\vspace{12pt}
\hrule
\vspace{08pt}

Another way to divide components is between \textbf{linear} and \textbf{non-linear}.

\begin{definition}{Linear Components}
  Linear components are... They also can be subdivided into components that store energy (capacitors and inductors) and components that dissipate energy (resistors).
\end{definition}

\begin{definition}{Non-Linear Components}
  ...
\end{definition}

\subsection{Overview of Concepts}
\label{ssec:overviewOfConcepts}

\subsubsection{Charge and Current}
\label{sssec:chargeAndCurrent}

\begin{definition}{Charge}
  Charge ($Q$) is an electrical property of the atomic particles of which matter consists, measured in coulombs ($C$).
  \begin{equation*}
    Q(t) = \int_{}^{} i(t) \,dt
  \end{equation*}
\end{definition}

When charged particles move in space over time, such as throughout a wire in a circuit, an electrical current is generated.

\begin{definition}{Electric Current}
  Electric current ($I$) is the time rate of change of charge, measured in amperes ($A$).
  \begin{equation*}
    i(t) = \frac{dQ}{dt}
  \end{equation*}
\end{definition}

Current is simply a rate of change of a charge, and thus the two are related as the others' derivative/integral.

\subsubsection{DC vs. AC}
\label{sssec:dcvsac}

\begin{definition}{Direct Current}
  A direct current (dc) flows only in one direction and can be constant or time varying.
\end{definition}

\begin{wrapfigure}[4]{r}{0.4\textwidth}
  \centering
  \begin{subfigure}[H]{0.15\textwidth}
    \centering
    \includestandalone{figures/fig_001}
    \caption{Conventional}
    \label{fig:001}
  \end{subfigure}
  \begin{subfigure}[H]{0.15\textwidth}
    \centering
    \includestandalone{figures/fig_002}
    \caption{Electron}
    \label{fig:002}
  \end{subfigure}
  \caption{Electron Flow in Direct Currents}
  \label{fig:electronFlowInDirectCurrent}
\end{wrapfigure}

There are two ways of describing the \textit{direction} in which the electrons flow in a direct current: \textbf{conventional flow} and \textbf{electron flow}. Both are shown in Figure \ref{fig:electronFlowInDirectCurrent}.

\begin{definition}{Alternating Current}
  An alternating current (ac) is a current that changes direction with respect to time.
\end{definition}

\begin{definition}{Voltage}
  Voltage (or \textit{potential difference}) is the energy required to move a unit charge form a reference point (-) to another point (+), measured in Volts ($V$).
  \begin{equation*}
    v(t) = v_b - v_b = \frac{dw}{dQ} = \frac{dE}{dQ}
  \end{equation*}
\end{definition}

% \begin{wrapfigure}[]{l}{0.3\textwidth}
%   \centering
%   \includestandalone{figures/fig_003}
%   \caption{Finding Voltage}
%   \label{fig:003}
% \end{wrapfigure}

\subsubsection{Power and Energy}
\label{sssec:powerAndEnergy}

\begin{definition}{Power}
  Power is the time rate of expending or absorbing energy measured in watts ($W$). It is the rate of change of energy.
  \begin{align*}
    p(t) &= \frac{dE}{dt} = \frac{dE \cdot dQ}{dt \cdot dQ} = \frac{dE}{dQ} \cdot \frac{dQ}{dt} \\
    p    &= v \cdot i
  \end{align*}
\end{definition}

\begin{definition}{Energy}
  Energy is the capacity to do work measured in Joules ($J$).
  \begin{equation*}
    E(t) = \int_{}^{} p(t) \, dt
  \end{equation*}
\end{definition}

Again, these two values are related as the derivative/integral of the other.

\subsection{Circuit Elements}
\label{ssec:circuitElements}

\begin{definition}{Ideal Independent Source}
  An ideal independent source is an active element that provides a specified voltage or current that is completely independent of other circuit elements.
\end{definition}

\begin{definition}{Ideal Dependent Source}
  Also called an \textit{Ideal Controlled Source}, this is an active element in which the source quantity is controlled by another voltage or current.
\end{definition}

% TODO: include figures of each type as wrapfig

There are four types of dependent sources:
\begin{enumerate}
  \itemsep0em
  \item Voltage Controlled Voltage Source (VCVS)
  \item Voltage Controlled Current Source (VCCS)
  \item Current Controlled Voltage Source (CCVS)
  \item Current Controlled Current Source (CCCS)
\end{enumerate}

\end{document}
