\documentclass[12pt]{article}

%%%% GRAPHICS %%%%
\usepackage{tikz}
\usepackage[siunitx, american, RPvoltages]{circuitikz}
\usetikzlibrary{arrows.meta}
\usepackage{tikz-3dplot}
\usepackage{graphicx}
\usepackage{pgfplots}
  \pgfplotsset{compat=1.18}
\usetikzlibrary{arrows}
\newcommand{\midarrow}{\tikz \draw[-triangle 90] (0,0) -- +(.1,0);}

%%%% FIGURES %%%%
\usepackage{subcaption}
\usepackage{wrapfig}
\usepackage{float}
\usepackage[skip=5pt, font=footnotesize]{caption}

%%%% FORMATTING %%%%
\usepackage{parskip}
\usepackage{tcolorbox}
\usepackage{ulem}

%%%% TABLE FORMATTING %%%%
\usepackage{tabularray}
\UseTblrLibrary{booktabs}

%%%% MATH AND LOGIC %%%%
\usepackage{xifthen}
\usepackage{amsmath}
\usepackage{amssymb}
\usepackage{amsfonts}

%%%% TEXT AND SYMBOLS %%%%
\usepackage[T1]{fontenc}
\usepackage{textcomp}
\usepackage{gensymb}

%%%% OTHER %%%%
\usepackage{standalone}

%%%% LOGIC SYMBOLS %%%%
\newcommand*\xor{\oplus}

%%%% STYLES %%%%

% Packages
\usepackage[paper=letterpaper,tmargin=45pt,bmargin=45pt,lmargin=45pt,rmargin=45pt]{geometry}
\usepackage{titlesec}
\usepackage[rgb]{xcolor}
\selectcolormodel{natural}
\usepackage{ninecolors}
\selectcolormodel{rgb}

% Colors
\definecolor{pg}{HTML}{24273A}
\definecolor{fg}{HTML}{FFFFFF}
\definecolor{bg}{HTML}{24273A}
\definecolor{re}{HTML}{F38BA8}
\definecolor{gr}{HTML}{A6E3A1}
\definecolor{ye}{HTML}{F9E2AF}
\definecolor{or}{HTML}{FAB387}
\definecolor{bl}{HTML}{89B4FA}
\definecolor{ma}{HTML}{CBA6F7}
\definecolor{cy}{HTML}{94E2D5}
\definecolor{pi}{HTML}{F2CDCD}

\definecolor{copper}{HTML}{B87333}

\usepackage{nameref}
\makeatletter
\newcommand*{\currentname}{\@currentlabelname}
\makeatother

\titleformat{\section}
  {\normalfont\scshape\Large\bfseries}
  {\thesection}
  {0.75em}
  {}

\titleformat{\subsection}
  {\normalfont\scshape\large\bfseries}
  {\thesubsection}
  {0.75em}
  {}

\titleformat{\subsubsection}
  {\normalfont\scshape\normalsize\bfseries}
  {\thesubsubsection}
  {0.75em}
  {}

% Formula
\newcounter{formula}[section]
\newenvironment{formula}[1]{
  \stepcounter{formula}
  \begin{tcolorbox}[
    standard jigsaw, % Allows opacity
    colframe={fg},
    boxrule=1px,
    colback=bg,
    opacityback=0,
    sharp corners,
    sidebyside,
    righthand width=18px,
    coltext={fg}
  ]
  \centering
  \textbf{\uline{#1}}
}{
  \tcblower
  \textbf{\thesection.\theformula}
  \end{tcolorbox}
}

% Definition
\newcounter{definition}[section]

\newenvironment{definition*}[1]{
  \begin{tcolorbox}[
    standard jigsaw, % Allows opacity
    colframe={fg},
    boxrule=1px,
    colback=bg,
    opacityback=0,
    sharp corners,
    coltext={fg}
  ]
  \textbf{#1 \hfill}
  \vspace{5px}
  \hrule
  \vspace{5px}
  \noindent
}{
  \end{tcolorbox}
}

\newenvironment{definition}[1]{
  \stepcounter{definition}
  \begin{tcolorbox}[
    standard jigsaw, % Allows opacity
    colframe={fg},
    boxrule=1px,
    colback=bg,
    opacityback=0,
    sharp corners,
    coltext={fg}
  ]
  \textbf{#1 \hfill \thesection.\thedefinition}
  \vspace{5px}
  \hrule
  \vspace{5px}
  \noindent
}{
  \end{tcolorbox}
}

% Example Problem
\newcounter{example}[section]
\newenvironment{example}{
  \stepcounter{example}
  \begin{tcolorbox}[
    standard jigsaw, % Allows opacity
    colframe={fg},
    boxrule=1px,
    colback=bg,
    opacityback=0,
    sharp corners,
    coltext={fg}
  ]
  \textbf{Example \hfill \thesection.\theexample}
  \vspace{5px}
  \hrule
  \vspace{5px}
  \noindent
}{
  \end{tcolorbox}
}

\tikzset{
  cubeBorder/.style=fg,
  cubeFilling/.style={fg!20!bg, opacity=0.25},
  gridLine/.style={very thin, gray},
  graphLine/.style={-latex, thick, fg},
}

\pgfplotsset{
  basicAxis/.style={
    grid,
    major grid style={line width=.2pt,draw=fg!50!bg},
    axis lines = box,
    axis line style = {line width = 1px},
  }
}

%%%% REFERENCES %%%%
\usepackage{hyperref}
\hypersetup{
  colorlinks  = true,
  linkcolor   = bl,
  anchorcolor = bl,
  citecolor   = bl,
  filecolor   = bl,
  menucolor   = bl,
  runcolor    = bl,
  urlcolor    = bl,
}

\author{Ethan Anthony}
\newcommand*{\equal}{=}


\title{Lecture 001}
\date{January 06, 2025}

\begin{document}

\section{Basic Concepts}
\label{sec:basicConcepts}

\subsection{Charge and Current}
\label{ssec:chargeAndCurrent}

\begin{definition}{Charge}
  Charge ($Q$) is an electrical property of the atomic particles of which matter consists, measures in coulombs ($C$).
\end{definition}

Throughout this course, it is generally the charge of an electron  ($e$) that will be considered.
\begin{equation*}
  e = -1.602 \cdot 10^{-19}\ C
\end{equation*}
Thus, if there is some known quantity of electrons $n$, the total charge of those electrons can be calculated:
\begin{equation*}
  Q = n \cdot e
\end{equation*}

Current relates closely with charge, being a measurement of the movement of charge (or electrons).

\begin{definition}{Current}
  Electric current ($I$) is the rate of change of charge, measured in amperes ($A$).
\end{definition}

Since current is a rate of change (over time) of charge, current and charge relate as each other's (anti)derivative.

\begin{equation*}
  I(t) = \frac{dQ}{dt}\ A \ \ \ \ \ \ \ \ \ \ \ \ \ \ \ Q(t) = \int_{}^{} I(t) \,dt \ C
\end{equation*}

\subsection{Direct vs. Alternating Current}
\label{ssec:directVsAlternatingCurrent}

\subsubsection{Direct Current}
\label{sssec:directCurrent}

\begin{definition}{Direct Current}
  A direct current (dc) flows only in one direction and can be constant or time varying.
\end{definition}

\begin{wrapfigure}[5]{r}{0.4\textwidth}
  \centering
  \begin{subfigure}[H]{0.15\textwidth}
    \centering
    \includestandalone{figures/fig_001}
    \caption{Conventional}
    \label{fig:001}
  \end{subfigure}
  \begin{subfigure}[H]{0.15\textwidth}
    \centering
    \includestandalone{figures/fig_002}
    \caption{Electron}
    \label{fig:002}
  \end{subfigure}
  \caption{Electron Flow in Direct Currents}
  \label{fig:electronFlowInDirectCurrent}
\end{wrapfigure}

There are two ways of describing the \textit{direction} in which the electrons flow in a direct current: \textbf{conventional flow} and \textbf{electron flow}. Both are shown in Figure \ref{fig:electronFlowInDirectCurrent}.

\subsubsection{Alternating Current}
\label{sssec:alternatingCurrent}

Direct current flow isn't the only way current can flow. Some currents utilize \textbf{Alternating Current (AC)}.

\begin{definition}{Alternating Current}
  An alternating current (ac) is a current that changes direction with respect to time.
\end{definition}

\subsection{Voltage}
\label{ssec:voltage}

\begin{definition}{Voltage}
  Voltage (or \textit{potential difference}) is the \textit{energy required} to move a unit charge from a reference point (-) to another point (+), measured in Volts ($V$).
\end{definition}

The potential different that voltage measures is a literal different in potential between two points in a circuit. As seen in Figure \ref{fig:potentialDifference}, the voltage from $a$ to $b$ is different than the voltage from $b$ to $a$.
\begin{figure}[H]
  \centering
  \begin{subfigure}[H]{0.45\textwidth}
    \centering
    \includestandalone{figures/fig_004}
  \end{subfigure}
  \begin{subfigure}[H]{0.45\textwidth}
    \centering
    \includestandalone{figures/fig_005}
  \end{subfigure}
  \caption{Potential Difference}
  \label{fig:potentialDifference}
\end{figure}
The energy required to move an object is expressed in Joules ($J$), and remains consistent with measurements of energy to move regular objects like a elevator up a shaft. Since voltage is the energy per unit charge, it can be expressed as:
\begin{equation*}
  v(t) = v_b - v_b = \frac{dw}{dQ} = \frac{dE}{dQ}
\end{equation*}
Where $w$ is work and $E$ is energy.

\subsection{Power and Energy}
\label{ssec:powerAndEnergy}

\begin{definition}{Power}
  Power ($P$) is the rate of change of energy measured in watts ($W$).
\end{definition}
Previously, it's been seen that current is the rate of change of charge (\ref{ssec:chargeAndCurrent}), and voltage is the amount of energy required to move charge (\ref{ssec:voltage}). Putting these two ideas together, it follows that power can be expressed as the product of current and voltage:
\begin{equation*}
  P = I \cdot V
\end{equation*}
Power is the rate of change of charge multiplied by the amount of energy required to move some charge. Another way of expressing this in terms of calculus is:
\begin{align*}
  P(t) &= \frac{dE}{dt} = \frac{dE \cdot dQ}{dt \cdot dQ} = \frac{dE}{dQ} \cdot \frac{dQ}{dt} \\
\end{align*}
Where
\begin{equation*}
  V(t) = \frac{dE}{dQ}\ \ \ \ \ \textup{and}\ \ \ \ \ I(t) = \frac{dQ}{dt}
\end{equation*}

\begin{definition}{Energy}
  Energy is the capacity to do work measured in Joules ($J$).
  \begin{equation*}
    E(t) = \int_{}^{} P(t) \, dt
  \end{equation*}
\end{definition}

Currents follow the \textbf{Law of Conservation of Energy}. This means that the total change in energy within a closed circuit must sum to zero:
\begin{equation*}
  \sum_{}^{} P = 0
\end{equation*}
Thus, the total power supplied to a circuit must be equal to the total power absorbed by that circuit.

The difference between supplying and absorbing energy is a matter of convention and does not matter given that it remains consistent throughout the full analysis of a circuit. Generally, the \textbf{passive sign convention} is used.

\begin{definition}{Passive Sign Convention}
  Passive sign convention is satisfied when the current enterts through the positive terminal of an element and $P=+V \cdot I$. If the current enters through the negative terminal, $P = -V \cdot I$.
\end{definition}

\end{document}
