\documentclass[12pt]{article}

\input{../../xlatex/imports/preamble}

\title{Lecture 001}
\date{January 06, 2025}

\begin{document}

\section{Introduction}
\label{sec:introduction}

\subsection{Categorization of Components}
\label{ssec:categorizationOfComponents}

Rather than trying to understand full circuits at once, it is easier to break them into two groups: \textbf{passive} components and \textbf{active} components.

\begin{definition}{Passive Components}
  Passive components are ones that don't require any power supply to operate. For example, a resistor or a capacitor are both passive components.
\end{definition}

\begin{definition}{Active Circuits}
  Active components require power to operate. In other words, they need to be connected to a power supply to function. Logic gates (74LSXX) are active since they require a power supply.
\end{definition}

\vspace{12pt}
\hrule
\vspace{08pt}

Another way to divide components is between \textbf{linear} and \textbf{non-linear}.

\begin{definition}{Linear Components}
  Linear components are... They also can be subdivided into components that store energy (capacitors and inductors) and components that dissipate energy (resistors).
\end{definition}

\begin{definition}{Non-Linear Components}
  ...
\end{definition}


\end{document}
